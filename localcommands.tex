\newcommand{\appref}[1]{Appendix~\ref{#1}}
\newcommand{\fnref}[1]{Footnote~\ref{#1}} 
\renewcommand{\sectref}[1]{Section~\ref{#1}} 

\newcommand{\0}{{\color{red}0}}

\newenvironment{langscibars}{\begin{axis}[ybar,xtick=data, xticklabels from table={\mydata}{pos}, 
        width  = \textwidth,
	height = .3\textheight,
    	nodes near coords, 
	xtick=data,
	x tick label style={},  
	ymin=0,
	cycle list name=langscicolors
        ]}{\end{axis}}
        
\newcommand{\langscibar}[1]{\addplot+ table [x=i, y=#1] {\mydata};\addlegendentry{#1};}

\newcommand{\langscidata}[1]{\pgfplotstableread{#1}\mydata;}

% \makeatletter
% \let\thetitle\@title
% \let\theauthor\@author 
% \makeatother

% \newcommand{\togglepaper}[1][0]{ 
% %   \bibliography{../localbibliography}
%   \papernote{\scriptsize\normalfont
%     \theauthor.
%     \thetitle. 
%     To appear in: 
%     Change Volume Editor \& in localcommands.tex 
%     Change volume title in localcommands.tex
%     Berlin: Language Science Press. [preliminary page numbering]
%   }
%   \pagenumbering{roman}
%   \setcounter{chapter}{#1}
%   \addtocounter{chapter}{-1}
% }


%add all your local new commands to this file

\newcommand{\smiley}{:)}

\renewbibmacro*{index:name}[5]{%
  \usebibmacro{index:entry}{#1}
    {\iffieldundef{usera}{}{\thefield{usera}\actualoperator}\mkbibindexname{#2}{#3}{#4}{#5}}}

% \newcommand{\noop}[1]{}

	\newcommand{\glossformat}[1]{\textsc{#1}}

	\newcommand{\firstperson}{\glossformat{1}\xspace}
	\newcommand{\secondperson}{\glossformat{2}\xspace}
	\newcommand{\thirdperson}{\glossformat{3}\xspace}
	\newcommand{\an}{\glossformat{an}\xspace}	
	\newcommand{\acc}{\glossformat{acc}\xspace}
	\newcommand{\add}{\glossformat{add}\xspace}
	\newcommand{\adj}{\glossformat{adj}\xspace}
	\newcommand{\aor}{\glossformat{aor}\xspace}
	\newcommand{\correl}{\glossformat{correl}\xspace}
	\newcommand{\dat}{\glossformat{dat}\xspace}
	\newcommand{\discp}{\glossformat{discp}\xspace}
	\newcommand{\fem}{\glossformat{f}\xspace}
	\newcommand{\gen}{\glossformat{gen}\xspace}
	\newcommand{\gerund}{\glossformat{gerund}\xspace}
	\newcommand{\hab}{\glossformat{hab}\xspace}
	\newcommand{\illa}{\glossformat{illa}\xspace}	
	\newcommand{\imp}{\glossformat{imp}\xspace}
	\newcommand{\impr}{\glossformat{impr}\xspace}	
	\newcommand{\incl}{\glossformat{incl}\xspace}	
	\newcommand{\infv}{\glossformat{inf}\xspace}
	\newcommand{\ins}{\glossformat{ins}\xspace}
	\newcommand{\intenp}{\glossformat{intenp}\xspace}
	\newcommand{\ipfv}{\glossformat{ipfv}\xspace}
	\newcommand{\KonjI}{\glossformat{KonjI}\xspace}
	\newcommand{\KonjII}{\glossformat{KonjII}\xspace}
	\newcommand{\negation}{\glossformat{neg}\xspace}
	\newcommand{\nom}{\glossformat{nom}\xspace}
	\newcommand{\lptcp}{l-\glossformat{ptcp}\xspace}
	\newcommand{\masc}{\glossformat{m}\xspace}
	\newcommand{\n}{\glossformat{n}\xspace}
	\newcommand{\nvir}{\glossformat{n-vir}\xspace}
	\newcommand{\object}{\glossformat{o}\xspace}	
	\newcommand{\passaux}{\glossformat{pass.aux}\xspace}
	\newcommand{\purcomp}{\glossformat{pur.comp}\xspace}
	\newcommand{\pred}{\glossformat{pred}\xspace}	
	\newcommand{\prog}{\glossformat{prog}\xspace}	
	\newcommand{\pst}{\glossformat{pst}\xspace}
	\newcommand{\ptcp}{\glossformat{ptcp}\xspace}
	\newcommand{\pfv}{\glossformat{pfv}\xspace}
	\newcommand{\pl}{\glossformat{pl}\xspace}
% 	\newcommand{\prog}{\glossformat{prog}\xspace}	
	\newcommand{\refl}{\glossformat{refl}\xspace}
    \newcommand{\rep}{\glossformat{rep}\xspace}
	\newcommand{\comp}{\glossformat{comp}\xspace}	
	\newcommand{\subj}{\glossformat{subj}\xspace}	
	\newcommand{\sg}{\glossformat{sg}\xspace}
	\newcommand{\topi}{\glossformat{top}\xspace}	
	\newcommand{\vir}{\glossformat{vir}\xspace}
	\newcommand{\vptcl}{\glossformat{vptcl}\xspace}

	\newcommand{\quelle}[1]{\hfill(#1)}
	\newcommand{\nquelle}[1]{\newline\phantom{x}\hfill(#1)}	
	
	\newcommand{\glhead}[1]{#1:\vspace{-6pt}}

	\usetikzlibrary{calc}
	\newcommand{\movesquare}[4][-1]{\draw [thick,black,->] let \p{E} = (#2), \p{D} = (#3), \p{M} = ($(#2) + (0,#1)$) in (#2) -- (\x{M},\y{M}) -- node [label,above] {#4} (\x{D},\y{M}) -- (#3)} 

% Eigene Befehle
\newcommand{\exemph}[1]{\textbf{#1}} % BeispeilÃŒberschrift
\newcommand{\exhead}[1]{\textbf{#1}} % Hervorhebung bei Beispielen
\newcommand{\gcat}[1]{\textsc{#1}}   % Kategorien in Glossen

\newcommand{\Sem}[1]{\ensuremath{⟦#1⟧}}
