% !BIB program = biber
\documentclass[output=paper,modfonts,nonflat,citecolor=brown,
% hidelinks,
showindex
]{langsci/langscibook} 
\author{Chiara Gianollo\affiliation{Università di Bologna}}
\title{Grammaticalization parameters and the retrieval of alternatives: Latin {\em{nec}} from discourse connector to uninterpretable feature} 
\abstract{By means of the study of Latin focus-sensitive negation {\em{nec}} (`furthermore not'; `neither'; `not even'), I address a more general question on the scope and the cyclicity of semantic change. I review Lehmann's syntagmatic parameters of grammaticalization (structural scope, bondedness, syntagmatic variability) with the aim to evaluate to what extent they are reflected in some types of semantic change. With {\em{nec}} we see, from Latin to Romance, the evolution of a discourse-structuring particle with an additive component into the building block of new emphatic (scalar) negative polarity items, which in turn are later reanalyzed as elements of Negative Concord (endowed with uninterpretable formal features). I argue that an important aspect of this change concerns the way alternatives to the focused element are retrieved in the context. I propose that increase in bondedness and decrease in syntagmatic variability correlate with a change in the form taken by alternatives, which decrease in scope from discourse units to individual alternatives.}

%move the following commands to the "local..." files of the master project when integrating this chapter
\usepackage{tabulary}
\usepackage{langsci-gb4e}
\usepackage{langsci-optional}
\bibliography{localbibliography}
%CG: what follows are packages that I added and need for correct formatting
\usepackage{multirow} %to merge cells vertically in a table
%\usepackage{tabulary} % to wrap text in tables
\usepackage[normalem]{ulem}
\usepackage{amsmath} %avoid italic in math
\usepackage{framed}
\usepackage{tipa} %allows stacking diacritics
\usepackage{qtree}
\usepackage{bbding} %for \Checkmark and \XSolid

\IfFileExists{../localcommands.tex}{%hack to check whether this is being compiled as part of a collection or standalone
   
% % my packages:
% \usepackage{linguex}
% % begin ----------------for (1') labels in linguex
% \usepackage{refcount}
% \makeatletter
% \newcommand{\Vref}[2]{% #1 is the label, #2 is the modifier
%   \begingroup\edef\Vref@temp{\endgroup
%     \noexpand\Vref@\getrefnumber{#1}(\getrefnumber{#1})\noexpand\@nil%
%   }(\Vref@temp#2)%
% }
% \def\Vref@ #1(#2)#3\@nil{#2}
% \makeatother
% % end ----------------for (1') labels in linguex

\usepackage[normalem]{ulem} % for strike through text
\usepackage{csquotes} %for block-paragraph quotes
\usepackage{tablefootnote}
\usepackage{graphicx}

\usepackage{tabularx} 

\usepackage{listings}
\lstset{basicstyle=\ttfamily,tabsize=2,breaklines=true}

\usepackage{./langsci/styles/langsci-optional}
\usepackage{./langsci/styles/langsci-lgr}
\makeatletter
\let\pgfmathModX=\pgfmathMod@
\usepackage{pgfplots,pgfplotstable}%
\let\pgfmathMod@=\pgfmathModX
\makeatother

\definecolor{lsDOIGray}{cmyk}{0,0,0,0.45}

\usepackage{xassoccnt}
\newcounter{realpage}
\DeclareAssociatedCounters{page}{realpage}
\AtBeginDocument{%
  \stepcounter{realpage}
}

%move the following commands to the "local..." files of the master project when integrating this chapter
\usepackage{tabulary}

%CG: what follows are packages that I added and need for correct formatting
\usepackage{multirow} %to merge cells vertically in a table
\usepackage{multicol}
\usepackage{longtable}
%\usepackage{tabulary} % to wrap text in tables
\usepackage{framed}
\usepackage{pifont}

\usepackage{covington}
\usepackage[linguistics]{forest}

\usepackage{stmaryrd}
\usepackage{subdepth}
\usepackage[version=3]{mhchem}

%\setotherlanguage{german, polish}

\usepackage{soul}
\usepackage{xspace}

\usepackage{siunitx}
\sisetup{output-decimal-marker={.},detect-weight=true, detect-family=true, detect-all, input-symbols={\%}, free-standing-units, input-open-uncertainty= , input-close-uncertainty= ,table-align-text-pre=false,uncertainty-separator={\,},group-digits=false,detect-inline-weight=math}
\DeclareSIUnit[number-unit-product={}]{\percent}{\%}
\makeatletter \def\new@fontshape{} \makeatother
\robustify\bfseries % For detect weight to work

\usepackage{langsci-gb4e}

   \newcommand{\appref}[1]{Appendix \ref{#1}}
\newcommand{\fnref}[1]{Footnote \ref{#1}} 

\newenvironment{langscibars}{\begin{axis}[ybar,xtick=data, xticklabels from table={\mydata}{pos}, 
        width  = \textwidth,
	height = .3\textheight,
    	nodes near coords, 
	xtick=data,
	x tick label style={},  
	ymin=0,
	cycle list name=langscicolors
        ]}{\end{axis}}
        
\newcommand{\langscibar}[1]{\addplot+ table [x=i, y=#1] {\mydata};\addlegendentry{#1};}

\newcommand{\langscidata}[1]{\pgfplotstableread{#1}\mydata;}

% \makeatletter
% \let\thetitle\@title
% \let\theauthor\@author 
% \makeatother

% \newcommand{\togglepaper}[1][0]{ 
% %   \bibliography{../localbibliography}
%   \papernote{\scriptsize\normalfont
%     \theauthor.
%     \thetitle. 
%     To appear in: 
%     Change Volume Editor \& in localcommands.tex 
%     Change volume title in localcommands.tex
%     Berlin: Language Science Press. [preliminary page numbering]
%   }
%   \pagenumbering{roman}
%   \setcounter{chapter}{#1}
%   \addtocounter{chapter}{-1}
% }


%add all your local new commands to this file

\newcommand{\smiley}{:)}

\renewbibmacro*{index:name}[5]{%
  \usebibmacro{index:entry}{#1}
    {\iffieldundef{usera}{}{\thefield{usera}\actualoperator}\mkbibindexname{#2}{#3}{#4}{#5}}}

% \newcommand{\noop}[1]{}

\newcommand{\exemph}[1]{\textbf{#1}} % BeispeilÃŒberschrift
	\newcommand{\exhead}[1]{\textbf{#1}} % Hervorhebung bei Beispielen
	\newcommand{\gcat}[1]{\textsc{#1}}   % Kategorien in Glossen


	\newcommand{\glossformat}[1]{\textsc{#1}}

	\newcommand{\firstperson}{\glossformat{1}\xspace}
	\newcommand{\secondperson}{\glossformat{2}\xspace}
	\newcommand{\thirdperson}{\glossformat{3}\xspace}
	\newcommand{\acc}{\glossformat{acc}\xspace}
	\newcommand{\adj}{\glossformat{adj}\xspace}
	\newcommand{\aor}{\glossformat{aor}\xspace}
	\newcommand{\correl}{\glossformat{correl}\xspace}
	\newcommand{\dat}{\glossformat{dat}\xspace}
	\newcommand{\discp}{\glossformat{discp}\xspace}
	\newcommand{\fem}{\glossformat{f}\xspace}
	\newcommand{\gen}{\glossformat{gen}\xspace}
	\newcommand{\gerund}{\glossformat{gerund}\xspace}
	\newcommand{\hab}{\glossformat{hab}\xspace}
	\newcommand{\imp}{\glossformat{imp}\xspace}
	\newcommand{\infv}{\glossformat{inf}\xspace}
	\newcommand{\ins}{\glossformat{ins}\xspace}
	\newcommand{\intenp}{\glossformat{intenp}\xspace}
	\newcommand{\ipfv}{\glossformat{ipfv}\xspace}
	\newcommand{\KonjI}{\glossformat{KonjI}\xspace}
	\newcommand{\KonjII}{\glossformat{KonjII}\xspace}
	\newcommand{\negation}{\glossformat{neg}\xspace}
	\newcommand{\nom}{\glossformat{nom}\xspace}
	\newcommand{\lptcp}{\glossformat{\textit{l}-ptcp}\xspace}
	\newcommand{\masc}{\glossformat{m}\xspace}
	\newcommand{\n}{\glossformat{n}\xspace}
	\newcommand{\nvir}{\glossformat{n-vir}\xspace}
	\newcommand{\passaux}{\glossformat{pass.aux}\xspace}
	\newcommand{\purcomp}{\glossformat{pur.comp}\xspace}
	\newcommand{\pst}{\glossformat{pst}\xspace}
	\newcommand{\ptcp}{\glossformat{ptcp}\xspace}
	\newcommand{\pfv}{\glossformat{pfv}\xspace}
	\newcommand{\pl}{\glossformat{pl}\xspace}
	\newcommand{\refl}{\glossformat{refl}\xspace}
	\newcommand{\sg}{\glossformat{sg}\xspace}
	\newcommand{\vir}{\glossformat{vir}\xspace}
	\newcommand{\vptcl}{\glossformat{vptcl}\xspace}

	\newcommand{\quelle}[1]{\hfill(#1)}
	\newcommand{\nquelle}[1]{\newline\phantom{x}\hfill(#1)}	
	
	\newcommand{\glhead}[1]{#1:\vspace{-6pt}}

	\usetikzlibrary{calc}
	\newcommand{\movesquare}[4][-1]{\draw [thick,black,->] let \p{E} = (#2), \p{D} = (#3), \p{M} = ($(#2) + (0,#1)$) in (#2) -- (\x{M},\y{M}) -- node [label,above] {#4} (\x{D},\y{M}) -- (#3)} 

 
    \togglepaper[23]
}{}

\begin{document}

\maketitle

\section{Introduction}

Recent formal research on semantic change has dealt in particular with change affecting elements of the functional lexicon, and has already provided a number of significant generalizations on the way diachronic phenomena of this kind are triggered and develop over time. Some of these generalizations confirm and sharpen observations that had previously emerged from typological research and, in particular, from the investigation of grammaticalization. 

I adopt a very general, theory-neutral definition of grammaticalization as ``a process which may not only change a lexical into a grammatical item, but may also shift an item `from a less grammatical to a more grammatical status' , in Kury\l{}owicz's words" (\citealt[13]{Lehmann15}).

Grammaticalization provides important insights to formal approaches to diachronic semantics. This is due, on the one hand, to the fact that grammaticalization phenomena follow systematic trajectories and, thus, disclose regularities and general mechanisms of language change. On the other hand, grammaticalization phenomena are multidimensional, in the sense that they involve various linguistic levels and require the simultaneous consideration of morphosyntactic, semantic and pragmatic factors.

Research on grammaticalization unanimously acknowledges the existence of systematicity in grammaticalization phenomena, although the evaluation of the forms and extent of such systematicity vary considerably across frameworks, and often involve a radical discussion of the notion of grammaticalization itself (cf. the contributions in \citealt{Campbell01} (ed.) for discussion).

In typological research on grammaticalization, the way generalizations have been formulated is clearly influenced by the intrinsically multidimensional nature of grammaticalization phenomena: structural as well as semantic factors are encompassed, and often no clear-cut distinction is drawn between them. For formal approaches, this raises the question of how to distinguish which linguistic modules, and consequently which factors within them, are responsible for the observed regularities. 

Structural factors have more readily lent themselves to individuation: \citet[]{Lehmann15} (whose first version appeared in 1982) singled out a number of paradigmatic and syntagmatic parameters of grammaticalization, and generative research uncovered recurrent mechanisms, such as the reanalysis of movement dependencies as local merge relations, or of phrasal elements as heads (cf. \citealt[]{RobertsRoussou03, Gelderen04a} for a comprehensive discussion). 

A still open question concerns the possibility of singling out similar general mechanisms affecting the semantic component, in grammaticalization as well as in other phenomena, and of expressing them in a formal theory: Eckardt's (\citeyear{Eckardt06}) seminal study has paved the way for this kind of research, which has already yielded significant results (cf. \citealt[]{Eckardt12, Deo15, Gianolloetal15} for an overview). 

As has been the case with formal diachronic syntax, in order to reach an answer it is necessary to collect a consistent amount of cross-linguistic evidence by means of empirical research. The present study is an attempt in this direction: my aim is to provide an analysis of the diachronic path followed by the Latin particle {\em{nec}} (`furthermore not'; `neither'; `not even') and its Romance continuations, and to compare the emerging generalizations about the involved semantic trajectory with those formulated with respect to the structural aspects of change. I analyze {\em{nec}} as a focus particle in all its functions, and I derive its different uses, and their diachronic distribution, from the way alternative meanings to the focus associate are computed and retrieved in context. I then propose that the format of the changes observed in this respect is comparable to Lehmann's (\citeyear[]{Lehmann15}) syntagmatic parameters of grammaticalization, capturing this way important parallelisms between the syntactic and the semantic side of context-dependence.  

The discussion in this chapter largely abstracts away from the broader debate on the nature of grammaticalization, and focuses on well-attested systematic diachronic tendencies, which I regard as part of a grammaticalization process (but remain empirically valid even if they are not considered specific to grammaticalization), and which are argued to affect in a parallel fashion the morphosyntactic and the semantic-pragmatic components.

The structure of the chapter is as follows: in section \ref{generalintro} I provide a first description of the functions and of the diachronic development of {\em{nec}}, and I single out the reasons why I believe this case study to be particularly relevant for our more general understanding of semantic change. Section \ref{distributionfunctions} is dedicated to a more in-depth analysis of the particle's various functions. In section \ref{sectionanalysis} an analysis of the steps involved in the grammaticalization path is provided. In section \ref{sectionconclusions} I compare the  conclusions emerging from the case study to Lehmann's (\citeyear{Lehmann15}) syntagmatic parameters for grammaticalization, showing the existence of clear correlations between structural and meaning change in grammaticalization, and I summarize the main conclusions reached. 

\section{Latin {\em{nec}} from discourse connector to uninterpretable feature} \label{generalintro}

Thanks to the uninterrupted and rich documentation on Latin and its Romance descendants, it is possible to follow the history of {\em{nec}} for millennia and to see how this element developed multiple functions: some of them coexist since the most ancient texts, some others represent later developments; at least one of these functions is uniformly continued in Romance, whereas others were lost in all or in some Romance languages. Section \ref{corefacts} gives a first overview of these functions and of their diachronic distribution. Section \ref{theoreticalrelevance} comments on the theoretical relevance of the case study.
 

\subsection{A first overview} \label{corefacts}

The etymology of the Latin particle {\em{nec}} is straightforward: it derives from the Indo-European negative morpheme *{\em{n\u{e}}} and the enclitic conjunction -{\em{que}}, yielding {\em{neque}}.{\footnote{The etymological facts are complicated only by the occurrence, in Archaic Latin, of {\em{nec}} in a usage that does not fall into the canonical functions of the particle and that disappears at later stages, namely the expression of plain sentential negation with no apparent correlative function. Scholars tend towards an explanation in terms of an etymologically different particle in these cases: \citet[29-30]{OrlandiniPoccetti07} defend a deictic origin, motivated as negation strengthening, for the element {\em{-c}} (\textless *{\em{ke}}, cf. Latin {\em{hic}} `this') in this archaic particle.}} The form {\em{nec}} is derived by apocope of the last syllable of {\em{neque}}: the two forms coexist and are functionally equivalent in Early and Classical Latin. For simplicity, I mostly refer to {\em{nec}} because it is the most pertinacious form from a diachronic point of view, {\em{neque}} becoming rarer in Late Latin texts and being continued in Romance only by Romanian.

The negative particle *{\em{n\u{e}}} is continued in Latin only in univerbation with other elements. It yields negative indefinites, such as e.g. {\em{nemo}} `nobody' (\textless \ {\em{n\u{e}}} + {\em{homo}} `man'); {\em{nullus}} `no' (\textless \ {\em{n\u{e}}} + {\em{ullus}} diminutive of `one'). It is also at the core of the negative marker {\em{n\=on}} `not' (\textless \ {\em{n\u{e}}} + {\em{oenum}} `one'), originating through a process of reinforcement (\citealt[]{Fruyt08a, Gianollo18}), and of other connectors, such as e.g. {\em{nisi}} `if not', {\em{ne...quidem}} `neither, not even'. It also appears as negative morpheme in verbs, such as {\em{n\u{e}scio}} `ignore' (\textless \ {\em{n\u{e}}} + {\em{scio}} `know'), {\em{nolo}} `not-want' (\textless \ {\em{n\u{e}}} + {\em{volo}} `want'), etc.{\footnote{The particle {\em{n\=e}} `lest', the negation used a.o. in prohibitions and as negative complementizer in purpose clauses, has a different etymological origin, as evidenced by the long vowel that characterizes it. For discussion of its controversial etymology cf. \citet{deVaan}.}}

The clitic conjunction -{\em{que}} is employed for the coordination of various types of constituents in Latin; its syntactic distribution is constrained by its postpositive nature, cf. (\ref{postpositiveque}).{\footnote{In the Latin examples, the glosses follow the Leipzig Glossing Rules and are limited to basic morphological information, for the sake of readability (case on nominals and number on verbs; for non-finite forms, information on mood is provided). The abbreviations for Latin authors and works follow the {\em{Thesaurus Linguae Latinae}} (\url{http://www.thesaurus.badw.de/en/user-tools/index/}) Texts are cited according to the editions in Brepols' electronic corpus {\em{Library of Latin Texts - Series A}} (\url{http://www.brepolis.net}).}} The positive counterpart of {\em{neque}} / {\em{nec}} is represented by the pair {\em{atque}} / {\em{ac}} `and (also)'.

{\begin{exe}
\ex \label{postpositiveque} \gll terra mari{\bf{que}} \hfill `on land and sea'\\
land:{\sc{abl}} see:{\sc{abl}}.and\\
\end{exe}}

\noindent The particle {\em{nec}} itself is found in combinations with other particles, yielding complex elements such as {\em{necdum}} `(and) not yet', {\em{necne}} `or not', {\em{necnon}} `and also, and yet' (the latter yielding a positive meaning in conformity to the Double Negation system of Latin).

Latin {\em{nec}} is a multifunctional element (\citealt[]{Orlandini01, OrlandiniPoccetti07}). Three main functions can be singled out: 

{\begin{exe}
\ex \label{listfunctionsnec} Functions of Latin {\em{nec}}
\begin{xlist}
\ex (i) discourse-structuring connector `and not'; `furthermore, it is not the case that', at the beginning of new textual units;
\ex (ii) correlative particle `neither'...`nor';
\ex (iii) stand-alone focus particle with an additive (`also not') or a scalar (`not even') interpretation. 
\end{xlist}
\end{exe}}

\noindent The examples in (\ref{firstdiscoursenec}-\ref{firstfocusnec}) show {\em{nec}} in the functions listed in (\ref{listfunctionsnec}). Each function will be described in more detail in Section 2, where I will spell out the criteria for contextual disambiguation. For now, it is sufficient to remark that {\em{nec}} is always intrinsically negative. In the clearest examples that show its function as discourse-structuring connector, it performs a polarity switch with respect to a positive antecedent, cf. (\ref{firstdiscoursenec}).

{\begin{exe}
\ex (i) discourse-structuring connector 

\label{firstdiscoursenec} \gll Accessum est ad Britanniam omnibus navibus meridiano fere tempore, {\bf{neque}} in eo loco hostis est visus.\\
approached be:{\sc{3sg}} to Britannia all:{\sc{abl}} ships:{\sc{abl}} midday:{\sc{abl}} around time:{\sc{abl}}, and.not in that place enemy:{\sc{nom}} be:{\sc{3sg}} seen\\

`All ships got into Britain at around noon, {\bf{and}} {\bf{no}} enemy was spotted there.' (Caes.{\em{ Gall.}} 5.8)

\end{exe}}

{\begin{exe}
\ex (ii) correlative particle 

\label{firstcorrelativenec} \gll nam postquam exercitui praeesse coeperat, {\bf{neque}} terra {\bf{neque}} mari hostes pares esse potuerant.\\
for after army:{\sc{dat}} command:{\sc{inf}} start:{\sc{3sg}} and.not land:{\sc{abl}} and.not sea:{\sc{abl}} enemy:{\sc{nom}} equal:{\sc{nom}} be:{\sc{inf}} can:{\sc{3pl}}\\

`And after he (Alcibiades) started to be the army commader, the enemies could not compete, neither by land nor by sea'
(Nep. {\em{Alc.}} 6.2)
\end{exe}}

{\begin{exe}
\ex (iii) stand-alone focus particle 

\label{firstfocusnec} \gll Veteres quattuor omnino servavere per totidem mundi partes {- ideo} {\bf{nec}} Homerus plures nominat\\
ancient:{\sc{nom}} four altoghether observe:{\sc{3pl}} for as.many world:{\sc{gen}} part:{\sc{acc}} therefore and.not Homer:{\sc{nom}} more:{\sc{acc}} mention:{\sc{3sg}}\\

`The ancients reckoned only four (winds) corresponding to the four parts of the world - and also Homer does not mention more' (Plin. {\em{nat.}} 2.119)
\end{exe}}

\noindent Functions (i) and (ii) are historically primary, and are attested since the beginning of the textual tradition. Their respective fate is quite different. Function (i) is not productively continued in Romance (although it shows some signs of retention in Old Romance, this usage is infelicitous in Modern Romance). Function (ii), instead, is continued by all Standard Romance languages (e.g. French and Spanish {\em{ni}}, Italian {\em{n\'e}}, Romanian {\em{nici}}). As for function (iii), it is only sporadically attested in Early and Classical Latin, and gains in significance only later (1st cent. CE), first with an additive and then with a scalar meaning, which becomes very frequent in Late Latin (from the 3rd-4th cent. CE). 

Function (iii) is continued to various degrees in Romance. As a focus particle, {\em{nec}} typically undergoes a cycle of reinforcement of the additive / scalar component: cf. e.g. Spanish {\em{ni siquiera}} `not even' (originally: `not even if you wish'),{\footnote{Spanish {\em{ni}} can also be used by itself, without {\em{siquiera}}, cf. \citet[]{Aranovich06}.}} Portuguese {\em{nem mesmo}} `not even' (originally: `not even itself'), Italian {\em{neppure, neanche}} `neither, not even' ({\em{ne-}} + multifunctional particle {\em{pure}} `also, though' or {\em{anche}} `also'), Romanian {\em{nici mac\u{a}r}} `not even' (originally: `not even if you wish'); alternatively, it is substituted by another element (French {\em{m\^eme pas}} `not even', originally `itself not'). 

Function (iii) also motivates the employ of {\em{nec}} as negative morpheme in many newly grammaticalized Romance indefinites that become elements of Negative Concord (n-words), like e.g. Spanish {\em{ninguno}}, Portuguese {\em{nenhum}}, Old French {\em{neuns}}, Italian {\em{nessuno}} `nobody'. In fact, if a Romance n-word is negatively marked, the negative morpheme always derives from {\em{nec}}.{\footnote{For a more detailed analysis of the etymological origin of these indefinites, which sometimes contain further building blocks (e.g. {\em{ipse}} `himself' in Italian {\em{nessuno}}) and vary in the retention of the velar component of {\em{nec}}, see \citet[225-228]{Gianollo18}.}} 

This latter outcome is indicated with (iv) in the table in (\ref{overviewtable}), which provides an overview of the diachronic distribution of the various functions. A further, pervasive change, which is omitted from the table for readability, concerns the reanalysis of the negative feature carried by {\em{nec}}, which is reanalyzed from a semantic feature [Neg] in Latin into a formal uninterpretable feature [uNeg], according to the general change from a Double Negation to a Negative Concord system from Latin to Old Romance (\citealt[chapters 4-5]{Gianollo18}). The table just indicates that the newly grammaticalized indefinites containing {\em{nec}} are elements of Negative Concord (i.e., [uNeg] indefinites) since the beginning.

{\begin{exe}
\ex \label{overviewtable} {\em{nec}}: overview of the diachronic developments
\end{exe}}

{\renewcommand{\arraystretch}{1.5} %to have more space between lines
\renewcommand{\tabcolsep}{0.2cm} %to have more space between lines

\begin{tabularx} {350pt}{ l l l l }
{\sc{Function}} & {\sc{Latin}} & {\sc{Old Romance}} & {\sc{Mod. Romance}} \\
\cline{1-4}
(i) discourse connector & \Checkmark & receding & \XSolid \\
(ii) correlative particle & \Checkmark & \Checkmark & \Checkmark \\
(iii.a) focus part. - additive & \Checkmark & \Checkmark (reinforced) & \Checkmark (reinforced) \\
(iii.b) focus part. - scalar & \Checkmark & \Checkmark (reinforced) & \Checkmark (reinforced) \\
(iv) morpheme of indef. & \XSolid & \Checkmark [uNeg] & \Checkmark  [uNeg] \\
\end{tabularx} 
}

\mbox{}

\noindent As exemplification of the Modern Romance outcomes, consider the data from Italian in (\ref{overviewita}): from a morphological point of view, Latin {\em{nec}} is most directly continued by the correlative particle {\em{n\'e}} in Italian. This particle is unacceptable as discourse connector (function (i)). It is typically used in correlative structures (function (ii)), but cannot be used in function (ii) to join two clauses of different polarity, i.e. to perform a polarity switch, unlike what happens in Latin. For function (iii) the reinforced form {\em{neanche}} may be used as focus particle, and is ambiguous between and additive and a scalar reading.{\footnote{For the additive value of {\em{anche}} cf. \citet[]{Francoetal16b}; also further reinforced forms exist, e.g. {\em{neppure}}, {\em{nemmeno}}, whose diachronic development deserves to be studied more carefully in future research.}} Finally, a continuation of Latin {\em{nec}} is visible in the initial morpheme of the word for `nobody', {\em{nessuno}}: in fact, the morphological makeup of the indefinite is not transparent for Modern Italian speakers, but the originally negative element can still be attributed the function of carrying a formal uninterpretable feature for negation, which allows it to enter Negative Concord (see further section \ref{necwordssection}). 

{\begin{exe}
\ex \label{overviewita} Italian
\begin{xlist}
\ex \label{polswitchfirstex} (i) {\em{discourse particle}}\\
Maria \`e andata al supermercato. * {\bf{N\'e}} ha ricordato di portare la borsa.\\
`Maria went to the supermarket. {\sc{N\'e}} she remembered to bring the bag.'
\ex (ii) {\em{correlative particle}}\\
Maria non ha comprato {\bf{n\'e}} il latte {\bf{n\'e}} il burro.\\
`Maria bought neither milk nor butter'
\ex (iii) {\em{focus particle}}\\
Maria non ha comprato {\bf{neanche}} i biscotti.\\ 
`Maria didn't buy cookies either' / `Maria didn't even buy cookies'
\ex (iv) {\em{morpheme of indefinite}}\\
Maria non ha fatto {\bf{ne}}ssun progresso.\\
`Maria didn't make any progress'
\end{xlist}
\end{exe}}

\subsection{Broader theoretical relevance of the case study} \label{theoreticalrelevance}

The table in (\ref{overviewtable}) gives us a first impression of the remarkable diachronic path followed by {\em{nec}}. We see it starting as an element operating on discourse units, and ending up as a word-internal component (a morpheme and eventually a functional feature). In the development from Latin to Romance, {\em{nec}} turns out to be diachronically pertinacious, but at the same time it undergoes a wide-ranging grammaticalization process affecting its multifunctionality. This process can be understood as a form of {\em{functional enrichment}} that `depletes lexical items of their semantic and interpreted features and eventually reduces them to purely functional elements with only uninterpreted features' (\citealt[73]{Kiparsky15}). At the same time, however, we see some of the original functions coexisting with the newly developed ones.  

The history of Latin {\em{nec}}, thus, raises a number of issues for our theoretical understanding of semantic change: how are the different functions related? What determines whether and how these functions coexist at a certain stage? Why are some functions lost and others newly developed? And does the shape of this grammaticalization process tell us something more general on possible formats of semantic change?

In the following sections I try to provide at least partial answers to these questions. First of all, I account for the multifunctionality of the particle: I analyze the functions of {\em{nec}} synchronically and diachronically, and, capitalizing on the bimorphemic nature ({\em{ne-c}}, {\em{ne-que}}) of the particle, I propose that, across functions, it shares a homogeneous internal syntactic structure, corresponding to its two basic semantic components: additivity and negation. 

I further show how the various uses can be derived from the interaction between these two operators and the surrounding structure into which the particle is merged. The focus-sensitive nature of the particle, i.e. its sensitivity to alternatives, is held responsible for its multifunctionality: the structural position of the particle influences its pragmatic properties, in terms of the form of the evoked alternatives and the way they are retrieved. 

On the one hand, the mechanism governing the retrieval of alternatives is involved in the development of a scalar reading for the particle (pragmatic enrichment), which is an intrinsic possibility for additive particles but seems to gain ground in Latin only after a functional competitor, the particle {\em{ne...quidem}}, falls out of use.  

On the other hand, increase in bondedness and decrease in syntagmatic variability witnessed in the diachrony of {\em{nec}} correlate with a change in the form taken by alternatives, which decrease in scope from discourse units to individual alternatives ordered on a scale. In the development of {\em{nec}} we see, thus, the semantic-pragmatic counterpart of Lehmann's syntagmatic parameters of grammaticalization, resulting in decrease or loss of discourse-dependence.  

\section{The functions of {\em{nec}}: distribution} \label{distributionfunctions}

In this section we will have a closer look at the various functions of Latin {\em{nec}}. I will focus on data from Classical Latin (1st cent. BCE - 1st cent. CE), but occasionally also Late Latin data (3rd-4th cent. CE) will be taken into consideration, in order to show the functional extension that the particle undergoes already during the history of Latin. Note that, as seen in section \ref{corefacts}, the two forms of the particle, {\em{neque}} and {\em{nec}}, are functionally indistinguishable in the texts on which I base my conclusions.

\subsection{Discourse-structuring connector}

As a discourse-structuring connector, {\em{nec}} introduces a full clause belonging to a new discourse unit, which may be connected in the discourse to a previous clause independent of the polarity of the latter. Latin is a Double Negation language: each negatively marked element introduces a semantic negative operator, independently of its position in the clause (\citealt[]{Gianollo16}). The particle {\em{neque}} / {\em{nec}} conforms to this system and typically suffices to negate a clause (or a smaller constituent) by itself.

In (\ref{discoursenecNumidae}) the clause preceding the one introduced by {\em{neque}} has positive polarity. The particle marks the subsequent one for negative polarity. The discourse function of the clause introduced by the particle is to bring forward the narration in a temporal progression: {\em{neque}} therefore connects two clauses expressing a coordinating discourse relation according to \citet[]{Asher93}.

{\begin{exe}
\ex \label{discoursenecNumidae} \gll Concurrunt equites inter se; {\bf{neque}} vero primum impetum nostrorum Numidae ferre potuerunt, sed interfectis circiter CXX reliqui se in castra ad oppidum receperunt\\
clash:{\sc{3pl}} knight:{\sc{nom}} between {\sc{refl}}:{\sc{acc}} and.not indeed first:{\sc{acc}} impact:{\sc{acc}} our:{\sc{gen}} Numidian:{\sc{nom}} resist:{\sc{inf}} can:{\sc{3pl}} but killed:{\sc{abl}} around 120 remaining:{\sc{nom}} {\sc{refl}}:{\sc{acc}} in camp:{\sc{acc}} towards city:{\sc{acc}} withdraw:{\sc{3pl}}\\

`The respective cavalries clashed, but the Numidians were not able to withstand the initial impact of our men. Instead, after about a hundred and twenty were killed, the rest withdrew into the camp next to the city.' (Caes. {\em{civ.}} 2.25)
\end{exe}}

\noindent In some cases the demarcation between the two discourse units is even sharper, and is highlighted by modern editors through the insertion of specific punctuation or even paragraph breaks, as in (\ref{discoursenecparagraph}). Often a contrastive flavor is present, also because of accompanying particles ({\em{vero}} `in fact', {\em{tamen}} `however'), as in both (\ref{discoursenecNumidae}) and (\ref{discoursenecparagraph}).

{\begin{exe}
\ex \label{discoursenecparagraph} \gll [7.4] Qua ex re creverat cum fama tum opibus, magnamque amicitiam sibi cum quibusdam regibus Threciae pepererat. [8.1] {\bf{Neque}} tamen a caritate patriae potuit recedere.\\
\hspace{10 mm} which:{\sc{abl}} from thing:{\sc{abl}} grow:{\sc{3sg}} and reputation:{\sc{abl}} and power:{\sc{abl}} close:{\sc{acc}} friendship:{\sc{acc}} {\sc{refl:dat}} with certain:{\sc{abl}} king:{\sc{abl}} Thracia:{\sc{gen}} procure:{\sc{3sg}} \hspace{10 mm} and.not however from love:{\sc{abl}} fatherland:{\sc{gen}} can:{\sc{3sg}} recede:{\sc{inf}}\\

`Because of this he (Alcibiades) grew in reputation and power, and he procured close friendships with certain Thracian kings. Nonetheless he could never abandon the love for his own country'
(Nep. {\em{Alc.}} 7.4-8.1)
\end{exe}}

\noindent As a discourse particle, {\em{nec}} does not impose any constraint on the polarity of the previous unit, therefore it can perform a polarity switch. This possibility is still attested for {\em{n\'e}} in Old Italian (\ref{OI}), but is lost in Modern Italian (\ref{MI}; cf. also \ref{polswitchfirstex}), cf. \citet{Zanuttini10} for Old Italian, and \citet[]{Doetjes05} for similar Old French uses:{\footnote{The example in (\ref{OI}) is a strict Negative Concord structure, where negation is expressed both on the particle {\em{n\'e}} and on the negative marker {\em{non}}, yielding a single-negation reading. Similar structures are not grammatical in Modern Italian, but are in line with the grammar of Old Italian, which allowed for strict Negative Concord under some conditions (\citealt[]{Garzonio18}, \citealt[chapter 5]{Gianollo18}).}} 

{\begin{exe}
\ex
\begin{xlist}
\ex \label{OI} \gll e perci\'o in mezzo della via l'uccise; {\bf{n\'e}} Catone {\bf{non}} avea podere di difenderlo\\
and thus in middle of.the road him.killed and.not Cato not had faculty of defend.him\\

`and thus he killed him in the middle of the road; and Catone did not have the faculty of defending him' (Old Italian, Brunetto Latini {\em{Rett.}} p. 115 l. 9-10) 
\ex \label{MI} \gll *Francesco \`e andato a mensa {\bf{n\'e}} Giovanni lo ha accompagnato\\
Francesco is gone to mensa and.not Giovanni him has accompanied\\

`Francesco went to the mensa and Giovanni did not go with him' (Modern Italian)
\end{xlist}
\end{exe}}

\noindent The impossibility for Italian {\em{n\'e}} to perform a polarity switch amounts, in my framework, to the loss of the discourse-structuring function for the particle derived from {\em{nec}} in this language (as in the rest of Romance). The same-polarity requirement between the two discourse units, emerging in its diachronic development, results in a usage that is hardly distinguishable from the correlative one (where the conjuncts are subject to stricter parallelism constraints, encompassing polarity).  

\subsection{Correlative negation} \label{functioncorrelativeneg}

As correlative negation, {\em{nec}} co-occurs with another instance of {\em{nec}} or another negative element (e.g. the negative marker {\em{non}} or a negative indefinite) in the same syntactically complex discourse unit. Each negative element contributes a semantic negative operator, in compliance with the Double Negation nature of Latin: two or more propositions are at the same time coordinated and negated. In this use, {\em{nec}} can introduce clauses or smaller syntactic constituents. 

Coordination of full-fledged clauses by means of correlative {\em{nec}} can be seen in the two passages in (\ref{neccoordinatesfullclauses}), respectively with main and with subordinate clauses:

{\begin{exe}
\ex \label{neccoordinatesfullclauses}
\begin{xlist}
\ex \gll {\bf{nec}} satis exaudibam, {\bf{nec}} sermonis fallebar tamen, quae loquerentur\\
and.not enough hear:{\sc{1sg}} and.not conversation:{\sc{gen}} miss:{\sc{1sg.pass}} though which:{\sc{nom}} speak:{\sc{3pl.pass}} \\

`I couldn't hear perfectly what was being said, but I didn't miss the general drift of their conversation' (Plaut. {\em{Epid.}} 239-240)

\ex \gll animus autem solus {\bf{nec}} cum adest {\bf{nec}} cum discessit apparet\\
soul:{\sc{nom}} however alone:{\sc{nom}} and.not when be.present:{\sc{3sg}} ad.not when depart:{\sc{3sg}} appear:{\sc{3sg}}\\

`only the soul remains unseen, both when it is present and when it departs' (Cic. {\em{Cato}} 80)
\end{xlist}
\end{exe}}


\noindent Often, {\em{nec}} apparently coordinates non-clausal constituents (cf. \ref{necconstituentneg}); however, in section \ref{analysiscorrelative} we will see how in fact these structures can be analyzed as cases of ellipsis affecting clausal constituents. 

{\begin{exe}
\ex \label{necconstituentneg} \gll {\bf{neque}} enim obscuris personis {\bf{nec}} parvis in causis res agetur\\
and.not indeed obscure:{\sc{abl}} character:{\sc{abl}} and.not small:{\sc{abl}} in issue:{\sc{abl}} situation:{\sc{nom}} develop:{\sc{3sg.pass}}\\

`for the persons involved are not obscure, nor are the issues trivial'\\(Cic. {\em{fam.}} 3.5.2)
\end{exe}}

\noindent In Latin the particles introducing each of the coordinated element can be identical, like in the Romance languages and unlike in many other languages (e.g. English {\em{neither...nor}}, German {\em{weder...noch}}; cf. \citealt[100-106]{BerniniRamat96} and \citealt{Haspelmath07} for a typological overview). However, in Latin different negative elements correlating with {\em{nec}} are possible as well, cf. (\ref{corrnonnec}):{\footnote{As an anonymous reviewer remarks, these examples often have the flavor of an afterthought: `Whales don't have gills, and dolphins don't have them either'.}}

{\begin{exe}
\ex \label{corrnonnec} \gll branchiae {\bf{non}} sunt ballaenis {\bf{nec}} delphinis\\
gill:{\sc{nom}} not be:{\sc{3pl}} whale:{\sc{dat}} and.not dolphin:{\sc{dat}}\\

`neither whales nor dolphins have gills' (Plin. {\em{nat.}} 9.19)
\end{exe}}

\noindent Examples (\ref{neccoordinatesfullclauses}-\ref{necconstituentneg}) are characterized by syntactic parallelism between the coordinated constituents, which also encompasses their (negative) polarity. However, there are also cases where polarity switch between conjuncts is attested, as in (\ref{polarityswitchcorr}), similarly to what we saw with the discourse-structuring connector:

{\begin{exe}
\ex \label{polarityswitchcorr} \gll eius enim nomine, optimi viri {\bf{nec}} tibi ignoti, maledicebat tibi\\
he:{\sc{gen}} in.fact name:{\sc{abl}} excellent:{\sc{gen}} man:{\sc{gen}} and.not you:{\sc{dat}} unknown:{\sc{gen}} slander:{\sc{3sg}} you:{\sc{dat}}\\

`he slandered you under the name of this man, a excellent person and not unknown to you' (Cic. {\em{Deiot.}} 33, from \citealt[688]{Pinkster15})
\end{exe}}

\noindent Polarity switch uses are often employed to create rhetorical effects: {\em{nec ignotus}} in (\ref{polarityswitchcorr}), where the adjective is marked by the negative prefix {\em{in}}-, yields a litotes, in a structure of `asymmetric coordination' that is known as `epitaxis' and is used to add parenthetical comments (cf. \citealt[]{OrlandiniPoccetti07} and references cited there). 

In many cases of polarity switch a contrastive flavor can be detected, and in these contexts it is not easy to tell apart the discourse-structuring connector use from the correlative use: in annotating the examples, I decided for the latter when the correlated constituents are built in a syntactically parallel way, since this contextual condition ensures by itself cohesion among correlates, with no need for the particle to create such cohesion itself, as is instead the case with the discourse-structuring connector (see \citealt[]{OrlandiniPoccetti07} for further discussion). 

\subsection{Focus particle} \label{functionfocusparticle}

In the stand-alone focus-particle use, no direct correlation with other negative constituents is present; rather, the particle attaches to a sub-clausal constituent (in the clearest cases, to a nominal element), and finds its antecedent not in a syntactically parallel structure, but in the broader discourse context. The particle requires for its interpretation that at least one alternative to the constituent in focus holds in the context (additive interpretation). In later texts a scalar reading emerges, which is typically dependent on general world knowledge (providing the scalar alternatives), rather than on specific contextual conditions. 

In (\ref{focuscuniculos}), {\em{nec}} attaches to the nominal phrase {\em{cuniculos}} `rabbits'. The previous context provides the alternative: the land of Ebusus (Ibiza) is known for driving away snakes; a further blessing of this land is that it does not produce rabbits, which can compromise the harvest: 

{\begin{exe}
\ex \label{focuscuniculos} \gll {\bf{nec}} cuniculos Ebusus gignit, populantes Baliarium messes\\
and.not rabbit:{\sc{acc}} Ebusus:{\sc{nom}} generate:{\sc{3sg}} destroy:{\sc{part.nom}} Balearic:{\sc{gen}} harvest:{\sc{acc}}\\

`Ebusus (the island of Ibiza) neither generates rabbits, which destroy the harvests of the Balearic islands' (Plin. {\em{nat.}} 3.78)
\end{exe}}

\noindent In (\ref{focussirens}), the focus associate is the nominal {\em{Sirenes}} `sirens': here Pliny is discussing birds and is discarding a series of fabulous birds that were mentioned by previous authors; these birds represent the contextually provided alternative set to the focus associate introduced by {\em{nec}}. After saying that he considers birds like {\em{pegasi}} and gryphons as invented ({\em{fabulosos reor}} `I consider them legendary'), Pliny adds a comment expressing similar scepticism about sirens:

{\begin{exe}
\ex \label{focussirens} \gll {\bf{nec}} Sirenes impetraverint fidem\\
and.not siren:{\sc{nom}} obtain:{\sc{3pl}} credit:{\sc{acc}}\\

`Also the sirens cannot obtain great credit with me' (Plin. {\em{nat.}} 10.136)
\end{exe}}

\noindent Cases where the element in focus is not a nominal sub-constituent of the clause, but the whole predication, are more difficult to distinguish from the correlative use, since typically the previous clause directly provides a symmetric alternative, which could be interpreted as a first conjunct. For instance, (\ref{dubiouscorrfoc}) seems an intermediate case:

{\begin{exe}
\ex \label{dubiouscorrfoc} \gll quoniam hoc solum animal ex marinis {\bf{non}} percutiat, sicut {\bf{nec}} e volucribus aquilam\\
since this:{\sc{acc}} only:{\sc{acc}} animal:{\sc{acc}} from marine:{\sc{abl}} not strike:{\sc{3sg}} as and.not from bird:{\sc{abl}} eagle:{\sc{acc}}\\

`since this is the only animal, among the marine ones, which it (the thunder) never strikes; similarly, neither (it strikes) the eagle, among birds' (Plin. {\em{nat.}} 2.146)
\end{exe}}

\noindent It seems that the use as stand-alone focus particle is closely related to the correlative one, but emerges in cases where the syntactic parallelism among alternatives is not obvious. This could of course have had a role in providing bridging contexts (in the sense of \citealt{Heine02}) during the diachronic development, since the focus particle function is attested later than the discourse-structuring and correlative one. 

The use as focus particle is encountered very rarely in Early and Classical Latin texts, and becomes more frequent only from the Imperial age (1st. cent. CE) on. Apart from sporadic attestations in early documentation (typically in cases where {\em{nec}} is accompanied by {\em{saltem}} `at least'), the scalar reading of the focus particle emerges even later, in texts dating to the 3rd-4th century CE. Due to the peculiarities of the Latin documentation, it is difficult to assess whether the focus particle use was a feature of the spoken language, which has a late attestation in our documents only due to prescriptive control during the Classical stage. What emerges quite clearly, though, is that the increase in the use of {\em{nec}} as focus particle correlates with the decrease of its main competitor in this function, the discontinuous particle {\em{ne...quidem}} `neither', `not even' (see \citealt[chapter 7]{Orlandini01}, \citealt[]{Gianollo17} for the functions of this particle and the possible causes for its demise).

An example of {\em{nec}} with a scalar intepretation is given in (\ref{quoniamneccooccur}):

{\begin{exe}
\ex \label{quoniamneccooccur} \gll dico autem vobis quoniam {\bf{nec}} Salomon in omni gloria sua coopertus est sicut unum ex istis\\
say:{\sc{1sg}} yet you:{\sc{dat}} that and.not Salomon:{\sc{nom}} in all:{\sc{abl}} splendor:{\sc{abl}} his:{\sc{abl}} dressed:{\sc{pt}} be:{\sc{3sg}} as one:{\sc{nom}} from this:{\sc{abl}}\\

`Yet I tell you that not even Solomon in all his splendor was dressed like one of these' (Vulg. {\em{Matth.}} 6.29)
\end{exe}}

\noindent Note that Spanish, unlike other Romance languages, can still reproduce this use with {\em{ni}}, the continuation of {\em{nec}}:

{\begin{exe}
\ex Spanish translation of (\ref{quoniamneccooccur}) (Nueva Biblia)\\
Pero les digo que {\bf{ni}} Salom\'on en toda su gloria se visti\'o como uno de ellos.
\end{exe}}

\noindent Many of the negative scalar focus particles seen in Indo-European languages also have an employ as correlative negation (cf. \citealt[chapter 4]{Koenig91}, \citealt[]{Haspelmath07}). As seen in section \ref{corefacts}, Romance languages typically use reinforced forms of the correlative particle in this function, pointing to a cyclical development in which the additive / scalar additive component is formally renewed.

\subsection{Combination with `one'}

The function of {\em{nec}} as scalar focus particle that develops in Late Latin is at the core of its further development into a negative morpheme in the new Romance n-words (i.e., elements of Negative Concord). The combination of {\em{nec}} with the scalar endpoint represented by the cardinal numeral {\em{unus}} `one' emerges in Late Latin as one of the ways to express emphatic negation. An example from a Christian author of the 5th century CE is given in (\ref{necunusconsolator}):

{\begin{exe}
\ex \label{necunusconsolator} \gll Quaesivi consolantem, et non inveni; tot milia saturati, tot milia salvati, infiniti edocti et {\bf{nec}} {\bf{unus}} inventus est mihi consolator\\
look.for:{\sc{1sg}} comforter:{\sc{acc}} and not find:{\sc{1sg}}; so.many thousand:{\sc{nom}} sated:{\sc{nom}} so.many thousand:{\sc{nom}} saved:{\sc{nom}} endless:{\sc{nom}} instructed:{\sc{nom}} and and.not. one:{\sc{nom}} found:{\sc{part.nom}} be:{\sc{3sg}} me:{\sc{dat}} comforter:{\sc{nom}}\\  

`I looked for someone to comfort me, and I did not find him; so many thousands of them sated, so many thousands of them saved, endless ones instructed and not one comforter for me has been found' (Arnob. Iun. {\em{in psalm.}} 68.35)
\end{exe}}

\noindent It is very interesting to remark that in (\ref{necunusconsolator}) {\em{nec}} co-occurs with the conjunction {\em{et}}, showing clearly that the particle in this use has lost its correlative function.

Another interesting fact observed in Late Latin texts is that sometimes, in less controlled registers, the use of the combination {\em{nec unus}} co-occurs with a further marker of negation in a single-negation reading (as in \ref{necunusredundancy}). These structures can be interpreted as a sign of the ongoing development of Negative Concord.

{\begin{exe}
\ex \label{necunusredundancy} \gll et de electis israhel {\bf{non}} dissonuit {\bf{nec}} {\bf{unus}}\\
and from chosen:{\sc{abl}} Israel:{\sc{gen}} not be.dissonant:{\sc{3sg}} and.not one\\ 

`and of the chosen ones of Israel not even one was dissonant' (Aug. {\em{loc. hept.}} 2.102)
\end{exe}}

\noindent Once the change from the Latin Double Negation system to the Romance Negative Concord ones is completed, we see the resulting indefinites, which have emerged through a process of univerbation (e.g. Portuguese {\em{nenhum}}, Spanish {\em{ninguno}}, Italian {\em{nessuno}}, Old French {\em{neuns}}, etc.), behave as n-words, i.e. as indefinites that can both express negation by themselves and co-occur with other negative elements in a single-negation reading. Romance n-words do not necessarily contain {\em{nec}} as its building block (cf. Spanish {\em{nada}}, French {\em{personne}}), but if they do contain a negative morpheme, this morpheme is invariably derived from Latin {\em{nec}}. This pan-Romance phenomenon hints at a pervasive use of the combination of {\em{nec}} and {\em{unus}} in the Late Latin varieties from which the Romance languages derive.

\section{The functions of {\em{nec}}: analysis} \label{sectionanalysis}

In the previous section I traced the evolution of a discourse-structuring particle into the building block of new emphatic (scalar) indefinites, which in Romance behave as elements of Negative Concord. The task of the present section is to provide an analysis of the functions of {\em{nec}} that accounts for the particle's multifunctionality in a parsimonious way, i.e., by assuming a common semantic and structural core and by deriving the various functions as an effect of contextual factors, either at the synchronic level (in the case of simultaneous availability of multiple functions) or at the diachronic level (in the case of reanalysis). The starting observation is the following: in all the functions surveyed above, {\em{nec}} contributes, besides negation, an additive component, whereby alternatives are provided by the broader surrounding discourse or by more local antecedents. 

\subsection{The common core}

The two basic semantic components of {\em{nec}}, additivity and negation, correspond to the two morphemes into which the particle can be analyzed: the negative morpheme {\em{ne-}} and the additive morpheme {\em{-que}} / {\em{-c}}. Assuming the two morphemes to be heads of their respective syntactic projection, I propose that {\em{nec}} is a syntactically complex lexical item, whose internal structure stays the same in all functions: it consists of the projection of a negative operator Op$\lnot$P, topped by the projection of an additive focus operator FocP (see \citealt[]{Gianollo17} for a first version of this proposal).

The resulting structure is shown in (\ref{corestructurenec}). In (\ref{corestructurenec}), Op$\lnot$P is the syntactic projection of [Neg], the semantic feature proposed by \citet[]{Zeijlstra04, Zeijlstra14} to characterize intrinsically negative items in Double Negation systems. The projection of Op$\lnot$P amounts to the instruction `Insert operator' for the interface, and has no further role in the syntactic computation (i.e., it does not enter into syntactic dependencies). 

{\begin{exe}
\ex Structure for {\em{n\u{e}c}} in all its functions \label{corestructurenec}
\end{exe}}


{\Tree [.FocP {} 
                     [.FocP [{-c / -que} ].Foc\0 
                                       [.Op$\lnot$P {} !\qsetw{1in}
                                                     [.Op$\lnot$P [ne- ].Op$\lnot$\0\\{\bf{~$[$Neg$]$}} [ \qroof{~~~~~~~~}.XP  ] !\qsetw{1in} ] !\qsetw{1in} ] !\qsetw{1in} ]  ]}
                                                     
\vspace{1em}

\noindent The FocP projection has a basic additive meaning: I adopt a presuppositional analysis, and assume accordingly that the particle contributes the presupposition that the predication about the element in focus {\em{p}} also holds of at least one of its alternatives {\em{q}} in context C:

\newpage
{\begin{exe}
\ex \label{additiveparticlespresupp} presuppositional analysis for additive particles\\
{\em{also}} p:\\ (1) {\em{p}}\\ (2) presupposition: $\exists$ {\em{q}} $\in$ C $\wedge$ {\em{q}} $\neq$ {\em{p}} 
\end{exe}}


\noindent Polarity switch cases, where the negative polarity of the conjunct introduced by {\em{nec}} contrasts with the positive polarity of the antecedent conjunct, show that the negative operator contributed by {\em{nec}} only takes scope over the conjunct directly introduced by the particle. That is, the additive component outscopes the negation (Additive Focus \textgreater \ Negation).

Now, the surface order of the two elements in {\em{neque}} / {\em{nec}} is the mirror image of their scope relation. The reason resides in prosodic facts governing the distribution of enclitic {\em{-que}} / {\em{-c}}, and more in general word formation in Latin. Enclitic -{\em{que}} is phonologically defective; it is a syntactic head but it is not a phonological word, thus it does not properly align a phonological word with a syntactic head (\citealt[]{AgbayaniGolston10}). As a repair strategy, {\em{ne-}} is raised to the superordinate head (prosodic inversion); {\em{ne-}} is itself proclitic: together, the two elements form a prosodically acceptable unit for Latin. 

FocP and Op$\lnot$P are syncategorematic functional shells: they do not select for a specific category, thus they may attach to elements of various semantic types and of different constituency; the focus associates remain transparent for c-selection (cf. \citealt[120-126]{Cinque99} for the status of negation in this respect, and \citealt[199-203]{BHR14} for conjunctions and other particles).

Because of its focus-sensitivity, {\em{nec}} requires the consideration of alternatives in order to be interpreted. Following \citet{Katzir07, FoxKatzir11}, I assume the generation of focus alternatives to be structure-based: alternatives are obtained by replacing the focused constituent with constituents that are at most as complex as the element in focus. The nature of alternatives therefore depends on the type of the constituent {\em{nec}} combines with. This naturally yields varying meaning effects in the case of a syncategorematic particle like {\em{nec}}.

The basic intuition on which my analysis rests is that the various functions of {\em{nec}} emerge from the interaction of the particle with the surrounding context (both in discourse and in structural terms). As is routinely assumed, contextual interaction determines how the alternatives to the ordinary semantic value are retrieved: alternatives exploited for the interpretation of {\em{nec}} may be represented by preceding discourse units, in the case of the discourse-structuring function, or more locally by preceding clauses within the same discourse unit (sentence topic), in the case of the correlative function. In the stand-alone use as focus particle, no strict syntactic parallelism is required: the additive reading is dependent on the anaphoric retrieval of a suitable discourse referent in the broader discourse context; the scalar reading emerges when such anaphoric link cannot be established and the alternatives are provided by a scale instead. 

In what follows, I more closely review the contextual factors that trigger the various functions of {\em{nec}}.

\subsection{Discourse-structuring connector} \label{analysisdiscuousestructuring}

In accordance with the structure-based mechanism of generation of alternatives, I propose that as a discourse-structuring connector {\em{nec}} `furthermore not' takes a whole discourse unit as its complement. I tentatively assume that the highest projection above the CP is a DIS(course) projection (\citealt{Giorgi15}) where (some) discourse relations are syntactically represented, and that this projection is taken by the particle as one of its arguments.

The specifier of the additive Focus projection contains a phonetically null propositional anaphor, which represents the other argument of the operator expressed by {\em{nec}}. The silent anaphor ensures discourse cohesion by connecting the newly introduced clause to the previous context, thus satisfying the additive presupposition of the particle. A null propositional anaphor is similarly assumed by \citet[22-27]{Poletto14} in her analysis of Old Italian {\em{e}} `and, thus' as discourse particle, and by \citet{Ahn15} in her analysis of {\em{too}} and {\em{either}} (cf. also \citealt[]{Beck06} for the structural representation of an anaphoric element in the presupposition of {\em{again}}). 

In my analysis, thus, discourse-structuring {\em{nec}} is considered as a focus particle operating at the discourse level. The salient alternative satisfying the additive presupposition is a previous discourse unit. No polarity requirement is imposed on it.{\footnote{I leave aside the issue of how to properly derive the interpretation of negation with discourse-structuring {\em{nec}}: although it surfaces high in the structure (discourse- and sentence-initially), the negation is interpreted as plain propositional negation, i.e. it operates at the propositional level, unlike some known cases of `high negation' with a denial interpretation, operating beyond the propositional level.}} 
 
Sentences introduced by {\em{nec}} in its use as discourse particle are never discourse-initial. In this, {\em{nec}} recalls the behavior of so-called one-place {\em{and}}, an adverbial connector according to \citet[]{ZeevatJasinskaja07}: 
 
{\begin{exe}
\ex one-place {\em{and}}:\\
{\bf{And}} John gave him a push (\citealt[their ex. 7]{ZeevatJasinskaja07})
\end{exe}}
 
 \noindent Discourse-structuring {\em{nec}} introduces coordinating discourse relations (List, Narration, Result): like {\em{and}}, {\em{nec}} is used when `the sentence topic of the pivot is abandoned to start dealing with a new topic' (\citealt[325]{ZeevatJasinskaja07}); one-place {\em{and}} `seems to mark a distinct sentence topic under the continued discourse topic' (\citealt[325]{ZeevatJasinskaja07}). In (\ref{firstdiscoursenec}) the general discourse topic is represented by the Romans' expedition in Britain; the sentence topic of the first clause is the arrival, the sentence topic of the second clause, introduced by {\em{nec}}, is the result of the first patrol. In (\ref{discoursenecNumidae}) the general discourse topic is the battle; the sentence topic of the first clause is the clash between the cavalries; the sentence topic of the following clause, introduced by {\em{neque}}, is the result of the confrontation. 
 
The fact that the units connected by {\em{nec}} share the same discourse topic shows that {\em{nec}} obeys the condition of `shared topicality' on additive particles discussed for {\em{too}} in \citet[]{SchwenterWaltereit10}. At the same time, discourse-structuring {\em{nec}} introduces a distinctness requirement (cf. \ref{additiveparticlespresupp}), leading to an update of the Common Ground.

\subsection{Correlative negation} \label{analysiscorrelative}

The common structural and semantic core proposed for {\em{nec}} in (\ref{corestructurenec}) can be maintained for correlative {\em{nec}} once also in this occurrence the particle is analyzed as focus-sensitive. Correlative particles have been accounted for as focus-sensitive particles in a number of works: \citet{Hendriks04}, followed by \citet{denDikken06}, proposes this analysis for English {\em{either}}, {\em{neither}}, {\em{both}}. Similarly, \citet{Wurmbrand08} treats {\em{nor}} in correlative structures as composed by an additive focus particle and a negation.

I thus analyze correlative {\em{nec}} as a focus particle introducing each of the conjuncts; in other words, also in the correlative function the morpheme -{\em{que}} / -{\em{c}} realizes an additive Focus operator, not a conjunction. Correlation between the conjuncts introduced by {\em{nec}} is analyzed as asyndetic coordination, adopting the structure for `edge coordination' (`not only...but also') proposed by \citet{BianchiZamparelli04}. In the structure in (\ref{treestructurecorrelation}), JP stands for Junction Phrase (cf. \citealt{Munn93, denDikken06, Szabolcsi13, MitrovicSauerland14}), the structure responsible for the coordination, whose null head hosts the conjunction operator. 

\newpage

{\begin{exe}
\ex \label{treestructurecorrelation} Syntax of correlation by {\em{nec}}
\end{exe}}

{\Tree
[.JP [.FocP [{-c / -que} ].Foc\0 [.Op$\lnot$P [ne-\\$\textrm{[Neg]}$ ].Op$\lnot$\0 \qroof{~~~~{\em{x}}~~~~}.XP
] ] [.JP [$\wedge$ ].J\0 [.FocP [{-c / -que} ].Foc\0 [.Op$\lnot$P [ne-\\$\textrm{[Neg]}$ ].Op$\lnot$\0 \qroof{~~~~{\em{y}}~~~~}.XP 
] ] !\qsetw{2.5in} ] !\qsetw{1.5in} ]
}



\vspace{0.5em}

\noindent Differently from the discourse-structuring function, the sentence topic does not change across conjuncts. Moreover, the conjuncts in correlative structures are subject to a parallelism constraint: they are parallel in terms of their organization into foreground and background (i.e., they have parallel foci, cf. \citealt[64]{Koenig91}) and, as we observed in section \ref{functioncorrelativeneg}, they are also parallel in terms of their syntactic structure.

That is, in the case of correlative negation, alternatives are provided locally by the correlative construction itself, and originate from the substitution of a sub-constituent of the clause, i.e. the element in focus. In the correlative construction it is particularly clear that the alternatives relevant for the interpretation are generated structurally and obey the complexity constraint discussed by \citet{Katzir07, FoxKatzir11}.

The parallel syntactic construction and the pragmatic role of focus are the factors licensing ellipsis within the conjuncts. In fact, correlates are arguably always propositional units, reduced by ellipsis; a proper treatment of the association of the semantic operators involved must eventually lead to such an analysis. For instance, TP-ellipsis would be involved in cases like (\ref{necconstituentneg}), thus only apparently an instance of constituent negation, as shown in (\ref{ellipsisnecconstituentneg}):

{\begin{exe}
\ex \label{ellipsisnecconstituentneg} cf. (\ref{necconstituentneg}), Cic. {\em{fam.}} 3.5.2:
\gll {\bf{neque}} enim obscuris personis [$_{TP}$ \sout{res} \sout{agetur}] {\bf{nec}} parvis in causis res agetur\\
and.not indeed obscure:{\sc{abl}} character:{\sc{abl}} {} situation:{\sc{nom}} develop:{\sc{3sg.pass}} and.not small:{\sc{abl}} in issue:{\sc{abl}} situation:{\sc{nom}} develop:{\sc{3sg.pass}}\\

`for the persons involved are not obscure, nor are the issues trivial'
\end{exe}}
%eventualmente cambia che e gia nel libro

\noindent Crucially, parallelism seems to extend to the polarity value of the conjunct, at least in the most obvious cases of correlation. Syntactically, this means that the complement of {\em{nec}} must be at least a Polarity Phrase. Same-polarity requirements have often been observed for polarity particles (cf. \citealt[519-520]{Wurmbrand08} for {\em{nor}}); they are also well known for being subject to diachronic change (cf. the observations on {\em{nor}} at earlier stages of English in \citealt[114]{Jespersen17}). They are usually treated as an additional presupposition encoded in the particle's lexical entry. It is not clear whether this would be the right analysis for Latin, though, since, as seen in section \ref{functioncorrelativeneg}, exceptions to the same-polarity requirement do arise in examples where {\em{nec}} has a special rhetorical effect. An alternative hypothesis, safeguarding the correspondence between the lexical entry of the discourse-structuring connector and of the correlative particle, would be not to incorporate the same-polarity presupposition in the lexical entry and to admit that correlative {\em{nec}} may express two different relations, Parallel or Contrast, in the sense of \citet{Asher93}. With Parallel alternatives are required to be of the same (negative) polarity, whereas with Contrast the polarity switch is at the core of the Contrast relation itself. 

The incorporation of the same-polarity presupposition into the lexical entry may represent a later diachronic step, correlating with the general loss of polarity-switch uses, as we saw for Italian in (\ref{OI}-\ref{MI}).

\subsection{Focus particle}

Since I treated correlative {\em{nec}} as a focus particle, the assumption that the stand-alone particle is structurally identical to the correlative particle naturally follows. The only difference concerns the syntactic context in which they are used: the stand-alone focus particle does not have an immediate overt syntactic correlate. Rather, it finds its alternatives in the broader contexts. In this, it is more similar to the discourse-structuring use; however, the form of the alternative(s) differs: while in the discourse-structuring use alternatives are provided by discourse units, in the focus-particle use alternatives are typically represented by propositional alternatives. For example, in (\ref{focuscuniculos}), the proposition `Ebusus does not generate rabbits' is interpreted in the context of its anaphorically available alternative `Ebusus drives away snakes'. 

I assume that this use is in principle always available for a correlative focus particle in virtue of its meaning, but that it may be blocked in some languages by the presence of a more suitable competitor. In Latin such a competitor is represented by the discontinuous particle {\em{ne...quidem}}. In Late Latin this particle loses productivity, thus opening an additional functional space for {\em{nec}} through loss of lexical blocking.

The ambiguity between an additive and a scalar reading for focus particles is cross-linguistically frequent (cf. \citealt[158-159]{Koenig91}, \citealt[24-25]{GastAuwera11}): it is observed, for instance, with German {\em{auch}} and Italian {\em{anche}}, as shown in (\ref{auchanche}).  

{\begin{exe}
\ex \label{auchanche}
\begin{xlist}
\ex {\bf{Auch}} Riesen haben klein angefangen \hfill German (\citealt[62]{Koenig91})\\
`Even giants started from small beginnings' ({\em{auch}} = {\em{sogar}})
\ex {\bf{Anche}} i giganti hanno iniziato in piccolo \hfill Italian\\
`Even giants started from small beginnings' ({\em{anche}} = {\em{perfino}})
\end{xlist}
\end{exe}}

\noindent The way alternatives are retrieved and the form the alternative set takes determine whether the reading for the particle is additive or scalar. In the scalar reading, alternatives ordered along a scale are evoked; the focus denotation is then the extreme of the scale. In the additive reading, instead, alternatives come in an unstructured set. Under the scalar reading, {\em{nec}} corresponds to the negative counterpart of {\em{even}}, whose contribution is schematically represented in (\ref{scalarpresupposition}), to be compared with (\ref{additiveparticlespresupp}): %(where {\em{p}} is the focus associate and {\em{q}} an alternative to {\em{p}} in context C):

{\begin{exe}
\ex \label{scalarpresupposition} presuppositional analysis for scalar particles:\\
{\em{even}} p:\\
(1) {\em{p}}\\
(2) presupposition: $\forall q \in$ C $[q \neq p \rightarrow p <_{\mu} q]$\\
(3) alternatives come in an ordered set, where ${\mu}$: contextually determined probability measure %likelihood scale
\end{exe}} 

\noindent Also in this case I adopt a presuppositional analysis, whereby scalarity originates from the presupposition that the alternative being predicated is striking with respect to some contextually established scale; in (\ref{scalarpresupposition}) I adopt a probability measure, but a scale of informational or pragmatic strength (cf. \citealt{GastAuwera11} for discussion) would work equally well for my purposes. A more fundamental assumption concerns my choice of a scope-based analysis for {\em{even}} to account for the reading obtained when it interacts with negation.{\footnote{For the debate on this issue and alternative, ambiguity-based analysis cf. \citet[]{Rooth85} and the recent proposal by \citet[]{Collins16}.}} The focus operator always takes wide scope with respect to the negative operator. This way scale reversal, operated by negation, obtains before focus applies, satisfying the scalar presupposition of the particle in situations where the complement of {\em{nec}} denotes the most probable (i.e., less striking) element to obtain (as is the case with minimizers or generalizers). In its scalar reading, {\em{nec}} means `even [not {\em{x}}]': it is even the case that the most probable alternative does not hold.

As discussed in section \ref{functionfocusparticle}, the scalar reading emerges later than the additive one, pointing to the fact that a reanalysis takes place, whereby the lexical entry of the focus particle is enriched by the scalar presupposition (a form of pragmatic enrichment in the sense of \citealt{TraugottDasher02}). This would motivate the observed divergence between the fate of the correlative particle (continued by all Romance languages), and the fate of the scalar particle, which often undergoes reinforcement or lexical substitution.

To explain how a scalar reading for {\em{nec}} emerges and subsequently becomes conventionalized, it is important to consider the different way in which alternatives are retrieved in the additive reading and in the scalar reading. This, in turn, influences the structure that the set of alternatives has, as has been shown for Italian {\em{neanche}} `also' by \citet{Tovena06}. The use as additive focus particle is possible only when suitable alternatives for the element in focus are explicitly provided in the context. This happens by means of correlation in the correlative use, and by anaphoric linking to an element in the broader previous discourse in the stand-alone focus particle use. No accommodation is possible (\citealt[]{Zeevat92} and following): the alternative has to be explicitly available in the conversational background, and this explains why additive particles out of the blue are infelicitous:

{\begin{exe}
\ex \symbol{35}  John had dinner in New York {\bf{too}}
\end{exe}}

\noindent That is, additive particles are strictly anaphoric and the lack of a proper antecedent leads to presupposition failure: presupposition accommodation with additive operators is impossible or highly restricted. 

In the absence of these preconditions, only a scalar interpretation is possible: in that case alternatives can be accommodated by evoking a scale, whose dimension is usually suggested by the element in focus: in (\ref{quoniamneccooccur}), `Solomon', the element in focus, suggests a scale of people likely to be splendidly dressed: a king is the most probable option on this scale. 

\citet{Tovena06} shows that this mechanism of accommodation regularly takes place with It. {\em{neanche}}.{\footnote{But see \citet{Umbach12} for the fact that not all scalar additives allow for accommodation, cf. Germ. {\em{noch}}.}} A process of presupposition accommodation by means of scale retrieval may have been responsible for the conventionalization of a scalar meaning for {\em{nec}}: hearer-based accommodation processes are costly and they have been argued to be a frequent trigger to processes of reanalysis driven by economy considerations (\citealt[]{TraugottDasher02, Eckardt06}, \citealt[]{SchwenterWaltereit10}). It is plausible to assume that, once the competitor {\em{ne...quidem}} lost ground and the employ of {\em{nec}} as focus particle became more frequent in Late Latin, the scalar reading originally resulting from accommodation may have ended up being conventionalized, by incorporating a scalar presupposition into the lexical entry of {\em{nec}}.
                                                     
\subsection{Combination with `one': the {\em{nec}}-words} \label{necwordssection}

The last step in the development of {\em{nec}} is represented by its recruitment as a morpheme of the newly created narrow-scope indefinites that I dubbed `{\em{nec}}-words' in previous work (\citealt[chapter 5]{Gianollo18}). While these elements appear to be fully grammaticalized in the earliest Romance documents, in Late Latin we encounter, with increasing frequency, their syntactic source: this has to be identified in structures like (\ref{necunusconsolator}-\ref{necunusredundancy}), where {\em{nec}} syntactically combines with the cardinal numeral {\em{unus}} `one'.

The frequent scalar use of Late Latin {\em{nec}} renders the particle a suitable item to strengthen negation, according to a crosslinguistically frequent pattern which witnesses `even' as a component of polarity-sensitive quantificational expressions (\citealt[]{Haspelmath97, Lahiri98, Watanabe04, Chierchia13}).Therefore, this last step of the grammaticalization process finds its prerequisite in the expansion of scalar uses for {\em{nec}} in Late Latin. Being a natural scalar endpoint, {\em{unus}} `one' represents an optimal strengthening strategy: it is logically entailed by all its alternatives, but, thanks to scale reversal under negation, it yields the most unlikely, informationally strongest proposition. 

On the basis of Lahiri's (\citeyear{Lahiri98}) analysis of Hindi {\em{ek bhii}} `one-even', the meaning of emphatic `even-one' NPIs is formalized by \citet[156]{Chierchia13} as follows, using the probability measure already seen in (\ref{scalarpresupposition}):

{\begin{exe}
\ex
\begin{xlist}
\ex \label{evenoneNPIChierchia} `even-one' NPIs, adapted from \citet[156]{Chierchia13}:\\ 
$\lambda P \lambda Q \ \exists x \ [one(x) \wedge P(x) \wedge Q(x) ]$\\
$\Vert one \Vert^{\sigma-ALT} = \{ \lambda P \lambda Q \ \exists x \ [n(x) \wedge P(x) \wedge Q(x)]: n \geq one \}$
\ex $ E_{ALT}(p) = p \wedge \forall q \in ALT [p <_{\mu} q]$
\end{xlist}
\end{exe}}

\noindent The scalar alternatives ($\sigma$) of {\em{one}} are strictly ordered; given the shape of the alternatives, in Chierchia's framework their exhaustification has to take place by means of the {\em{E}} operator, which corresponds to the meaning of `even'. The result is felicitous only in NPI-licensing contexts.

The Romance version of the indefinite contains, in addition, a negative morpheme, which I take to receive narrow scope with respect to the scalar focus operator.

In Romance, new indefinites formed with {\em{nec}} appear invariably as elements of Negative Concord grammars, i.e. as n-words.{\footnote{These indefinites also show negative-polarity uses, where they have a `positive' meaning in NPI-licensing contexts. I refer to \citet[chapter 5]{Gianollo18} and the literature cited there for attempts to reconcile these uses with the n-word analysis.}} 
According to a prominent line of analysis (\citealt[]{Zeijlstra04, Zeijlstra14}, \citealt[]{Penka11}), n-words are analyzed as bearers of a formal uninterpretable negative feature [uNeg], whose function is to trigger Agree operations, which result in the creation of syntactic dependencies. 

Once the indefinite behaves as an n-word, it is clear that {\em{ne(c)}} has become an uninterpretable feature: it does not directly contribute a semantically interpretable negative operator, but only a morphosyntactic signal to insert one in the further derivation. Moreover, it is clear from the Early Romance data that the focus contribution has disappeared already at that stage: the {\em{nec}}-words behave as plain existentials and do not contribute scalarity.

When {\em{nec}} is grammaticalized as the component of an indefinite, the change involves a restriction in the particle's possibilities to select a focus semantic value: consequently, the alternatives must now be lexically selected by the predicate determining the restriction of the quantification (cf. discussion in \citealt[]{Lahiri98}, \citealt[]{Chierchia13}).

According to \citet[156-157]{Chierchia13}, this explains why a sentence like (\ref{HindivsEnglish}) is ungrammatical in Hindi (where the focus particle is a morphological component of the indefinite {\em{ek bhii}} `even one'), unlike its English counterpart:

{\begin{exe}
\ex \label{HindivsEnglish} from \citet[156-157]{Chierchia13}
\begin{xlist}
\ex \gll {\bf{{*ek}}} {\bf{bhii}} aadmii aayaa\\ 
one even man came\\
`Any (=Even one) man came'
\ex {\bf{Even}} {\bf{one}} man came
\end{xlist}
\end{exe}}

\noindent According to Chierchia, because of the univerbation, in Hindi alternatives are strictly specified by the lexical content of the nominal restriction: therefore, in (\ref{HindivsEnglish}.a) they are lexically restricted to be men. This yields infelicity in an upward-entailing context, because a contradiction arises between the presupposition of the focus particle that the associate be the informationally strongest proposition and the fact that the lexically constrained alternatives (`two men came', `three men came') are actually informationally stronger. In English, instead, alternatives are not lexically constrained: depending on the broader context, it is in principle possible to evoke a scale of alternative denotations to the element in focus, comprising less striking alternatives (`a woman came', `a child came'); this way, the presupposition of {\em{even}} can be satisfied.

Assuming that an analogous mechanism is universally forced by univerbation, I conclude that the same applies to Latin {\em{nec}}-words, resulting in a maximal degree of bondedness of the particle and, consequently, in a context-independent, lexically restricted generation of alternatives.

\section{Conclusions: the grammaticalization path} \label{sectionconclusions}

In this final section I compare the conclusions emerging from the study of {\em{nec}} to the syntagmatic parameters for grammaticalization formulated by \citet{Lehmann15}, with the aim of ascertaining whether what we learn from the diachrony of {\em{nec}} can provide generalizations on the nature of semantic change similar to those proposed for syntactic change.  

Lehmann's (\citeyear[129-134; 152-170]{Lehmann15}) syntagmatic parameters of grammaticalization capture the main effects of grammaticalization processes at the syntactic level. They are listed in (\ref{Lehmannparameters}): 

{\begin{exe}
\ex \label{Lehmannparameters} Lehmann's (\citeyear[152-170]{Lehmann15}) syntagmatic parameters of grammaticalization
\begin{xlist}
\ex {\em{structural scope}}: the structural size of the construction which a grammatical item interacts with (Lehmann: `helps to form'); it decreases during grammaticalization; %e.g. case affix: NP; auxiliary: from sentence-level verb to VP-level; integration of clauses NB: he speaks about syntax; semantic scope may be increased; `the shrinking of structural scope in the course of grammaticalization ends at the stem level'
\ex {\em{bondedness}}: syntagmatic cohesion to another sign, varying from `juxtaposition' to merger; it increases during grammaticalization; %be part of two different XPs, of the same XP, of a head (word boundary, stem boundary) - also prosodic bondedness
\ex {\em{syntagmatic variability}}: degree of variation in combination, positioning, and syntactic dependence with respect to other phrases; it decreases during grammaticalization. %e.g. from adverb to adposition; auxiliaries
\end{xlist}
\end{exe}}

\noindent These syntactic parameters appear to have an interpretational correlate in the case of {\em{nec}}: the multiple functions of {\em{nec}} show varying degrees of discourse-dependence, according to `how far' the particle can look in order to retrieve a suitable alternative. Given the assumed structure-based mechanism for the generation of alternatives, the locus where the particle merges in the structure has a determining effect on the type of the alternatives. Increase in bondedness and decrease in syntagmatic variability with {\em{nec}} correlate with a change in the form taken by alternatives, which decrease in scope (i.e., size) from discourse units to (individual) scalar alternatives (ultimately lexically restricted once bondedness leads to univerbation).

Basing on what was discussed in section \ref{sectionanalysis}, we can formulate a cline for the retrieval of alternatives as in (\ref{clineretrieval}), which reflects the two parameters of structural scope and syntagmatic variability: 

{\begin{exe}
\ex \label{clineretrieval} Cline in the retrieval of alternatives:
\end{exe}}

\noindent across sentence topics \textgreater \ within the same sentence topic \textgreater \ within a scale introduced by the item in focus\\

\noindent When {\em{nec}} is a discourse-structuring connector, the alternatives are represented by the preceding discourse units across sentence topics. Also alternatives to the additive stand-alone focus particle can be found in the broader context, spanning across sentence topics. 

In the correlative negation use, instead, the alternatives are found within the same sentence topic: they are explicitly listed as syntactic conjuncts, joined by means of the Junction Phrase, and are subject to a information-structural and syntactic parallelism condition. This step in the cline is remindful of what \citet{ZeevatJasinskaja07} observe for English {\em{and}}: they propose a uniform analysis of two-place and one-place {\em{and}} as additive particle, and argue for a diachronic grammaticalization path from adverbial connector to clausal conjunction, consisting in the syntacticization of the retrieval of the alternative satisfying the particle's presupposition, which becomes fixed to the first conjunct. In section \ref{analysiscorrelative} I proposed that the loss of polarity-switch uses with the correlative particle may be due to a strengthening of the parallelism requirement as to encompass polarity. In turn, the loss of the discourse-structuring use attested in Romance may be diachronically connected to the loss of polarity switch with the correlative particle, due to a general stricter parallelism requirement on alternatives (that have to contain negation). The same-polarity requirement may be argued to lead to the loss of the contrastive value that the discourse connector could have.

In the innovative use as scalar particle, alternatives are not directly dependent on the surrounding discourse context, since a scale can be accommodated on the basis of world knowledge alone. The crucial ingredient of discourse-dependence is the anaphoric requirement imposed by the additive presupposition. Hence, the loss of this anaphoric requirement when the particle gets reanalyzed as scalar amounts to a decrease in discourse-dependence.

We see, therefore, that the cline in (\ref{clineretrieval}) closely corresponds to a cline of discourse-dependence in the interpretation of the particle, as summarized in (\ref{clinedependence}):

{\begin{exe}
\ex \label{clinedependence} Cline of discourse-dependence (from higher to lower):
\end{exe}}
\noindent discourse-structuring \textgreater \ additive focus marking \textgreater \ correlation \textgreater \ scalar focus marking\\

\noindent The cline in (\ref{clinedependence}) can in fact be argued to motivate the cline in (\ref{clineretrieval}): the loss of context-dependence lies at the core of the grammaticalization process, which in turn entails the decrease of the size of alternatives (decrease in scope and syntagmatic variability, and increase in bondedness).

The use of {\em{nec}} as a morpheme in the new Romance n-words shows a complete absence of discourse-dependence (in the relevant sense), representing the endpoint of the grammaticalization process: an extreme increase in bondedness brings to conclusion the process of functional enrichment, which causes the particle to become a mere morphosyntactic expression of uninterpretable formal features.

To conclude, the main generalization on semantic change emerging from this case study is that the loss of context-dependence is a determining factor in grammaticalization processes involving functional items. The main dimension of context-dependence considered here is the mechanism of retrieval of alternatives, which is dependent on the level of syntactic attachment of the particle. Different degrees of context-dependence depend on the portion of context within which the particle may look for a suitable antecedent. In the case of {\em{nec}}, the loss of the anaphoric requirement linked to its additive semantics emerges as the main trigger towards the decrease in the scope of the alternatives that is connected to the development of a scalar reading.  


\section*{Acknowledgements}
I am grateful to the public of FoDS 2, the editors of this volume, three anonymous reviewers, Mira Ariel, Lieven Danckaert, Katja Jasinskaja, Alan Munn for discussing the content of this chapter with me. Their constructive input helped improve and clarify many aspects of the presentation of data and of the analysis. 

\printbibliography[heading=subbibliography,notkeyword=this]

\end{document}
