\documentclass[output=paper]{langsci/langscibook} 
\title{From a collective to a free choice determiner in Biblical Hebrew}
\author{Edit Doron\affiliation{Hebrew University of Jerusalem}}

\abstract{The paper is a diachronic study of the Hebrew universal determiner \textit{kol}. In Biblical Hebrew (BH), \textit{kol} was originally a noun meaning ‘entirety’ which grammaticalized as a collective determiner akin to \textit{all}. \textit{Kol} induces maximality, like the determiner \textit{all}, but, unlike \textit{all}, it is not quantificational, hence its maximality does not preclude homogeneity. Semantically, \textit{kol} \textit{NP} is interpreted as the plural property corresponding to NP. In argument position, the strongest interpretation of \textit{kol} \textit{NP} results from the application of the definite type-shift (the \textit{iota} type-shift). But within the scope of certain modals and in downward entailing environments, the indefinite type-shift (existential closure) yields a stronger interpretation. This results in the free choice (FC)/ negative polarity (NPI) \textit{any} interpretation of \textit{kol} in these environments. In post-Biblical times, the \textit{any} interpretation evolved into the distributive interpretation \textit{every}. The paper thus traces the development of \textit{kol}’s extensive meaning variation ‘all\slash any\slash every’.}

\begin{document}
\maketitle


\section{Introduction}
% % 1 

How does universal quantification develop in a language? \citet{Haspelmath1995}  suggests that collective universal determiners (like English \textit{all}) often originate in an adjective meaning \textit{entire/whole}, and that distributive universal determiners (such as English \textit{every}) have various sources – free choice (FC) determiners like \textit{any}, or collective universal determiners like \textit{all}. The [FC ${\rightarrow}$ distributive] development was elucidated by \citet{Beck2017}, and here I would like to describe the [collective ${\rightarrow}$ distributive] development. I claim that at least for some languages, the latter development is a cycle which includes the former, as shown in \REF{ex:doron:1} below:\footnote{\textrm{The cyclical nature of \REF{ex:doron:1} is due to its reversibily} (cf. \citet{Gelderen2011} on the pervasive nature of cyclical change). \textrm{\textit{Every}} \textrm{in present-day English} has completed the Distributivity Cycle and is acquiring a collective interpretation, as in \textit{Everyone} \textit{gathered} \textit{in} \textit{the} \textit{hall,} by re-entering the cycle.}  

\ea%1
    \label{ex:doron:1}
            collective universal determiner  ${\rightarrow}$   FC determiner  ${\rightarrow}$  distributive universal determiner               
\z


In \REF{ex:doron:2} I add the original first step, where an Adj/Noun meaning \textit{entire(ty)} \linebreak evolves into a collective universal determiner:

\ea%2
    The Distributivity Cycle\label{ex:doron:2}\\
    \gll\relax                        {}                        I                 {}                    II         {}          III\\
                   {Adj/Noun \textit{entire(ty)}}  ${\rightarrow}$  {collective univ. det.}  
                   ${\rightarrow}$   {FC det.}  ${\rightarrow}$  {distributive univ. det.} \\
    \z

I will motivate the Distributivity Cycle on the basis of the history of Hebrew. Steps I + II took place in Biblical Hebrew (BH): The BH noun \textit{kol} ‘entirety’ grammaticalized as the collective determiner \textit{all}, and did not have a distributive meaning other than as a free choice (FC)/ negative polarity (NPI) determiner akin to \textit{any}. Modern Hebrew (and probably much earlier) underwent step III, whereby FC \textit{kol} also came to have the universal distributive meaning \textit{every}.\footnote{Hence Hebrew conforms to Haspelmath’s view on the direction of development from ‘any’ to ‘every’ rather than the other way round, despite his own assessment of Hebrew as a counterexample (\citealt{Haspelmath1997}:156 fn.13).}  The present analysis thus accounts for the surprising array of interpretations ‘all/ any/ every’ of \textit{kol} in Modern Hebrew without alleging that \textit{kol} is existential rather than universal (\citealt{LevMargulis2013}).\footnote{The existential analysis of \textit{kol} in Modern Hebrew was applied to the structure \textit{kol} \textit{NP} with a predicate NP. The partitive \textit{kol} \textit{DP} is undisputedly universal in Modern Hebrew, casting doubt on the existential analysis of \textit{kol}. I return to Modern Hebrew at the end of the article, in section 6. For now, I note that the root \textit{kll} of \textit{kol} (and the related roots \textit{klkl,} \textit{kwl,} \textit{kly}) derive a plethora of nouns and verbs denoting completeness, containment, inclusiveness and generality. Biblical Hebrew has \textit{kālā} ‘to complete (intrans.)’, \textit{killa} ‘to complete (trans.)’, \textit{kalīl} ‘completely’, \textit{hēḵīl} ‘to contain’, \textit{klī} ‘container’, \textit{kāl} ‘to measure’, \textit{kilkēl} ‘to contain/ sustain’. Later periods innovated \textit{klal} ‘whole’, \textit{klali} ‘general’, \textit{biḵlal} ‘at all’, \textit{miḵlol} ‘ensemble’, \textit{tḵula} ‘content’, \textit{kalal} ‘to include’, \textit{kolel} ‘including’, \textit{hiḵlil} ‘to generalize’, \textit{haḵlala} ‘generalization’. Not a single noun or verb derived from \textit{kll} in any period of Hebrew has an existential interpretation. These factors have motivated the analysis of \textit{kol} as universal (\citealt{DoronMittwoch1986}, \citealt{Glinert1989}, \citealt{FrancezGoldring2012}, \citealt{Danon2013}).} 

The structure of the paper is the following. \ref{sec:doron:2} shows that BH should be classified as a NP (rather than a DP) language. \sectref{sec:doron:2.1} argues that BH had no definite (or indefinite) determiner. \sectref{sec:doron:2.2.} demonstrates that the determiner \textit{kol} was originally a noun – it had both the morphology and the distribution of other nouns in the language. \textit{kol} was often found heading the pseudo-partitive construction, and accordingly underwent an independent-to-functional meaning-shift which grammaticalized it as the determiner \textit{all}. \sectref{sec:doron:3} discusses the semantic properties of the determiner \textit{kol}. \ref{sec:doron:3.1} shows that it was not distributive – it was never interpreted as \textit{every}. Sections 3.2 and 3.3 discuss maximality and homogeneity, and show that \textit{kol}’s homogeneity did not result in the lack of maximality which would be expected by \citet{Križ2016}. \ref{sec:doron:4} describes the operator \textit{each} which was responsible for distributivity in BH. \ref{sec:doron:5} discusses the emergence of the free choice (FC) interpretation of \textit{kol} within the scope of certain modal operators. \ref{sec:doron:6} briefly relates the post-Biblical development whereby the FC reading gave rise to a distributive reading. This development is not elaborated in the present paper, relying on \citet{Beck2017}. \ref{sec:doron:7} is the conclusion.

\section{Biblical Hebrew as a NP-language}\label{sec:doron:2}%2
Biblical Hebrew (BH) did not have a distributive universal determiner. This has been claimed for other languages as well, e.g. Salish (\citealt{Jelinek1993}; \citealt{Davis2010}; \citealt{DavisEtAl2014};  \citealt{FintelMatthewson2008}; \citealt{Matthewson2001}, 2014). Yet BH did not just lack a distributive universal determiner, but other determiners as well. According to the typology of \citet{Bošković2008}, BH is an NP-language (in contrast to DP-languages). To derive the interpretation of NPs in argument position, BH makes use of type-shifts, in particular the \textit{definite} \textit{type-shift} (the \textit{iota} type-shift) and the \textit{indefinite} \textit{type-shift} (existential closure). This accords with the fact that BH is a language without either a definite determiner or an indefinite determiner, and hence relies on the corresponding type-shifts instead. This is the topic of the next subsection.

\subsection{The BH definite article as an inflectional prefix}\label{sec:doron:2.1}%2.1
As argued by \citet{DoronMeir2013,DoronMeir2016}, the Hebrew article \textit{han-}, though glossed as \textit{the-}, is historically not a D but a word-level inflectional prefix.\footnote{See \citet{Rubin2005}: 65 for the history of the article \textit{han-}. Phonological processes delete its final /\textit{n}/, resulting in the prefix \textit{hā-}, or assimilate /\textit{n}/ to the first consonant of the ensuing noun.}  It does not mark definiteness – which is a phrase-level category, but \textit{state} – which is a word-level category. The article marks nouns (and adjectives) as being in the \textit{emphatic} \textit{state}. The emphatic state alternates with the other two values of the state category: the unmarked \textit{absolute} \textit{state} and the \textit{construct} \textit{state}, which marks the noun as relational/possessee.\footnote{\textrm{The term} \textrm{\textit{emphatic}} \textrm{in ‘emphatic state’ is a Semiticists’ term, used mostly in descriptions of Aramaic, marking a particular value of the inflectional state of a noun and is unrelated both to the phonological term} \textrm{\textit{emphatic}} \textrm{in the sense of} \textrm{\textit{stressed}} \textrm{and to the phonetic term} \textrm{\textit{emphatic}} \textrm{in the sense of} \textrm{\textit{pharyngealized}}\textrm{. The} \textrm{\textit{emphatic} \textit{state}} \textrm{form of N will be glossed as ‘the-N’ in the examples below, and the} \textrm{\textit{construct} \textit{state}} \textrm{– as ‘N(of)’.}} A noun in the emphatic state projects its emphaticity value to containing NPs, and eventually results in its maximal NP projection being interpreted as definite, through the definite type-shift to ${\iota}$x.[[NP]](x).\footnote{\textrm{${\iota}$x.P(x) is the maximal individual satisfying P, defined both for singular and plural predicates \citep{Sharvy1980}.}} In the simplest case, an unmodified emphatic N forms an emphatic NP by itself, and is interpreted as definite. For example the noun \textit{water} in \REF{ex:doron:3}a is also a maximal NP, hence its prefixation by \textit{han-} is understood as definite: \textit{the} \textit{water}. On the other hand, the noun \textit{water} in \REF{ex:doron:3}b is not a maximal NP but part of a larger NP. Accordingly, its prefixation by \textit{han-} marks it as emphatic, not as definite. It is its emphatic NP projection \textit{well} \textit{of} \textit{water} which is interpreted as definite, not \textit{a} \textit{well} \textit{of} \textit{the} \textit{water} but \textit{the} \textit{well} \textit{of} \textit{water}:\footnote{Unless stated otherwise, all Biblical translations are from the New King James Version (NKJV). \textrm{The pairs of} allophones \textit{b-β,} \textit{g-ɣ,} \textit{d-ð,} \textit{k-x,} \textit{p-f,} \textit{t-θ,} \textrm{are transcribed according to the Hebraist transcription} \textrm{\textit{b-}}\textit{ḇ,} \textit{g-ḡ,} \textit{d-ḏ,} \textit{k-ḵ,} \textit{p-\={p}, t-ṯ.} Three vowel qualities are distinguished, in accordance with the Tiberian tradition, e.g. \textit{ā} vs. \textit{a} vs. epenthetic \textit{ă}. Glosses use the following abbreviations: \textsc{acc} – Accusative case; \textrm{\textsc{dual}} \textsc{–} Dual number; \textsc{exst} \textsc{–} Existential copula; \textsc{f} – Feminine; \textsc{ill} – Illative case; \textsc{impr} \textsc{–} \textsc{I}mperative;\textrm{ \textsc{inf} }\textrm{– Infinitive;} \textsc{juss} – Jussive;  \textsc{m} – Masculine; \textsc{mod} – Modal; \textsc{neg} – Negation; \textsc{p} – Plural;  \textsc{poss} – Possessive case; \textsc{pron} – Pronominal copula; \textsc{prstv}\textrm{ \textsc{–} \textsc{P}}\textrm{resentative;}\textrm{ \textsc{ptcp} }– Participle; \textsc{q} – Question particle; \textsc{s} – Singular; \textsc{supr} – superlative.} 

\ea%3
    \label{ex:doron:3}
    \ea
    \gll \textit{way.yōmɛr}       \textit{ʔɛlōhīm}   \textit{yəhī}               \textit{rāqīaʕ}  \textit{bə.ṯōḵ}   \textit{ham-māyim}\\
         and.said.\textsc{3ms}    God       be.\textsc{juss}.3\textsc{ms}   sky       inside   the-water\\
    \glt Then God said, Let there be a firmament in the midst of the water. (Gen. 1:6)
    \ex
    \gll \textit{hinnē}    \textit{ʔānōḵī}  \textit{niṣṣaḇ}               \textit{ʕal} \textit{ʕēn}         \textit{ham-māyim}\\
          \textsc{prstv}   I           stand.\textsc{ptcp.ms}   at   well(of)  the-water\\
    \glt  Behold, I stand by the well of water. (Gen. 24:43)
    \z
\z


In contrast, an absolute-state NP is unmarked for definiteness. It is typically interpreted as indefinite as in \REF{ex:doron:4}: 

\ea%4
    \label{ex:doron:4}
    \gll \textit{way.yēlɛḵ}           \textit{way.yimṣāʔ-ēhū}              \textit{ʔaryē}      \textit{b-ad-dɛrɛḵ}   \textit{way.yəmīṯ-ēhū}    \\
         and.went.\textsc{3ms}    and.met.\textsc{3ms-acc.3ms}   lion\textsc{.ms}   in-the-road   and.killed.\textsc{3ms-acc.3ms}\\
    \glt When he was gone, a lion met him on the road and killed him.~(1Kings 13:24)
    \z

The absolute-state subject \textit{lion} of the main clause of \REF{ex:doron:4} denotes the predicate λx.lion(x). This predicate can combine with the clause’s predicate λx.P(x) by predicate modification: λx.lion(x) \& P(x). The truth value of the sentence is calculated by applying the indefinite type-shift (existential closure): ${\exists}$x.lion(x) \& P(x).

But since an absolute-state NP is unmarked, it can on principle also be interpreted as definite. The definite interpretation is normally thwarted by the principle of \textit{Maximize} \textit{Presupposition} \citep{Heim1991}, which would favour the use of an emphatic-state NP to indicate definiteness. Yet there are special cases. An absolute-state NP may be interpreted as definite when the property it denotes holds of a unique entity by virtue of its meaning. This is the case of kind-names \citep{Doron2003}, as in \REF{ex:doron:5}, or NPs headed by \textit{kol}, as in \REF{ex:doron:6}, to which we return in section 3.

\ea%5
    \label{ex:doron:5}
    \gll \textit{wə.ḡār}                        \textit{zəʔēḇ}      \textit{ʕim}  \textit{kɛḇɛś…}       \textit{wə-ʔaryē}        \textit{k-ab-bāqār}   \textit{yōḵal}              \textit{tɛḇɛn}\\
         and.dwell.\textsc{mod.3ms}   wolf.\textsc{ms}  with lamb.\textsc{ms}… and-lion\textsc{.ms}   as-the-cattle  eat.\textsc{mod.3ms}  straw\\
    \glt The wolf shall dwell with the lamb… and the lion shall eat straw like the ox. (Isa. 11:7)
    \z


\ea%6
    \label{ex:doron:6}
    \gll \textit{way.yōsɛ\={p}              ʕōḏ    dāwiḏ  ʔɛṯ   kol        bāħūr          bə-yiśrāʔēl   šəlōšīm ʔālɛ\={p}}\\
         and.gathered.\textsc{3mp}  again David  \textsc{acc} \textsc{kol}(of) warrior.\textsc{ms} in-Israel      thirty    thousand\\
    \glt Again David gathered all~the~choice~men~of Israel, thirty thousand. (2Sam. 6:1)
    \z

           
\subsection{The BH pseudo-partitive construction}\label{sec:doron:2.2}\label{sec:doron:2.2.}%2.2
Pseudo-partitives, also called measure constructions, denote an amount (a particular degree of a measure function) of some substance \citep{Selkirk1977}. In Hebrew, the substance is denoted by an indefinite NP complement of the determiner. The indefinite substance-denoting NP may be in the absolute state (as in the (a) examples below) or in the emphatic state (as in the (b) examples below) since emphaticity does not mark the substance NP but the whole construction as definite. The head of the construction is a degree N which partitions the substance into portions (\citealt{Schwarzschild2002}; \citealt{Ruys2017}): \REF{ex:doron:7} partitions days/ commandments into groups of ten, \REF{ex:doron:8} and \REF{ex:doron:9} partition the substance into small/ large groups respectively. \REF{ex:doron:10} partitions the craftsmen into groups consisting of all the craftsmen;  since there is only one such group, the absolute version in \REF{ex:doron:10}a and the emphatic version in \REF{ex:doron:10}b both denote a unique group:\footnote{Accordingly, \textrm{\textit{kol} \textit{NP}} \textrm{is often overtly case-marked in object position by the accusative} \textrm{\textit{ʔɛṯ}} \textrm{which marks definite direct objects, even when NP is headed by a noun in the absolute state. This was already shown in \REF{ex:doron:6} above, and is shown again here in (i) and (ii):}\textrm{(i)}\textrm{  \textit{way.yōmɛr}      \textit{ʔɛlōhīm} \textit{hinnē}   \textit{nāṯattī}    \textit{lāḵɛm}    \textit{ʔɛṯ}   \textit{kol}    \textit{ʕēśɛḇ}       \textit{zōrēaʕ}             \textit{zɛraʕ}} \textrm{and.said}\textrm{\textsc{.3ms}} \textrm{God} \textrm{\textsc{prstv}} \textrm{gave.1}\textrm{\textsc{s}} \textrm{to.2}\textrm{\textsc{mp}} \textrm{\textsc{acc}} \textrm{\textsc{kol} } \textrm{herb.}\textrm{\textsc{ms}} \textrm{seed.}\textrm{\textsc{ptcp.ms}} \textrm{seed}\textrm{And God said, See, I have given you every herb}{\textrm{~}}\textrm{that}{\textrm{~}}\textrm{yields seed.} {\textrm{(Gen. 1:29)}}{\textrm{(ii)} }\textit{way.yaħărīm}           \textit{ʔɛṯ}   \textit{kol}  \textit{nɛ}\textrm{\textit{\={p}}}\textit{ɛš}     \textit{ʔăšɛr}  \textit{bah}  and-destroyed.\textsc{3mp} \textsc{acc} \textsc{kol} soul.\textsc{fs}  that    in.\textsc{3fs}\textrm{and destroyed all the people who~were~in it (Josh. 10:39)}}\textsuperscript{,}\footnote{The BH \textit{kol} \textit{NP} is indeed a pseudo-partitive rather than a partitive construction where NP denotes an individual. Though the complement may be a name, as in \textit{kol} \textrm{\textit{yiśrāʔēl}} ‘all Israel’ (1Kings 12:20), \textit{kol} \textit{miṣrāyim} ‘\textrm{all Egypt’ (Gen. 41:55), the name in this position never denotes an individual but a set of people, i.e. ‘all Israelites’, ‘all Egyptians’. To express the totality of the geographic entity, the name has to be explicitly modified so as to clarify what kind of portions are being measured:} \textrm{\textit{kōl} \textit{ʔɛrɛṣ}} \textrm{\textit{yiśrāʔēl}} ‘all the land of Israel’ (1Sam. 13:19),\textrm{ \textit{kol} \textit{ʔɛrɛṣ}} \textit{miṣrāyim} ‘all the land of Egypt’ (Ex. 9:9).}

\ea%7
    \label{ex:doron:7}
    \ea
    \gll \textit{ʕăśɛrɛṯ}     \textit{yāmīm}\\
         ten(of)    days\\
    \glt ten days (Jer. 42:7)  
    \ex
    \gll \textit{ʕăśɛrɛṯ}    \textit{had-dəḇārim}\\
            ten(of)    the-commandments\\
    \glt the ten commandments (Exod. 34:28)
    \z
\z



\ea%8
    \label{ex:doron:8}
    \ea
    \gll \textit{məʕaṭ}      \textit{mayim}\\
         little(of)  water\\
    \glt a little water (Gen. 18:4)
    \ex
    \gll  \textit{məʕaṭ}     \textit{haṣ-ṣōn}     \textit{hā-hēnnā}\\
         few(of)  the-sheep  the-those\\
    \glt those few sheep (1Sam. 17:28)
    \z
\z
        

\ea%9
    \label{ex:doron:9}
    \ea
    \gll \textit{rōḇ}          \textit{ħoḵmā}\\
         much(of) wisdom\\
    \glt much wisdom (Eccles. 1:18)
    \ex  
    \gll \textit{rōḇ}           \textit{ziḇħē-ḵɛm}\\
         many(of)  sacrifices-\textsc{poss.2mp}\\
    \glt the multitude of your sacrifices (Isa. 1:11)
    \z
\z

    

\ea%10
    \label{ex:doron:10}
    \ea
    \gll \textit{kol}          \textit{ħaḵmē}             \textit{lēḇ}\\
         \textsc{kol}(of)  skilled.\textsc{mp}(of) heart  \\
    \glt all who are gifted artisans (Ex. 28:3)
    \ex  
    \gll \textit{kol}\textsc{\textsubscript{} }         \textit{hā-ħăḵāmīm}   \\
         \textsc{kol}(of)  the-skilled.\textsc{mp}  \\
    \glt all the craftsmen (Ex. 36:4)
    \z
\z

\section{The determiner \textit{kol}}\label{sec:doron:3}%3

As just shown in \REF{ex:doron:10}, \textit{kol} functions as a degree N which heads the pseudo-partitive construction; it denotes the \textit{entirety} degree. The distribution of \textit{kol} indicates that it originally was a noun. Indeed, traditional grammars of the Bible describe \textit{kol} as an “abstract substantive denoting totality” (\citealt{Joüon1923}: \sectref{sec:doron:139}e). It occurs in the Bible not only in the construct-state form as in \REF{ex:doron:10} above, but also in the absolute and emphatic states, as in \REF{ex:doron:11} and \REF{ex:doron:12} below. In these forms, \textit{kol}’s vowel is not shortened as it often is in the construct state (cf. \textit{kol} in \REF{ex:doron:10}), but is rather a long /\textit{ō}/, as in \textit{kōl} in \REF{ex:doron:11} and \REF{ex:doron:12}:

\ea%11
    \label{ex:doron:11}
    \ea
    \gll \textit{bə-rāʕāḇ}    \textit{ū-ḇə{}-ṣāmā      ū-ḇə{}-ʕērōm            ū-ḇə{}-ħōsɛr         kōl} \\
         in-hunger   and-in-thirst   and-in-nakedness  and-in-need(of)  \textsc{kol}                     \\
    \glt in hunger, in thirst, in nakedness, and in need of everything (Deut. 28:48)
    \ex  
    \gll \textit{kī}           \textit{ħann-ani}                      \textit{ʔɛlōhīm}  \textit{wə-ḵī}              \textit{yɛš}    \textit{lī}       \textit{ḵōl}\\
         because favoured.\textsc{3ms-acc.1s}  God        and-because  \textsc{exst} to\textsc{.1s}  \textsc{kol}\\
    \glt for God has been generous to me and I have all I need (Gen. 33:11) 
    \z
\z



\ea%12 
 \label{ex:doron:12}
 \ea
 \gll \textit{hăḇēl}          \textit{hăḇālīm}   \textit{hak-kōl}    \textit{hāḇɛl}\\
      futility(of)  futilities   the-\textsc{kol}  futility \\
 \glt Futility of futilities, all{~}is{~}futility. (MEV, \emph{\textup{Eccles.} \emph{1:2)}}
 \ex  
 \gll \textit{wa-}\textsc{yhwh}   \textit{bēraḵ}             \textit{ʔɛṯ}    \textit{ʔaḇrāhām}    \textit{b-ak-kōl}\\
      and-Lord   blessed.\textsc{3ms}   \textsc{acc}  Abraham     in-the-\textsc{kol}\\
 \glt and the~\textsc{\textsc{L}ord}~had blessed Abraham in all things (Gen. 24:1)
 \z
\z

The nominal origin of  \textit{kol} is also evident in examples where it is still interpreted as the noun ‘totality’, e.g. when it heads the event-nominalization \textit{count} in \REF{ex:doron:13}:

\ea%13
    \label{ex:doron:13}
    \gll \textit{kol}         \textit{mispar}       \textit{rāšē}         \textit{hā-ʔāḇōṯ} \textit{…}      \textit{ʔalp-ayim}          \textit{wə-šēš}  \textit{mēʔ-ōṯ}\\
         \textsc{kol}(of) count(of)  chiefs(of) the-officers … thousand-\textsc{dual} and-six  hundred-\textsc{pl}\\
    \glt The total number of chief officers\textsuperscript{} …~was~two thousand six hundred.~(2Chr. 26:12)
    \z

I reiterate that the translations of the Biblical verses are not my own, but are received translations, mostly from the New King James Version (NKJV). The translations are faithful to the meaning of each verse as a whole, but cannot be used to gauge the various nuances of the meaning of \textit{kol} or other lexical items.

\subsection{Non-distributivity of \textit{kol}}\label{sec:doron:3.1}%3.1

The present subsection demonstrates that \textit{kol} \textit{NP} is not quantificational/distributive. It denotes the entirety of a (group) individual rather than quantifying over its members/ parts. 

The first piece of evidence for the non-quantificational nature of \textit{kol} \textit{NP} is the possibility of predicating cardinality of it, unlike the English \textit{all} \textit{NP}, of which cardinality cannot be predicated. \textit{All} \textit{NP} contrasts in this respect with definite \textit{NP}s: \textit{The} \textit{apostles} \textit{were} \textit{twelve/} *\textit{All} \textit{the} \textit{apostles} \textit{were} \textit{twelve} (\citealt{Dowty1987}; \citealt{Winter2002}). In BH we find cardinals predicated of \textit{kol} \textit{NP}:\footnote{For the sake of brevity I will henceforth mostly use the gloss \textsc{kol} rather than \textsc{kol}(of).}

\ea%14
    \label{ex:doron:14}
    \gll \textit{kol}   \textit{han-nɛ\={p}ɛš   lə-ḇēṯ            yaʕăqōḇ  hab-bāʔā            miṣraym-ā    šiḇʕīm}\\
         \textsc{kol} the-soul.\textsc{fs} of-house(of) Jacob       the-go.\textsc{ptcp.fs}   Egypt-\textsc{ill}      seventy\\
    \glt All the persons of the house of Jacob who went to Egypt were seventy. (Gen. 46:27)
\z

Second, as shown in \REF{ex:doron:15}, \textit{kol} \textit{NP} does no distribute over another argument in the clause. For example, \REF{ex:doron:15}a is unlike English and other languages, where the universal subject scopes in two different ways relative to the object, yielding ambiguity in \textit{All} \textit{the} \textit{artisans} \textit{made} \textit{ten} \textit{curtains}. 

\ea%15
    \label{ex:doron:15}
    \ea
    \gll \textit{way.yaʕăśū}      \textit{kol}   \textit{ħăḵam}            \textit{lēḇ}     \textit{bə-ʕōśē}                          \textit{ham-məlāḵā}  …  \textit{ʕɛśɛr}  \textit{yərīʕōṯ}\\
         and.made.\textsc{3mp} \textsc{kol} skilled.\textsc{ms}(of) heart among-do.\textsc{ptcp.mp}(of)  the-work      \textsc{…}   ten    curtains  \\
    \glt Then all the gifted artisans among them who worked … made ten curtains.   (Ex. 36:8)  (non-distributive only)
    \ex  
    \gll \textit{yōm} \textit{la-}\textsc{yhwh}\textit{\MakeUppercase{} } \textit{ṣəḇāʔōṯ}    \textit{ʕal}  \textit{kol}    \textit{gēʔɛ}    \textit{wā-rām}       \\
         day  to-Lord   Sabaoth   for  \textsc{kol}  proud and-lofty  \\
    \glt The~Lord~Almighty has a day~in store for all the proud~and lofty. (NIV, Isa. 2:12) (non-distributive only)
    \z
\z

I am not aware of examples like \REF{ex:doron:15} where \textit{kol} \textit{NP} distributes over another argument.

Third, even when its complement NP is singular, \textit{kol} \textit{NP} denotes the entirety of a group and functions as subject of collective predicates, unlike other languages where NP\textsubscript{sing} only cooccurs with distributive \textit{every}: 

\ea%16
    \label{ex:doron:16}
    \ea
    \gll \textit{way.yiṯqabṣū}         \textit{ʔēlāw}    \textit{kol}   \textit{ʔīš}           \textit{māṣōq}\\
         and.gathered.3\textsc{mp}  to.\textsc{3ms}  \textsc{kol} man(of) distress \\
    \glt And everyone~who was~in distress … gathered to him. (1Sam. 22:2)
    \ex  
    \gll \textit{way.yiqqāhălū}          \textit{ʔɛl} \textit{ham-mɛlɛḵ}  \textit{šəlōmō}    \textit{kol}   \textit{ʔīš}          \textit{yiśrāʔēl}\\
         and.assembled.3\textsc{mp}  to  the-king      Salomon \textsc{kol} man(of) Israel\\
    \glt Therefore all the men of Israel assembled with King Solomon. (1Kings 8:2)
    \ex  
    \gll \textit{wə-ʔēlay}   \textit{yēʔās\={p}ū                 kol   ħārēḏ                       bə-ḏiḇrē       ʔɛlōhē      yiśrāʔēl}\\
         and-to.\textsc{1s}  congregated.3\textsc{mp}  \textsc{kol} tremble.\textsc{ptcp.3ms}  at-words(of)  God(of)  Israel\\
    \glt Then everyone who~trembled at the words of the God of Israel assembled to me. (Ezra 9:4)
    \z
\z


In other examples with NP\textsubscript{sing}, \textit{kol} \textit{NP} denotes the entirety of an individual: \textit{the} \textit{whole} \textit{NP/all} \textit{the} \textit{NP.}\footnote{These examples argue against Naudé’s \citeyearpar{Naudé2011interpretation} account of \textit{kol}, which consists in translating \textit{kol} as \textit{every} with NP\textsubscript{+count$-$def} and as \textit{all} with NP\textsubscript{±count+def}. Naudé’s account is mistaken for \REF{ex:doron:17}. Moreover, it is incompatible with the lack of distributive interpretation of NP\textsubscript{+count$-$def} in \REF{ex:doron:15} and \REF{ex:doron:16}: we would expect distributivity with \textit{every}. Naudé’s account ignores \textit{kol} applied to NP\textsubscript{$-$count$-$def} as in \REF{ex:doron:18}, which Naudé claims does not exist (2011a: 418), and also ignores all examples where \textit{kol} can be translated as neither \textit{all} nor \textit{every}, cf. section 3.3 below.}

\ea%17
    \label{ex:doron:17}
    \ea
    \gll \textit{bə-ḵol}       \textit{lēḇ}      \textit{ū-ḇə{}-ḵol           nɛ\={p}ɛš} \\
         with-\textsc{kol} heart  and-with-\textsc{kol}  soul        \\
    \glt with all~his~heart and all~his~soul (2Kings 23:3) 
    \ex  
    \gll \textit{kol}    \textit{rōš}     \textit{lā-ħŏlī}        \textit{wə-ḵol}     \textit{lēḇāḇ}        \textit{dawwāy}\\
         \textsc{kol} head  in-sickness  and-\textsc{kol} heart.\textsc{ms}  faint.\textsc{ms}\\
    \glt The whole head is sick and the whole heart faints. (Isa 1:5)
    \z
\z

NP may also be an absolute-state mass term:\footnote{The nouns \textit{gold} and \textit{silver} are mass nouns in BH, just as they are in Modern Hebrew and in English, since they do not pluralize, and, though singular, denote space-filling substance:
\ea 
    \gll \textit{ʔim} \textit{yittɛn}                \textit{lī}      \textit{ḇālāq} \textit{məlō}             \textit{ḇēṯ-ō}                    \textit{kɛsɛ}\textrm{\textit{\={p}}} \textit{wə-zāhāḇ}\\
            if    give.\textsc{mod.3ms} to.\textsc{1s} Balak fullness(of)  house-\textrm{\textsc{poss.3ms}}  silver and-gold \\
    \glt If Balak gave me his house full of silver and gold … (MEV, Num. 22:18)
\z}

\ea%18
    \label{ex:doron:18}
    \gll \textit{wə-ḵōl}     \textit{kɛsɛ\={p}        wə-zāhāḇ       ū{}-ḵlē                 nəħōšɛṯ  u-ḇarzɛl   qōḏɛš        hū            la-}\textsc{yhwh}\\
         and-\textsc{kol} silver.\textsc{ms} and-gold.\textsc{ms}  and-vessels(of)  bronze   and-iron  sacred.\textsc{ms}  \textsc{pron.3ms}  to-Lord\\
    \glt But all the silver and gold, and vessels of bronze and iron,{~}are{~}consecrated to the{~}\textsc{\textsc{L}ord. (Josh. 6:19)} 
    \z

Fourth, verbal agreement provides additional evidence for the lack of distributivity of \textit{kol} \textit{NP}. If \textit{kol} were distributive, we would expect \textit{kol} \textit{NP}\textsubscript{sing} to strictly agree in the singular like \textit{every} and unlike \textit{all} (which agrees either in the plural or the singular). Yet irrespective of the number marking of NP, verbal agreement is often plural, even for singular NP. Example \REF{ex:doron:19} shows plural agreement when NP is plural, as is to be expected. \REF{ex:doron:20} shows the same plural agreement when NP is singular. The relevant NPs are in the absolute state in the (a) examples, and in the emphatic state in the (b) examples: 

\ea%19
    \label{ex:doron:19}
    \ea
    \gll \textit{wə-ḵol}     \textit{birkayim}   \textit{tēlaḵnā}                  \textit{mmayim}\\
         and-\textsc{kol}  knees.\textsc{fp}   become\textsc{.mod.3fp}   water\\
    \glt and all knees will be weak~as~water (Ezek. 21:12)
    \ex  
    \gll \textit{kī}   \textit{mēṯū}        \textit{kol}    \textit{hā-ʔănāšīm}   \textit{ha-məḇaqšīm}         \textit{ʔɛṯ}    \textit{na\={p}š-ɛḵā}\\
         for died.\textsc{3pl}   \textsc{kol}  the-men        the-seek.\textsc{ptcp.mp}  \textsc{acc}  soul-\textsc{poss.2ms}\\
    \glt for all the men who sought your life are dead (Ex. 4:19)
    \z
\z


\ea%20
    \label{ex:doron:20}
    \ea
    \gll \textit{way.yēṣʔū}      \textit{kol}   \textit{ʔīš}            \textit{mēʕāl-āw}\\
         and.left\textsc{.3pl}   \textsc{kol}  man.\textsc{ms}  from-\textsc{3ms}\\
    \glt So everyone left. (NET, 2Sam. 13:9)
    \ex  
    \gll \textit{kol}    \textit{hā-ʔɛzrāħ}         \textit{bə-yiśrāʔēl}  \textit{yēšḇū}                \textit{b-as-sukkōṯ}\\
         \textsc{kol}  the-native\textsc{.ms}  in-Israel       sit.\textsc{mod.3mpl}  in-the-booths\\
    \glt All who are native Israelites shall dwell in booths. (Lev. 23:42)
    \z
\z

If \textit{kol} were distributive, it would be unexpected for \textit{kol} \textit{NP}\textsubscript{sing} to cooccur with V\textsubscript{pl} in \REF{ex:doron:20a}, unless we think that Biblical subject-verb agreement is haphazard: there indeed are many  other examples where \textit{kol} \textit{NP} cooccurs with V\textsubscript{sing}. But in fact these are all \textit{kol} \textit{NP}\textsubscript{sing}. There are no examples where \textit{kol} \textit{NP}\textsubscript{pl} cooccurs with V\textsubscript{sing}.\footnote{I exclude irrelevant examples such as left-conjunct agreement \citep{Doron2005}, passive verbs, and verbs where the subject of \textrm{V}\textrm{\textsubscript{sing}} is actually not \textit{kol} \textit{NP}\textsubscript{pl} but a null expletive as in (i) below: \textrm{(i)}\textrm{  \textit{wa.yəhī}   \textit{kol}   \textit{han-nō}}\textit{\={p}līm         b-ay-yōm  ha-hū    mē-ʔiš  wə-ʕaḏ            ʔiššā     šənēm.ʕā}\textrm{\textit{śār} \textit{ʔālɛ}}\textit{\={p}}  was.\textsc{3ms} \textsc{kol} the-fall.\textsc{ptcp.mp} in-the-day the-that of-man and-including woman twelve         thousand  \textrm{So it was~that~all who fell that day, both men and women,~were~twelve thousand} (Josh. 8:25)} This agreement pattern is actually systematic under the assumption that \textit{kol} \textit{NP} is collective and may hence be marked as plural [kol NP]\textsubscript{pl} independently of the number feature of NP. Accordingly, V\textsubscript{sing} only cooccurs with \textit{kol} \textit{NP}\textsubscript{sing}, whereas V\textsubscript{pl} cooccurs both with [\textit{kol} NP\textsubscript{sing}]\textsubscript{pl} and [\textit{kol} NP\textsubscript{pl}]\textsubscript{pl}.\footnote{Under Naudé’s 2011a account, the agreement pattern remains mysterious.} 

Lastly, it is important to distinguish distributivity from what has been called \textit{lexical} \textit{distributivity} \citep{Winter2000}, which is due to the lexical nature of the predicate. E.g. \textit{weeping} in \REF{ex:doron:21} below can only be predicated of a group by attributing it to the individual members of the group.\footnote{Lexical distributivity can be averted by the use of collective adverbs such as \textit{together}, e.g.\textrm{(i)}\textrm{  \textit{yaħaḏ}     \textit{ʕālay} \textit{yiṯlaħă}}\textit{š}\textrm{\textit{ū}                 \textit{kol}   \textit{ś}}\textit{ōnəʔ-āy}  together at.\textsc{1s}  whisper.\textsc{mod.3mp}  \textsc{kol}  hate.\textsc{ptcp.mp}{}-\textsc{poss.1s}  \textrm{All who hate me whisper together against me} (Ps. 41:8)} Lexical distributivity does not induce scopal ambiguity (\citealt{Vries2017}) and is not mediated by quantifiers. 

\ea%21
    \label{ex:doron:21}
    \gll \textit{wə-ḡam}   \textit{ham-mɛlɛḵ}  \textit{wə-ḵol}     \textit{ʕăḇāḏ-āw}                  \textit{baḵū}          \textit{bəḵī}         \textit{gāḏōl} \textit{məʔōḏ}\\
         and-also  the-king      and-\textsc{kol} servants-\textsc{poss.3ms}  wept.\textsc{3mp}   weeping  big      very\\
    \glt Also the king and all his servants wept very bitterly. (2Sam. 13:36)
\z



I conclude that \textit{kol} is not quantificational. Rather, \textit{kol} applies to a NP which denotes substance, mass or count, singular or plural, and yields a portion of the NP substance that consists of the entirety of those individuals whose parts satisfy NP. Hence \textit{kol} maps a predicate P to the set of all individuals, atoms or sums, satisfying *P.\footnote{\textrm{\textsuperscript{*}}\textrm{P denotes the minimal divisive predicate \citep{Krifka1989} which includes P: if P is itself divisive, i.e. plural or mass, then} \textrm{\textsuperscript{*}}\textrm{P=P; otherwise} \textrm{\textsuperscript{*}}\textrm{P is the pluralized version of P.}}\textsuperscript{,}\footnote{\textrm{I assume that the absolute/emphatic} \textrm{\textit{kōl} }\textrm{in \REF{ex:doron:11}/(12) above combines with a null P which spans the entire relevant domain.}}  

\ea%22
    \label{ex:doron:22}\relax
    [[kol]] =  ${\lambda}$P. ${\lambda}$x. *P(x) 
    \z
          
In argument position, the predicate \textit{kol} \textit{NP} is given a definite interpretation as the maximal individual ιx.\textit{kol}[[\textit{NP}]](x) satisfying it.\footnote{\textrm{In a downward entailing environment, the definite interpretation is disfavoured, as it is weaker than the indefinite (existential closure) interpretation. We return to this below in section 3.3.}} 

\subsection{Maximality of \textit{kol}}%3.2

We have seen that \textit{kol} does not contribute distributivity. So what does it contribute? Why say ‘all the men’ rather than simply ‘the men’, if it is not for the purpose of allowing distributivity?

The answer seems to be that \textit{kol} \textit{NP} denotes the sum of \textit{all} parts of NP, including absolutely all of them \citet{Brisson1997,Brisson2003}. This is illustrated by the following example, which demonstrates that tearing away the kingdom is compatible with not tearing away all the kingdom:


\ea%23
    \label{ex:doron:23}
    \gll \textit{qārōaʕ}   \textit{ʔɛqraʕ}          \textit{ʔɛṯ}   \textit{ham-mamlāḵā}   \textit{mē-ʕālɛḵā} \textit{…} \textit{raq} \textit{ʔɛṯ}   \textit{kol}   \textit{ham-mamlāḵā} \textit{lō}      \textit{ʔɛqraʕ}          –   \textit{šēḇɛṭ}      \textit{ʔɛħāḏ}    \textit{ʔɛtēn}               \textit{li-ḇn-ɛḵā}\\
         tear.\textsc{inf}  tear.\textsc{mod.1s} \textsc{acc} the-kingdom     from-over.2\textsc{ms} \textsc{…} but \textsc{acc} \textsc{kol} the-kingdom    \textsc{neg}  tear.\textsc{mod.1s}  –   tribe\textsc{.ms} one.\textsc{ms}  give.\textsc{mod.1s}   to-son-\textsc{poss.2ms}\\
    \glt I will surely tear the kingdom away from you … However, I will not tear away the whole kingdom; I will give~one tribe to your son. (1Kings 11:13)
    \z

\textit{kol} disallows the slack allowed by \textit{the\textsubscript{pl}} (\citealt{Krifka2006}; \citealt{Lasersohn1999}; \citealt{Schwarz2013}). Lasersohn characterizes \textit{slack} as pragmatic looseness which involves approximation to the truth that does not affect truth conditions. When speaking loosely, the speaker takes it to be unlikely that the (possible) difference between the actual world and his assertion is relevant for present purposes. To adapt an example of \citet{Lauer2012}, \textit{I} \textit{live} \textit{in} \textit{Tel-Aviv} is true in a context where the speaker lives in Jaffa, which abuts Tel-Aviv, but is not part of it. Various expressions, such as \textit{proper}, are seen as \textit{slack} \textit{regulators} in this respect. \textit{I} \textit{live} \textit{in} \textit{Tel-Aviv} \textit{proper} cannot be used with slack: it is never appropriate if the speaker lives in Jaffa. 

The plural definite \textit{the\textsubscript{pl}} displays pragmatic slack: it makes a sentence such as \textit{The} \textit{boys} \textit{smiled} true even if there are some exceptions, assuming those exceptions do not matter for present purposes. The role of \textit{all}, on this view, is that of a slack regulator. It disallows the flexibility permitted by the plural definite \textit{the\textsubscript{pl}}. This view integrates what \citet{Dowty1987} called the ‘maximizing effect’ of \textit{all.} \textit{All} \textit{the} \textit{boys} \textit{smiled} is interpreted maximally. 

\citet{Winter2001} attributes the maximality of \textit{all} to its being quantificational. Winter shows that \REF{ex:doron:24}a but not \REF{ex:doron:24}b is entailed by \REF{ex:doron:25}: 

\ea%24
    \label{ex:doron:24}
    \ea The members of the organizing committee met.
    \ex All the members of the organizing committee met.
    \z
\z


\ea%25
    \label{ex:doron:25}
    The organizing committee met.
\z 

(24a) has a reading equivalent to \REF{ex:doron:25}. Under this reading the denotation of the definite \textit{the} \textit{members} \textit{of} \textit{the} \textit{organizing} \textit{committee} is mapped to a group individual representing the committee itself. Such a process is impossible in \REF{ex:doron:24}b, where the only way to achieve collectivity is to use quantification which requires every committee member in \REF{ex:doron:24}b to participate in the meeting. 

In BH, the maximality of \textit{kol} is not due to quantification over individuals, since \textit{kol} is not quantificational. Rather, the maximality of \textit{kol} is a consequence of measurement as expressed by the pseudo-partitive construction. Measuring an individual requires taking into account its full extent, preventing non-maximality.\footnote{I therefore beg to differ from one passage in the medieval Rabbinic exegetical literature \citep[245]{Assaf1929}, where the maximality of \textit{kol} is disputed, and it is argued that \textit{kol} only gives rise to an existential commitment. The problem is the apparent contradiction between two verses in Ch. 9 of the book of Exodus, the first describing the extinction of all Egyptian livestock by the plague, and the second – Moses’ subsequent words to Pharaoh, which presuppose that not all the livestock had perished.  (i)  \textit{way.yāmāṯ}  \textit{kōl}   \textit{miqnē}            \textit{miṣrāyim}      (ii)  \textit{wə-ʕattā}  \textit{šəlaħ}  \textit{hāʕēz} \textit{ʔɛṯ}    \textit{miqnə-ḵā}  and.died      \textsc{kol} livestock(of) Egypt    and-now  send  gather \textsc{acc} livestock-\textsc{poss.2ms}\textrm{And all the livestock of Egypt died.   (}Ex. 9:6) \textrm{Send now~and~gather your livestock.} (Ex. 9:19)} 

\subsection{Homogeneity of \textit{kol}}%3.3

In dictionaries and traditional grammars of Biblical Hebrew, \textit{kol} is translated as \textit{all} (sometimes as \textit{every}, mistakenly in my view). But in addition, these sources mention that in combination with negation, \textit{kol} is interpreted as \textit{none} \textit{at} \textit{all} (rather than \textit{not} \textit{all}). Hence, it seems to exhibit what has been called \textit{polarity} \citep{Löbner2000} or \textit{homogeneity} (recently \citet{Križ2016}), which is surprising, since this phenomenon is said to be incompatible with the maximality of \textit{all} (Križ argues that maximality is the by-product of lack of homogeneity).

\subsubsection{The puzzle}%3.3.1.

\textit{Homogeneity} is a property of plural predication which requires that a plurality not be mixed with respect to the property predicated of it (or its negation). For \REF{ex:doron:26}a below to be true, the subject must have reacted to all the external stimuli. For \REF{ex:doron:26}b to be true, the subject must have reacted to none of the external stimuli. In mixed scenarios, where the subject reacted to some but not all of the stimuli, neither \REF{ex:doron:26}a nor \REF{ex:doron:26}b is true. These scenarios are what Križ calls an “extension gap”, where \REF{ex:doron:26}a and \REF{ex:doron:26}b are neither true nor false: 

\ea%26
    \label{ex:doron:26}
\ea The subject reacted to the external stimuli.
\ex The subject did not react to the external stimuli.
\z
\z

Homogeneity is also found with measure phrases, as in the following English examples from the web. 

\ea%27
    \label{ex:doron:27}
    \ea I didn't add the glass of chardonnay.   (i.e. I didn’t add any of it)
    \ex It said it had friction modifier already in it so I didn't add the bottle of motorcraft modifier.
\z
\z

Homogeneity disappears in English in the presence of \textit{all}. In \REF{ex:doron:28}, if the subject reacted to some but not all of the stimuli, \REF{ex:doron:28a} is simply false and \REF{ex:doron:28b} is true. 

\ea%28
    \label{ex:doron:28}
    \ea The subject reacted to all the external stimuli.
    \ex The subject did not react to all the external stimuli.
\z
\z

It is therefore surprising that in BH, sentences with \textit{kol} do exhibit homogeneity. In BH, negating a sentence with \textit{kol} does not yield ‘not all’ but ‘none at all’, i.e. ‘not any’. 

\ea%29
    \label{ex:doron:29}
    \ea
    \gll \textit{wə-ḵōl}     \textit{śīaħ}         \textit{haś-śāḏɛ} \textit{ṭɛrɛm}     \textit{yihǝyɛ}           \textit{ḇ-ā-ʔārɛṣ}    \textit{wə{}-ḵōl    ʕēśɛḇ       haś-śāḏɛ ṭɛrɛm} \textit{yiṣmāħ}\\
         and-\textsc{kol} plant(of) the-field  still.not be.\textsc{mod.3ms} in-the-earth and-\textsc{kol} herb(of) the-field still.not grow.\textsc{mod.3ms}\\
    \glt before any~plant of the field was in the earth and before any herb of the field had grown (Gen. 2:5)\\ ${\neq}$  before all plants of the field were in the earth and before all herbs of the field had grown 
    \ex
    \gll \textit{kol}   \textit{ʔăšɛr} \textit{lō}     \textit{yāḏəʕū}      \textit{ʔēṯ}   \textit{kol}   \textit{milħămoṯ}   \textit{kənāʕan}\\
         \textsc{kol} that   \textsc{neg} knew.\textsc{3mp}  \textsc{acc} \textsc{kol} wars(of)    Canaan\\
    \glt all who had not~experienced any of the wars in Canaan~(Judg. 3:1)\\${\neq}$  all who had not experienced all of the wars in Canaan
    \ex  
    \gll \textit{lō}     \textit{təḇaʕărū}               \textit{ʔēš}  \textit{bə-ḵōl}   \textit{mōšḇōṯ-ēḵɛm}              \textit{bə-yōm}      \textit{haš-šabbāṯ}\\
         \textsc{neg} kindle.\textsc{mod.2mp}  fire  in-\textsc{kol} dwellings-\textsc{poss.2mp}  on-day(of) the-Sabbath\\
    \glt You shall kindle no fire throughout your dwellings on the Sabbath day.  (Ex. 35:3)\\${\neq}$  You shall not kindle fire throughout all your dwellings. 
    \z
\z

There is a well-known dialogue in the story of the Garden of Eden, where the snake queries Eve as in \REF{ex:doron:30}. Her answer starts by denying that she and Adam had been forbidden from eating any of the fruit of the garden, thus indicating that she interprets the snake’s use of \textit{kol} as involving homogeneity:\footnote{Other translators, for example the NKJV, consider \textit{kol} here to be focused, and hence translate \textrm{\textit{Has} \textit{God} \textit{indeed} \textit{said,} \textit{‘You} \textit{shall} \textit{not} \textit{eat} \textit{of} }\textrm{\textbf{\textit{every}}}\textrm{ \textit{tree} \textit{of} \textit{the} \textit{garden’?}}}

\ea%30
    \label{ex:doron:30}
    \gll \textit{ʔa\={p}        kī         ʔāmar     ʔɛlōhīm  lō     tōḵlū               mik-kōl     ʕēṣ        hag-gān}\\
         indeed  indeed  said.\textsc{3ms}  God       \textsc{neg} eat\textsc{.mod.2mp} from-\textsc{kol} tree(of) the-garden  \\
    \glt Has God indeed said ‘You shall not eat of any tree of the garden’?  (MEV, Gen. 3:1)
    \z

According to Križ, maximality derives from lack of homogeneity, whereas here we see that the maximality of \textit{kol} is compatible with its homogeneity. A parallel puzzle in English is mentioned by Križ (2016: 515), where maximality does not depend on lack of homogeneity. His example is of definite plurals with numerals. These plurals are homogeneous in English, but are only interpreted maximally, e.g. \textit{The} \textit{six} \textit{professors} \textit{smiled} requires all of them to have smiled. Interestingly, the syntax of such plurals in BH parallels that of \textit{kol} \textit{NP}. Both have the structure \textit{N(of)} \textit{NP} where N functions as a degree determiner and NP is indefinite irrespective of its emphatic marking (as emphaticity marks the whole construction as definite rather than the complement NP):  

\ea%31
    \label{ex:doron:31}
    \ea
    \gll \textit{šēšɛṯ}      \textit{yəmē}      \textit{ham-maʕăśɛ}       \\
         six(of)  days(of)  the-work              \\
    \glt the six working days  (Ez. 46:1)                  
    \ex  
    \gll \textit{kōl}          \textit{yəmē}        \textit{ħayy-āw} \\
         \textsc{kol(}of)  days(of)   life.\textsc{pl-poss.3ms}\\
    \glt all the days of his life  (1Sam. 7:15)
    \z
\z

The structure in \REF{ex:doron:31} is that of the pseudo-partitive discussed above in section 2.2. In English too, definite plurals with numerals are not interpreted like other definite plurals. A definite plural does not presuppose anything beyond existence; in particular it does not presuppose uniqueness. A definite plural with the numeral \textit{six} presupposes contextual uniqueness of a group individual with the measure \textit{six}. The phrase \textit{the} \textit{six} \textit{working} \textit{days} is interpreted as the unique individual in the context of a week which has measure \textit{six} out of the substance \textit{working} \textit{days}. Accordingly, the English \textit{the} \textit{six} \textit{working} \textit{days} is a pseudo-partitive, i.e. a measure phrase, just like the BH \REF{ex:doron:31a}. The denotation of the relevant degree is given in \REF{ex:doron:32}, where \#x denotes the number of atoms that the individual x consists of. 

\ea%32
    \label{ex:doron:32}\relax
     [[\textit{šēšɛṯ}]] = ${\lambda}$P. ${\lambda}$x. \textsuperscript{*}P(x) \& \#x = 6
\z

As in the case of \textit{kol}, measurement is what guarantees maximality despite homogeneity, both in Hebrew and in English. Unlike the case of \textit{kol}, ${\iota}$x.[[\textit{šēšɛṯ} \textit{NP}]](x) is not necessarily defined (unless the cardinality of NP is 6).\footnote{When the complement NP of the numeral is in the absolute state, the measure phrase is interpreted as indefinite. \citealt{MoshaviRothstein2018} attributes the “durational measuring phrase” interpretation of such phrases, e.g. (i) below, to indefiniteness. Yet definite measure phrases are also attested, such as \REF{ex:doron:31a}. (i)  \textit{š}\textrm{\textit{ē}}\textit{š}\textrm{\textit{ɛṯ}     \textit{yāmīn} \textit{taʕăḇōḏ}              \textit{u-ḇ{}-ay-yōm      ha}}\textit{š}\textrm{\textit{{}-}}\textit{š}\textrm{\textit{əḇīʕī}     \textit{ti}}\textit{š}\textrm{\textit{bōṯ}}\textrm{six(of)  days   work.}\textrm{\textsc{mod.2ms} } \textrm{and-in-the-day  the-seventh  rest.}\textrm{\textsc{mod.2ms}}\textrm{Do your work in six days and rest on the seventh day. (CEV; Exod. 34:21)}}

\subsubsection{An account of homogeneity}%3.3.2.

As was shown in section 3.1, \textit{kol} \textit{NP} is a predicate, hence there are two ways of combining it with the sentence predicate VP which is of the same type. One way, represented in \REF{ex:doron:33a}, is to type-shift \textit{kol} \textit{NP} to type \textit{e} by applying the definite type-shift. The other way, represented in \REF{ex:doron:33b}, involves combining \textit{kol} \textit{NP} with VP via predicate modification, followed by the application of the indefinite type-shift (existential closure).  

\ea%33
    \label{ex:doron:33}
         a.            S\textsubscript{t}                  b            S\textsubscript{t}

              2                2

        NP\textsubscript{e}          VP\textsubscript{et}           ${\exists}$            S\textsubscript{et}

             2                2

           ι              NP\textsubscript{et}          NP\textsubscript{et}     VP\textsubscript{et}

              2                 2

           \textit{kol}       NP\textsubscript{et}             \textit{kol}       NP\textsubscript{et}
\z\todo[inline]{Example needs work}

In general, the stronger interpretation is the definite interpretation in \REF{ex:doron:33a}. But in a downward entailing environment, negation for example, \REF{ex:doron:33b} is stronger. If no element of NP satisfies VP, then neither does the maximal element. But if the maximal element of NP does not satisfy VP, this does not entail that no element of NP does. 

The two type shifts available for the derivation of a sentence with \textit{kol} \textit{NP}, coupled with the \textit{Stronger} \textit{Meaning} \textit{Hypothesis}: Pick the stronger meaning \citep{DalrympleEtAl1994}, predict the following:\footnote{\textrm{The analysis follows \citealt{Krifka1996} (also \citealt{Malamud2012}; \citealt{Spector2018}), where plural referential expressions are interpreted as universal/ existential on the basis of the Stronger Meaning Hypothesis.} } 

\ea%34
    \label{ex:doron:34}
\ea Definite type-shift  \REF{ex:doron:33a} in non-downward-entailing environments 
\ex Indefinite type-shift  \REF{ex:doron:33b} in downward-entailing environments
\z
\z


Indeed the indefinite type-shift is attested in downward entailing environments, including, besides negation, other downward entailing environments as well. Indefinite type-shifted \textit{kol} \textit{NP} can thus be interpreted as a Negative Polarity Item (NPI).\footnote{Raising \textit{kol} \textit{NP} out of the downward-entailing environment and interpreting it by the definite type-shift \REF{ex:doron:33a} does not yield the right truth conditions in the question example in \REF{ex:doron:40}, and is impossible in \REF{ex:doron:43} because of the island nature of the conditional protasis. Hence there is no way of forgoing the indefinite type-shift \REF{ex:doron:33b}.} 

 \ea
 negation\label{ex:doron:37}\\
 \ea
 \gll \textit{wə-ʔēn}     \textit{kol}   \textit{ħāḏāš}  \textit{taħaṯ}  \textit{haš-šāmɛš}\\
      and-\textsc{neg}  \textsc{kol} new     under the-sun\\
 \glt And there isn’t anything new under the sun. (Eccles. 1:9)\\${\neq}$  Not all new things are under the sun. 
 \ex
 \gll \textit{lō}      \textit{yirʔɛ}                \textit{kol}   \textit{ħaḵmē}              \textit{lēḇ}  \\
      \textsc{neg}  see.\textsc{mod.3ms}   \textsc{kol} skilled\textsc{.mp}(of)  heart\\
 \glt He shows no partiality to any~who are~wise of heart. (Job 37:24)\\${\neq}$  He shows no partiality to all~who are~wise of heart. 
 \ex
 \gll \textit{wə-lō}       \textit{māṣəʔū}      \textit{ḵol}   \textit{ʔanšē}     \textit{ħayil}    \textit{yəḏē-hɛm}\\
      and-\textsc{neg} found.\textsc{3mp}  \textsc{kol} men(of) might  hands{}-\textsc{poss.3mp}\\
 \glt And none of the mighty men have found the use of their hands. (Ps. 76:5)\\${\neq}$  And not all the mighty men have found the use of their hands.
 \z
\z



\ea%38
    generic NP\label{ex:doron:38}\\
    \ea
    \gll \textit{nɛ\={p}ɛš      ʔăšɛr tiggaʕ                bə-ḵol  dāḇār  ṭāmē …     wə-hū    ṭāmē      wə-ʔāšēm}\\
         soul.\textsc{fs}   that   touch.\textsc{mod.3fs}  at-\textsc{kol} thing  unclean… and-he   unclean and-guilty\\
    \glt a person who touches any unclean thing… he shall be unclean and guilty (NET, Lev. 5:2)\\${\neq}$   a person who touches all unclean things… he shall be unclean and guilty 
    \ex  
    \gll \textit{ʔārūr}          \textit{šōḵēḇ}          \textit{ʕim}   \textit{kol}   \textit{bəhēmā}\\
         cursed.\textsc{ms}  lie.\textsc{ptcp.ms}  with \textsc{kol} animal\\
    \glt Cursed is the one who lies with any kind of animal. (Deut 27: 21)\\${\neq}$  Cursed is the one who lies with all the kinds of animals.
    \z
\z

\ea%39
    FC NP\label{ex:doron:39}\\
    \gll \textit{kol}   \textit{nɛ\={p}ɛš   ʔăšɛr tōḵal             kol  dām     wə.niḵrəṯā                     han-nɛ\={p}ɛš   ha-hī         mē-ʕamm-ɛhā}\\
         \textsc{kol} soul.\textsc{fs} that   eat.\textsc{mod.3fs} \textsc{kol}  blood and.will.be.cut.off.\textsc{3fs}  the-soul.\textsc{fs} the-that.\textsc{fs} from-people{}-\textsc{poss.3fs}\\
    \glt Whoever eats any blood – that person will be cut off from his people. (Lev. 7:27)\\${\neq}$  Whoever eats all the blood – that person will be cut off from his people. 
    \z

\ea%40
    question\label{ex:doron:40}\\
    \gll \textit{hinnē}  \textit{ʔănī} \textsc{yhwh}  \textit{ʔɛlōhē}   \textit{kol}   \textit{bāśār} \textit{–}  \textit{hā-mimɛnni}  \textit{yippālē}                                  \textit{kol}   \textit{dāḇār}\\
         \textsc{prstv} I      Lord    God(of) \textsc{kol} flesh  \textit{–}  Q-from.1\textsc{s}    be.beyond.ability.\textsc{mod.3ms}  \textsc{kol} thing \\
    \glt Behold, I am the \textsc{\textsc{L}ord}... Is there anything too hard for Me? (Jer. 32:27)\\${\neq}$   Are all things too hard for Me? 
    \z     

\ea%41
    complement of adversative verbs\label{ex:doron:41}\\
    \gll \textit{wə-šōmēr}                  \textit{yāḏ-ō}                 \textit{mē-ʕăśōṯ}      \textit{kol}   \textit{rāʕ}\\
         and-keep.\textsc{ptcp.3ms}  hand{}-\textsc{poss.3ms} from-do.\textsc{inf}  \textsc{kol} evil \\
    \glt … and keeps his hand from doing any evil  (Isa. 56:2)\\${\neq}$… and keeps his hand from doing all evil things
    \z

\ea%42
    \textit{before}{}-PPs\label{ex:doron:42}\\
    \gll \textit{ʕōḏ-ɛnnū}  \textit{ḇə-ʔibb-ō}                \textit{lō}     \textit{yiqqāṭē}\textit{\={p}         \textit{wə-li}}\textit{\={p}\textit{nē}       \textit{ḵol}   \textit{ħāṣīr}  \textit{yīḇāš}}\\
         still-\textsc{3ms} in-green-\textsc{poss.3ms}  \textsc{neg}   cut.\textsc{mod.3ms}  and-before  \textsc{kol} plant  wither.\textsc{mod.3ms}\\
    \glt While it~is~yet green~and~not cut down,it withers before any~other~plant. (Job 8:12)\\${\neq}$  While it~is~yet green~and~not cut down,it withers before all~other~plants.
    \z

\ea%43
    conditional protasis\label{ex:doron:43}\\
    \ea
    \gll \textit{ʔim}    \textit{yiggaʕ}                 \textit{ṭəmē}             \textit{nɛpɛš}          \textit{bə-kol}   \textit{ʔēllɛ}  \textit{ha-yiṭmā}\\
         if       touch.\textsc{mod.3ms}   unclean(of)  dead.body  at-\textsc{kol} these  Q-be.unclean\textsc{.mod.3ms}   \\
    \glt If~one who is~unclean~touches any of these, will it be unclean?  (Hag. 2:13)\\${\neq}$   If~one who is~unclean~touches all of these, will it all be unclean? 
    \ex  
    \gll \textit{kī} \textit{yištaħū}                    \textit{l-aš{}-šɛmɛš  ʔō  l-ay-yārēaħ  ʔō   lə-kol   ṣḇā         haš{}-šāmayīm  …}\\
         if  worship.\textsc{mod.3ms}  to-the-sun  or  to-the-moon  or  to-\textsc{kol} host(of) the-heavens  \textsc{…} \\
    \glt If~[he] worships the sun or moon or any of the host of heaven … (Deut. 17:3)\\${\neq}$   If~[he] worships the sun or moon or all the host of heaven … 
    \z
\z

\ea%44
comparative PPs\label{ex:doron:44}\\
\gll  \textit{wat.tērɛḇ}                     \textit{ma}\textit{ś}\textit{ʔaṯ}           \textit{binyāmin}   \textit{mim-ma}\textit{ś}\textit{ʔōṯ}      \textit{kull-ām} \textit{ħāmēš} \textit{yāḏōṯ}\\
and.was.as.large.3\textsc{fs}  serving.\textsc{fs}(of) Benjamin  as-servings(of) all-\textsc{3mp}  five     portions\\
\glt   but Benjamin’s serving was~five times as much as any of theirs (Gen. 43:34)\\${\neq}$ but Benjamin’s serving was~five times as much as all of theirs
\z

\section{Distributivity in BH}\label{sec:doron:4}%4

In English, \textit{all} is quantificational, and may be interpreted distributively.  \linebreak As shown above, \textit{kol} is a non-quantificational degree determiner in BH, and is not distributive. Distributivity is achieved in BH by other means. Various BH syntactic structures express distributivity through the LF application of the operator \textit{each} (defined by \citealt{Link1987}) to a property:

\ea%45
    \label{ex:doron:45}\relax
    [[each]]  =   λP.λx.${\forall}$y ${\leq}$ x [Atom(y) → P(y)]
\z


We only expect the distributivity operator to modify VPs predicated of a subject \textit{kol} \textit{NP} if the latter is derived by the definite type shift \REF{ex:doron:33a}. Such \textit{kol} \textit{NP} denotes an individual, for which the ${\leq}$  part-of relation is defined. We indeed do not find the distributivity operator when \textit{kol} is interpreted as \textit{any}, by the application of the indefinite type shift \REF{ex:doron:33b}.  

\subsection{The lexical item \textit{ʔīš} ‘each’}%4.1.
In the simplest case, the distributivity operator is expressed by a VP-premodifier, the lexical item \textit{ʔīš} ‘each’ (literally ‘man’), sometimes reduplicated as in \REF{ex:doron:47}:

\ea%46
    \label{ex:doron:46}
    \ea
    \gll \textit{way.yaggīšū}          \textit{kol}    \textit{hā-ʕām}              \textit{ʔīš}     \textit{šōr-ō}                  \\
         and.brought\textsc{.3mp}   \textsc{kol} the-people.\textsc{ms}   each  ox-\textsc{poss.3ms}   \\
    \glt So every one of the people brought his ox.  (1Sam. 14:34)
    \ex  
    \gll \textit{kī}   \textit{kol}    \textit{hā-ʕammim}   \textit{yēlḵū}                   \textit{ʔīš}      \textit{bə-šēm}         \textit{ʔɛlōh-āw} \textit{wa-ʔănaħnū} \textit{nēlēḵ}                \textit{bə-šēm}        \textsc{yhwh}  \textit{ʔɛlōh-ēnū}\\
         for \textsc{kol}   the-peoples   walk.\textsc{mod.3mp}   each   in-name(of)  God-\textsc{poss.3ms} and-we         walk.\textsc{mod.1p}   in-name(of) Lord   God-\textsc{poss.1p}\\
    \glt For all people walk each in the name of his God, but we will walk in the name of the \textsc{Lord} our God. (Mic. 4:5)
    \z
\z


\ea%47
    \label{ex:doron:47}
    \gll \textit{way.yāḇōʔū}      \textit{kol}   \textit{ha-ħăḵāmīm…} \textit{ʔīš}     \textit{ʔīš}    \textit{mim-məlaḵt-ō}             \textit{ʔăšɛr}  \textit{hēmmā}  \textit{ʕōśīm}\\
         and.came\textsc{.3mp}  \textsc{kol} the-experts…   each each from-work- \textsc{poss.3ms} that    they       do.\textsc{ptcp.mp}\\
    \glt Then all the craftsmen … came each from the work he was doing. (Ex. 36:4)
    \z


\subsection{Reduplication}%4.2.
The distributivity operator can also be expressed by reduplicative adverbials, as shown by \citealt{BeckStechow2006}, \citealt{NaudeMillerNaude2015}:

\ea%48
    \label{ex:doron:48}
    \ea
    \gll \textit{way-yittənū}             \textit{ʔēlāw}   \textit{kol}  \textit{nəṣīʔē-hɛm}           \textit{maṭṭɛ} \textit{lə-nāṣī}      \textit{ʔɛħāḏ} \textit{maṭṭɛ} \textit{lə-nāṣī}    \textit{ʔɛħāḏ}\\
         and-give.\textsc{mod.3mp}  to.\textsc{3ms}  \textsc{kol} leaders-\textsc{poss.3mp} rod  for-leader one    rod   for-leader one\\
    \glt and each of their leaders gave him a rod apiece (Num. 17:21)
    \ex  
    \gll \textit{qəħū}                  \textit{lāḵɛm}  \textit{min}   \textit{hā-ʕām}      \textit{šənēm.ʕāśār} \textit{ʔănāšīm} \textit{ʔīš}    \textit{ʔɛħāḏ}  \textit{ʔīš}    \textit{ʔɛħāḏ}  \textit{miš-šāḇɛṭ}\\
         take.\textsc{impr.2mp}  to.2\textsc{mp} from the-people twelve           men       man one      man one    from-tribe\\
    \glt Take for yourselves twelve men from the people, one man from every tribe (Josh. 4:2)
    \ex  
    \gll \textit{middē}       \textit{šānā}  \textit{bə-šānā}\\
         whenever year  in-year \\
    \glt year after year (1Sam. 7:16)
    \ex  
    \gll \textit{bə-ḵol}  \textit{dōr}             \textit{wā-ḏōr}\\
         in-\textsc{kol} generation and-generation     \\
    \glt forever and ever  (Ps. 45:17)  
    \ex  
    \gll \textit{wə.sā\={p}əḏū …       kol  ham-mišpāħōṯ  han-nišʔārōṯ             mišpāħōṯ mišpāħōṯ ləḇāḏ}\\
         mourn.\textsc{mod.3mp}  \textsc{kol} the-families     the-remain.\textsc{ptcp.mp}  families families   alone   \\
    \glt all the families that remain shall mourn, every family by itself (Zech. 12:14)
    \z
\z

\subsection{Floated \textit{kol}}%4.3
Another VP-premodifier which is interpreted as \textit{each} is the inflected \textit{kol}.\footnote{This modification has been called “quantifier float” by \citealt{Shlonsky1991} and Naud\textrm{é 2011b.}} In \REF{ex:doron:49a}, the subject is null and the predicate is modified by \textit{kullām}, i.e. \textit{kol} inflected in the plural. In \REF{ex:doron:49b} the subject is a group individual, and the predicate is again modified by \textit{kullām}. 

\ea%49
    \label{ex:doron:49}
    \ea
    \gll \textit{kull-ām}             \textit{lə-dark-ām}            \textit{pānū}\\
         \textsc{kol-poss.3pl}  to-way-\textsc{poss.3mp}  turned.\textsc{3pl}\\
    \glt They all look to their own way. (Isa. 56:11)
    \ex  
    \gll \textit{wə-kol}     \textit{ṣāray-iḵ}                        \textit{kull-ām}            \textit{b-aš-šəḇī}            \textit{yēlēḵū}\\
         and-\textsc{kol} adversaries-\textsc{poss.2fs}   \textsc{kol-poss.3pl}  in-the-captivity  go.\textsc{mod.3mp} \\
    \glt And all your adversaries, every one of them, shall go into captivity.  (Jer. 30:16)
    \z
\z    

\textit{kol} may also be inflected in the singular, e.g. \textit{ḵullō} in \REF{ex:doron:50}:

\ea%50
    \label{ex:doron:50}
    \gll \textit{ʔɛ\={p}ɛs      qāṣē-hū                      ṯirʔɛ                 wə-ḵull-ō                  lō      ṯirʔɛ}\\
         edge(of) extremity-\textsc{poss.3ms}  see.\textsc{mod.}2\textsc{ms}   and-\textsc{kol}{}-\textsc{poss.}3\textsc{ms}  \textsc{neg}  see.\textsc{mod.}2\textsc{ms}\\
    \glt You shall see the outer part of them [the nation], and shall not see every one of them. (Num. 23:13)
    \z

\subsection{Dependent relational nouns}%4.4

Relational nouns denoting e.g. body-parts, kinship and socially defined roles, or other relations which involve atomic individuals rather than groups, give rise to distributive interpretations when they depend on group nouns. Examples are shown in \REF{ex:doron:51}. Each example includes a dependent relational noun, where the dependence is marked by \textsc{poss} inflection, as in \REF{ex:doron:51} a-b, by the presence of a possessor which raises in LF to yield inverse scope readings, as in \REF{ex:doron:51} c-d, or by the presence of an implicit possessor, as in \REF{ex:doron:51} e-f :

\ea%51
    \label{ex:doron:51}
    \ea
    \gll \textit{kol}   \textit{šōməʕ-ō}                           \textit{təṣillɛnā}             \textit{štē}               \textit{ʔozən-āw}\\
         \textsc{kol} hear.\textsc{ptcp.ms-poss.3ms}  tingle.\textsc{mod.3fp}  both.\textsc{fp}(of)  ear.\textsc{fp-poss.3ms}  \\
    \glt Both ears of everyone who hears it will tingle. (1Sam. 3:11)
    \ex  
    \gll \textit{kol}   \textit{hā-ʔănāšīm}   \textit{hay-yōḏəʕīm}          \textit{kī}      \textit{məqaṭṭərōṯ}      \textit{nəšē-hɛm}             \textit{lē-ʔlōhīm}  \textit{ʔăħērīm}\\
         \textsc{kol} the-men         the-know.\textsc{ptcp.mp} that  fume.\textsc{ptcp.fp}   wives-\textsc{poss.3mp}  to-gods     other\textsc{.mp}\\
    \glt all the men who knew that their wives had burned incense to other gods (Jer. 44:15)
    \ex  
    \gll \textit{ū-ḇə{}-lēḇ              kol   ħăḵam             lēḇ       nāṯattī           ħoḵmā}\\
         and-in-heart(of) \textsc{kol} skilled.\textsc{ms}(of) heart   have.put.\textsc{1s}  wisdom\\
    \glt I have put wisdom in the hearts of all the~gifted artisans. (Ex. 31:6)
    \ex  
    \gll \textit{ū-mik-kol}    \textit{hā-ħay}                 \textit{mik-kol} \textit{bāśār} \textit{šənayīm} \textit{mik-kol}   \textit{tāḇī} \textit{ʔɛl} \textit{hat-tēḇā} \\
         and-of-\textsc{kol} the-live.\textsc{ptcp.ms} of-\textsc{kol-}flesh   two        of-\textsc{kol}    bring.\textsc{mod.2ms} to the-ark\\
    \glt And of every living thing of all flesh you shall bring~two of every~sort~into the ark.~(Gen. 6:19)
    \ex  
    \gll \textit{kī}            \textit{kull-ō}      \textit{ħānē\={p}    ū-mēraʕ          wə-ḵol     pɛ          dōḇēr                nəḇālā}\\
         because  \textsc{kol-3ms} godless and-evildoing  and-\textsc{kol} mouth  speak.\textsc{ptcp.ms}  vileness\\
    \glt for the whole of it [of the nation] was godless and evildoing, every mouth was speaking vile words (NET, Isa. 9:17)
    \ex  
    \gll \textit{ħārū}             \textit{yōšəḇē}             \textit{ʔɛrɛṣ} \textit{…}  \textit{suggar}           \textit{kol}   \textit{bayiṯ}     \textit{mib-bō}\\
         burned.\textsc{3mp}  inhabitants(of) earth…  shut.up.\textsc{3ms}   \textsc{kol}  house  from-come.\textsc{inf}\\
    \glt the inhabitants of the earth are~burned … every house is shut up so that no one may go in (Isa. 24:10)
    \z
\z

The dependence of the relational noun on a group individual gives rise to the introduction of the distributivity operator at the predicate level \citep{Winter2000}. The LFs of (51a-f) can be represented as \REF{ex:doron:52} a-f respectively. The predicate abstracted from the clause which contains the bound x\textsubscript{i} is distributively predicated of the group subject: 

\ea%52
    \label{ex:doron:52}
    \ea\relax [[all who hear it] [each\textsubscript{i} [both ears of x\textsubscript{i} will tingle]]]
    \ex\relax [[all men] [each\textsubscript{i} [x\textsubscript{i} knew that x\textsubscript{i}’s wife had burned incense to other gods]]]
    \ex\relax [[all gifted artisans] [each\textsubscript{i} [I have put wisdom in the heart of x\textsubscript{i}]]]
    \ex\relax [[all kinds] [each\textsubscript{i} [bring two of x\textsubscript{i} into the ark]]]
    \ex\relax […[all of the nation] …] [each\textsubscript{i} [the mouth (of x\textsubscript{i}) was speaking vile words]]
    \ex\relax […[the inhabitants] …] [each\textsubscript{i} [the house (of x\textsubscript{i}) is shut up so that no one enter it]]
    \z
\z


\section{free choice}\label{sec:doron:5}%5
Existential modals such as \textit{may} give rise to what has been called \textit{the} \textit{distribution} \textit{requirement} by \citealt{KratzerShimoyama2002}. This requirement results in a free choice (FC) reading of particular expressions under existential modals.\footnote{According to \citealt{LevFox2017}, ${\lozenge}$(p \textrm{${\vee}$} q) excludes ${\lozenge}$(p \& q) by exhaustivity, but includes ${\lozenge}$p \& ${\lozenge}$q by innocent inclusion, hence implies FC.} We find the same reading for \textit{kol} \textit{NP} in Hebrew. Under an existential modal, \textit{kol} \textit{NP} receives a FC reading, as in the following examples, where \textit{kol} \textit{NP} is satisfied by different individuals in different accessible worlds:

\ea%53
    \label{ex:doron:53}
    \ea
    \gll \textit{wa.ʔăḵaltɛm}          \textit{ʔōṯ-ō}     \textit{bə-ḵol}   \textit{māqōm}  \textit{ʔattɛm}    \textit{ū-ḇēṯ-əḵɛm}\\
         and.eat.\textsc{mod.2mp}  \textsc{acc}{}-it   in-\textsc{kol}  place     you.\textsc{mp}   and-house-\textsc{poss.2mp}\\
    \glt You may eat it in any place, you and your households.  (Num. 18:31)
    \ex  
    \gll \textit{wə-ħēlɛḇ}    \textit{nəḇēlā} \textit{…}  \textit{yēʕāśɛ}                    \textit{lə-ḵol}   \textit{məlāḵā} \textit{wə-ʔāḵōl}      \textit{lō}     \textit{ṯōḵəlu-hū}            \\
         and-fat(of) animal … be.used.\textsc{mod.3ms}  to-\textsc{kol} craft     and-eat.\textsc{inf}  \textsc{neg} eat.\textsc{mod.2mp-acc.3ms}\\
    \glt And the fat of an animal … may be used in any other way; but you shall by no means eat it. (Lev. 7:24)
    \z
\z

The FC reading is also available in the scope of imperative/commissive modal operators (cf. \citealt{Dayal2013}) if \textit{kol} \textit{NP} is modified by a relative clause, as in \REF{ex:doron:54} below. In such examples, \textit{kol} is interpreted as \textit{whatever/} \textit{whoever} and receives a FC interpretation:

\ea%54
    \label{ex:doron:54}
    \ea
    \gll \textit{kol}   \textit{hab-bēn} \textit{hay-yilōḏ}                 \textit{ha-yəʔōr-ā}     \textit{tašlīḵu-hū}\\
         \textsc{kol} the-son  the-born.\textsc{ptcp.ms}  the-river-\textsc{ill}  cast.\textsc{mod.2mp-acc.3ms} \\
    \glt Every son who is born you shall cast into the river. (Ex. 1:22)
    \ex  
    \gll \textit{kol}   \textit{makkē}                 \textit{yeḇūsī}    \textit{b-ā-rīšōnā}     \textit{yihǝyɛ}            \textit{lə-rōš}     \textit{ū-lə-śar}\\
         \textsc{kol} attack.\textsc{ptcp.ms}  Jebusite  in-the-first  be.\textsc{mod.3ms} to-chief  and-to-captain\\
    \glt Whoever attacks the Jebusites first shall be\textsuperscript{} chief and captain. (1Chron. 11:6) 
    \z
\z

A minimal pair is shown in \REF{ex:doron:55}, where \textit{kol+relative} \textit{clause} has a FC interpretation in the commissive \REF{ex:doron:55a}, but merely a collective interpretation in the episodic \REF{ex:doron:55b}:  

\ea%55
    \label{ex:doron:55}
    \ea
    \gll \textit{wə-ḵōl}     \textit{ʔăšɛr}  \textit{tōmar}               \textit{ʔēlay}     \textit{ʔɛʕɛśɛ}\\
         and-\textsc{kol}  that    say.\textsc{mod}.\textsc{2ms}   to.\textsc{1s}     do.\textsc{mod.1s}\\
    \glt and I will do whatever you say to me (Num. 22:17)
    \ex  
    \gll \textit{way.yaʕaś}      \textit{kōl}   \textit{ʔăšɛr}  \textit{ʔāmār} \\
         and.did.\textsc{3ms}   \textsc{kol} that     said.\textsc{3ms}\\
    \glt so [Moses] did all that he had said (Ex. 18:24)
    \z
\z

FC readings have been accounted for by the pervasive view (from \citealt{KadmonLandman1993} to \citealt{Chierchia2013}) that FC items are existential.\footnote{\citealt{Benito2010} and \citealt{Zimmermann2008} treat FC items as universal, but this crucially depends on the distributive interpretation of the universal determiner, which} \textrm{\textit{kol}} \textrm{does not have.} In the case of \textit{kol}, the FC interpretation is due to the indefinite type-shift in \REF{ex:doron:33} b above. Under the present approach, the availability of this type-shift depends on its deriving a stronger reading than the competing definite type-shift. This indeed seems to be the case. If John or Mary may sign a check, then each of them may. But if John and Mary may sign the check, it is not clear they may each sign separately.\footnote{ \textrm{In general, ${\lozenge}$P(a${\vee}$b) ${\rightarrow}$  ${\lozenge}$P(a) \& ${\lozenge}$P(b), but  ${\lozenge}$P(a${\oplus}$}b) -/\textrm{${\rightarrow}$} ${\lozenge}$P(a) \& ${\lozenge}$P(b).}

I assume that the FC interpretation was eventually reanalysed as part of the lexical meaning of \textit{kol}. The change conforms to Eckardt’s (2006: 236) notion of semantic reanalysis – the overall sentence meaning did not change, but there was redistribution of conceptual content: \textit{kol} acquired FC interpretation in the environment of certain modal operators.

\section{Beyond step II}\label{sec:doron:6}%6

In Modern Hebrew (MH), we find that step III of the Distributivity Cycle has occurrred (perhaps as early as Rabbinic Hebrew). The universal determiner \textit{kol} is now interpreted as the distributive \textit{every} in addition to its categorization as \textit{any}:

\ea%56
    \label{ex:doron:56}
    \textit{kol} NP  =  every/any   NP\textsubscript{et} 
\z

I will not discuss step III in the present paper, and rely on Beck’s 2017 account of the development from FC to distributive interpretations. Beck shows how conjunction of the alternative propositions which underlies FC readings develops into universal quantification over individuals.

Moreover, in post-Biblical Hebrew, definite noun phrases are not NPs but are all headed by D; as shown by \citealt{DoronMeir2016}, the article \textit{ha-} was reanalized as a definite determiner of category D. Accordingly, when the complement of \textit{kol} is definite, it is categorized as an individual DP rather than a predicate NP:

\ea%57
    \label{ex:doron:57}
    \textit{kol} DP  =  all  DP\textsubscript{e}
\z

The construction in \REF{ex:doron:57} is definite due to its partitive structure. There isn’t any longer an indefinite type-shift  giving rise to NPI/FC interpretations, not even in downward entailing or modal environments, as shown in \REF{ex:doron:58}.  \REF{ex:doron:58} contrasts with parallel Biblical examples such as \REF{ex:doron:29} – \REF{ex:doron:30} or \REF{ex:doron:43} – \REF{ex:doron:44} above, which have a pseudo-partitive structure, and hence have NPI/FC interpretations. 

\ea%58
    \label{ex:doron:58}
    \ea
    \gll \textit{ha-hanhala}            \textit{lo}     \textit{hitħayḇa}   \textit{le-qabel}     \textit{et}    \textit{kol}    \textit{ha-tlunot} \\
         the-administration \textsc{neg} commited  to-accept  \textsc{acc} \textsc{kol}  the-complaints  \\
    \glt The administration did not commit to accept all/*any complaints. 
    \ex  
    \gll \textit{ha-hanhala}            \textit{hitħayḇa}    \textit{le-qabel}     \textit{et}    \textit{kol}   \textit{ha-tlunot}           \\
         the-administration commited  to-accept  \textsc{acc} \textsc{kol}  the-complaints  \\
    \glt The administration commited to accept all/*any complaints. 
    \z
\z

However, in construction \REF{ex:doron:56}, we do find NPI/FC interpretations, as shown in \REF{ex:doron:59}:

\ea%59
    \label{ex:doron:59}
    \ea
    \gll \textit{ha-hanhala}            \textit{lo}     \textit{hitħayḇa}   \textit{le-qabel}    \textit{kol}    \textit{tluna} \\
         the-administration \textsc{neg} commited  to-accept  \textsc{kol}  complaint   \\
    \glt The administration did not commit to accept every/any complaint.   
    \ex  
    \gll \textit{ha-hanhala}             \textit{hitħayḇa}    \textit{le-qabel}    \textit{kol}    \textit{tluna}          \\
         the-administration  commited  to-accept  \textsc{kol}  complaint  \\
    \glt The administration commited to accept every/any complaint.       
    \z
\z

The Biblical origins of the \textit{any} \textit{NP} construction in \REF{ex:doron:56} are also manifested by the number feature of \textit{any}’s complement within this construction. It is only within this construction that the complement of \textit{kol} can be a plural NP in MH, just like in the Biblical \REF{ex:doron:19a}, (37b-c). The following are MH examples found on the web:~

\ea%60
    \label{ex:doron:60}
    \ea
    \gll \textit{lo}     \textit{nimce’u}     \textit{kol}    \textit{tlunot}              \textit{mucdaqot}\\
         \textsc{neg} found         \textsc{kol}  complaint.\textsc{fp}  justified.\textsc{fp} \\
    \glt There weren’t any justified complaints found.   
    \ex  
    \gll \textit{anu} \textit{mitħayḇim}  \textit{le-facot}            \textit{etḵem}  \textit{begin}  \textit{kol}   \textit{nezaqim}        \textit{še-yaħulu}            \textit{aleyḵem}   \\
         we  commit       to-compensate you       for      \textsc{kol}  damage.\textsc{mp}  that-occur.3\textsc{mp}  on.you \\
    \glt We commit to compensate you for any damages incurred to you.    
    \z
\z

We thus find remnants of Biblical syntax within the MH \REF{ex:doron:56} construction where  \textit{kol} is interpreted as \textit{any}, alongside the new \textit{every} interpretation derived from it. The original definite interpretation of \textit{kol} as \textit{all} is now restricted to the partitive \REF{ex:doron:57} structure. This completes the account of the full array of \textit{kol}’s interpretations in MH.

\section{Conclusion}%7
Hebrew originally lacked a distributive determiner \textit{every}. Distributivity  \linebreak was achieved in Biblical Hebrew through operators applying to the sentence predicate, such as the distributivity operator \textit{each.} Step I of the Distributivity Cycle consisted in the noun \textit{kol} ‘entirety’ grammaticalizing as the collective determiner \textit{all}. The determiner was not quantificational – its combination with a NP yielded the plural property corresponding to NP. In argument position, it was interpreted either by the definite or the indefinite type-shift, depending on which yielded a stronger reading. This gave rise to step II, where \textit{kol} received NPI/FC interpretations in particular environments. It is only at step III that \textit{kol} acquired a distributive interpretation. Modern Hebrew \textit{kol} also retained its previous uses, which accounts for the extensive variation in its interpretations: ‘all/ any/ every’. The paper has shown how these interpretations unfolded along the Distributivity Cycle.

\section*{Aknowledgements}
I am grateful to the following people for helpful comments and suggestions which have greatly benefited the article: Chanan Ariel, Bar Avineri, \linebreak Moshe Elyashiv Bar-Lev, Ido Benbaji, Luka Crnič, Danny Fox, Itamar Francez, Kevin Grasso, Andreas Haida, Robert Holmstedt, Geoffrey Khan, Omri Mayraz, Wendy Sandler, Todd Snider, Ruth Stern, and Yoad Winter. This research has received funding from the Israel Science Foundation grant No. 1296/16  and from the European Research Council H2020 Framework Programme No. 741360.

{\sloppy\printbibliography[heading=subbibliography,notkeyword=this]}
\end{document}
