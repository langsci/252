\documentclass[output=paper
,modfonts
,nonflat]{langsci/langscibook} 

\markuptitle{Vagueness, context-sensitivity and scale structure of four types of adjectives with the suffix \textit{-ish}}{Vagueness, context-sensitivity and scale structure of four types of adjectives with the suffix -ish}
\renewcommand{\lsChapterFooterSize}{\small} %footers in editedvolumes
\renewcommand{\lsCollectionPaperFooterTitle}{Vagueness, context-sensitivity and scale structure of four types of adjectives with the suffix \noexpand\textit{-ish}}
\author{Tabea Harris\affiliation{University of Mannheim}}
% \chapterDOI{} %will be filled in at production

% \epigram{}

\abstract{In English, the adjective-forming suffix \textit{-ish} can be productively tacked onto relative adjectives (e.g. \textit{tall-ish}) and total absolute adjectives (e.g. \textit{dry-ish}), but not to most partial absolute adjectives (e.g. \textit{?bent-ish}) or inherently non-scalar adjectives (e.g. \textit{?pregnant-ish}). By applying \cite{Burnett2017}'s recent framework \textit{DelTCS} situated within \textit{Delineation Semantics}, which she enriched using the notions \textit{Tolerant, Classical, Strict}, first formulated in \cite{Cobreros2012}, I will show why suffixal \textit{-ish} is felicitous with the first two subtypes of adjectives, but not with the latter two. After a brief comparison with a similar framework \cite{Lasersohn1999} it will be shown that the scale structure in the \textit{DelTCS} approach is derived from the adjective's context-sensitivity and vagueness patterns. Furthermore, the discussion will point to a few instances that do not neatly fit into the mold of current semantic analyses as well as some suggestions on how to obtain a clearer picture of the actual attested data.}

\begin{document}
\maketitle
\shorttitlerunninghead{Vagueness, context-sensitivity and scale structure}
\section{Introduction} 
The suffix \textit{-ish} in Present-day English originates from the bound morpheme \textit{-isc} in Old English and denotes associative meaning with nouns (i.e. \textit{N.ish} `having the character/ nature of N', e.g. \textit{baby.ish}) and approximative meaning mostly with adjectives and numerals as bases (i.e. \textit{A.ish / Num.ish} `approaching the quality of A/Num', e.g. \textit{green.ish, 30-ish})\footnote{The notions \textit{associative} and \textit{approximative} in this context originate from \citet*{Traugott2013}.} (cf. the \textit{Oxford English Dictionary} \citet{OED2015} entry for \textit{-ish}1). Unlike nominal bases, which were present in Old English, adjectival bases started to appear in Middle English with color adjectives (e.g. \textit{yellowish} in 1379, cf. \textcite[306]{Marchand1969} \footnote{An anonymous reviewer refers to the results of a large-scale corpus study and states that some relative adjectives (e.g. \textit{thinnish}, \textit{thickish}) have started to appear as \textit{-ish} derivatives at roughly the same time as color adjectives. Unfortunately I have no way of verifying this claim as the reviewer has not disclosed the corpus or corpora used. My own investigation of the \textit{Penn-Helsinki Parsed Corpus of Middle English 2 (\citet{PPCME2})} has not turned up evidence of deadjectival \textit{-ish} adjectives in Middle English. After checking with the MEC \citep{MED-MEC}, formerly \textit{Middle English Dictionary)} and the OED, it can be said that the dates of color adjectives and relative adjectives of the type mentioned above are indeed not far apart: For example, the earliest instances of \textit{greenish}, \textit{yellowish}, \textit{reddish}, and \textit{whitish} are dated in both dictionaries to a1398, while \textit{blackish} and \textit{brownish} first appear around the early 15\textsuperscript{th} century (the OED dates \textit{brownish} to 1555 however). Conversely, \textit{thinnish} and \textit{thickish} appear slightly later in the MEC (a1425) and are dated to the middle of the 16\textsuperscript{th} century in the OED. Thus, depending on the source used, the earliest dates of occurrence will slightly change. It is not wrong to say, however, that deadjectival \textit{-ish} adjectives generally occurred with bases of color at one of the earliest stages.}. Both uses are still highly productive (cf. \textcite[305]{Bauer2013}; \textcite[235]{Dixon2014} ), uses which are attested in several corpora and in many cases already listed in the OED. Thus as a suffix,  \textit{-ish} is quite prolific, attaching to numerous bases, including verbal (e.g.  \textit{snappish}, 1542, see the OED entry for \textit{snappish}), adverbial (e.g.  \textit{nowish}; not listed in the OED yet, but attested in the COCA \citet{Davies2008} once and 42 times in the corpus iWeb, featuring data from countless websites), numeral bases (e.g.  \textit{elevenish} (1916),  \textit{fifty-five-ish} (1941), cf. OED, entry \textit{-ish}, suffix1), and proper names (e.g.  \textit{Heine-ish} (1887), cf. OED, entry \textit{-ish}, suffix1), as well as multi-word units such as compounds (e.g.  \textit{schoolgirlish} 1821, see corresponding OED entry), and phrases (e.g.  \textit{middle-of-the-nightish)}\footnote{BNCweb entry \cite{BNCweb} BMS 1806, Fiction and Verse: \textit{Gate-crashing the dream party}.}. The derivation of ethnic nouns and adjectives (e.g.  \textit{Engl.ish\textsubscript{A/N}}) is no longer productive and will be excluded from consideration here. \textcite{Kuzmack2007}, in \textcite*[234]{Traugott2013} observed that in the associative use,  \textit{-ish} denotes similarity to its base, while the approximative  \textit{-ish} stresses dissimilarity. The focus in this paper will be on the approximative use of  \textit{-ish} that most often occurs with adjectival bases. Consider examples (1) and (2) below:
% Check footnotes - done
\begin{examples}
	\item He was a stout, \textbf{tallish} young man. \\ (GloWbE, US G, http://www.mendele.com/WWD/WWDdead.html)
	\item Mola took his master's hat and gloves at the door, handing him a glass half-filled with a \textbf{greenish} liquid. \\ (COCA, Fiction, \textit{Everfair}, Shawl 2017)
\end{examples}

In both examples the addition of \textit{-ish} to the adjectival bases  \textit{tall} and  \textit{green}, respectively, explicitly marks that the standards set by the adjectives are approximated, but not reached completely.

The picture is complicated by the inherent vagueness of the relative adjective  \textit{tall} in (1) and, I propose, the total absolute adjective  \textit{green} in (2)\footnote{Whether color adjectives are actually classified as relative or as absolute adjectives is a matter of ongoing debate and the results thus far have eluded a clear picture (cf. \citet*{Hansen2017} for an experimental approach). \citet{Burnett2012a, Burnett2012b} considers them to fall into the relative camp due to the (syntactic) tests that are felicitous with them.}. How can we know that a person counts as tall if we do not have a standard to which we can compare that person? And how do we determine such a standard? Where is the minimum threshold above which an object can be considered tall? These are questions frequently discussed in the literature about vagueness and vague adjectives in particular\footnote{For instance, concerning the determination of the standard degree of tallness, \citet {von Stechow1984} proposes the positive operator \textit{Pos} which aims at giving a unified treatment of polar opposites such as \textit{tall-short}. I will not go into further detail here, the interested reader is kindly referred to von Stechow's work (e.g. \citeyear{vonStechow1984,vonStechow2009}). See also \citet*{Kennedy2005} and \citet{Kennedy2007} who employ the operator \textit{pos} in their frameworks.}. They address the central problem of vague predicates: determining borderline cases, fuzzy boundaries and the classical paradox of the Sorites, i.e. if we continuously add one centimeter to a building of average height, at one point we have to admit that it is tall. We do encounter the problem of not being able to say at which point exactly the building has reached the threshold and can unambiguously be considered tall, which is due to the incremental fashion of adding to the height of the building.
Now consider examples (3) and (4) below:

\begin{examples}
	\item The \textbf{wettish}, sticky cement floor sent chills all the way up to her temples. (COCA, Fiction,  \textit{The Evidence}, Qi 2005)
	\item My grass is all thin and \textbf{dead-ish}, what is your advice on overseeding? (GloWbE, US G, http://richsoil.com/lawn-care.jsp.)
\end{examples}

Example (3) features a partial absolute adjective  \textit{wet}, which acquires the meaning `less than fully wet' when \textit{-ish} is attached. Similarly, in (4) the inherently non-scalar adjective \textit{dead} is given a gradable meaning with \textit{-ish} and denotes that the lawn is not yet totally beyond repair, but in a state that requires (professional) help. Note here that only 7 tokens were found for \textit{wettish} in the \textit{Corpus of Contemporary American English} (COCA) and only 2 tokens for \textit{dead-ish} (and none for \textit{deadish}\footnote{The fact that the spelling of \textit{dead-ish} contains a hyphen can be an indicator that its usage is unusual and marked for the speaker. It is often found in cases with hiatus (e.g. \textit{country-ish}), when the base word consists of an abbreviation (e.g. \textit{Espn-ish}, \textit{Cia-ish}), after certain numerals (e.g. \textit{23-ish}), and frequently after proper names (e.g. \textit{Verne-ish}). One phonological reason why \textit{-ish} nevertheless attaches to \textit{dead} (but not to \textit{pregnant}, \textit{hexagonal}, \textit{illegal}, etc., which are also all non-scalar) is that the suffix primarily selects monosyllabic bases (cf. \textcite[235]{Dixon2014}. While it can easily be shown that this is not a constraint, preference of monosyllabic bases should be understood here in the sense of frequency.}) in the corpus \textit{Global Web-based English} ( GloWbE \citep{Davies2013}). Both corpora are considered representative and balanced and feature 560 million words ( COCA \citep{Davies2008}) and a considerable 1.9 billion words (GloWbE), respectively. Compared to the adjectives \textit{tallish} (43 tokens in GloWbE) and \textit{greenish} (751 tokens in COCA), both \textit{wettish} and \textit{dead-ish} are virtually non-existent by comparison. Of course, the usual caveats for corpus-analytic studies apply. Since the aim here is not to provide a full-fledged corpus study, it will suffice to say that the preliminary results of the corpora indicate that \textit{-ish} does not easily attach to adjectives that are partially absolute or non-scalar\footnote{In order to conduct a `proper' corpus analysis, among other things, we would need to expand the class of adjectives to include an equal number of each subclass which is then compared in each of the corpora mentioned above.}.

The paper is structured as follows: Section 2 will introduce the four adjectival subtypes mentioned earlier, section 3 will give an introduction to the \textit{Tolerant, Classical, Strict} framework that is employed in \citet{Burnett2017}. In this respect section 3 features an analysis of the scale structure of the four different subtypes of adjectives with \textit{-ish} and will encompass a discussion of why relative and total absolute adjectives are productive with the suffix, whereas partial absolute and non-scalar adjectives are infelicitous in most cases. Section 5 will conclude the findings and point to further areas of research.

\section{Four subclasses of adjectives}
\label{sec:four-adj}
In the classification of four subgroups of adjectives, I follow \citet{Burnett2017} and others who propose adjectives like \textit{tall}, \textit{expensive} and \textit{cheap} as belonging to Relative Adjectives (RA), \textit{empty}, \textit{clean}, \textit{straight} and others as being part of Total Absolute Adjectives (AA\textsuperscript{T}), \textit{dirty}, \textit{bent} and \textit{wet} as included in the Partial Absolute Adjective class (AA\textsuperscript{P}) and, finally, adjectives such as \textit{pregnant}, \textit{dead}, or \textit{hexagonal} which belong to the Non-Scalar adjective type (NS) (cf. \textcite[4]{Burnett2017} \footnote{The distinction between total and partial adjectives is said to have originated from \citet{Yoon1996} (cf. \textcite*[355]{Kennedy2005}.} The distinction into relative and absolute adjectives is nothing new, just as the observation that the absolute class incorporates two distinct subclasses is well established (see, for example, \citet*{Rotstein2004}; \citet*{ Kennedy2005}; \citet*{Kennedy2007}; \citet{Toledo2011}. For instance, \citet{Kennedy2005} have investigated the RA-AA distinction with respect to the felicity of degree modifiers (e.g. \textit{slightly},  \textit{perfectly},  \textit{completely}) and \textit{for-}phrases, the latter of which explicitly specifies the contextual domain that determines the standard degree and which are natural for relative adjectives, but odd for absolute adjectives. They follow \citet{Unger1975}, who claims that only relative adjectives are context-sensitive and vague.
A different view is proposed by \citet*{Rotstein2004}, who investigate the total-partial distinction for gradable adjectives and who contend that absolute adjectives can \textit{also} be context-sensitive and vague when they are used in 'loose talk' (cf. \citet{Lewis1979}. In these cases, the standard for total/partial absolute adjectives is not necessarily the exact maximal or minimal degree on a scale. Evidence for this view is provided by the use of the degree modifier \textit{completely}, which selects absolute standards. For instance, consider the `maximality modifier' \textit{completely} as discussed in \citet{Kennedy2005} and \cite*{Rotstein2004}, among others. In (5) below, when \textit{completely} modifies the AA\textsuperscript{T} \textit{clean}, the adjective conveys the maximal amount of cleanliness, where the floor is so immaculate as to be able to literally eat from it. 

\begin{examples}
	\item The floor is not clean, it is \textit{completely} clean.
\end{examples}

\citet*{Toledo2011} maintain that both accounts have their merits, but neither is entirely able to account for all the facts p.140. They propose an approach which takes comparison classes (CCs) into account, an approach which also finds application in \citet{Burnett2017} concerning comparison-class based context sensitivity.
Relative adjectives show a more substantial form of both context-sensitivity and vagueness than absolute adjectives in that they exhibit universal \textit{and} existential context-sensitivity (AAs only show the latter) and in that they permit potentially vague positive \textit{and} negative forms (AAs show a different distribution depending on the subtype) (cf. \citet{Burnett2017}. The notions above require some explanation. Universal context-sensitivity, according to \textcite[41]{Burnett2017}, corresponds to the adjective's ability to shift its thresholds in any comparison class. \textcite[28]{Kennedy2007} and others have suggested to employ a definite description test to distinguish whether an adjective is able to shift its standard of comparison:

\begin{examples}
	\item Show me the expensive one.
	\item Show me the green one.
\end{examples}

According to this test, the adjective in (6) would independently be considered true of two objects (e.g. watches) outside of the test, but has been shown to pertain to the \textit{more} expensive one in the context of this utterance. In (7) however, when both objects (e.g. sweaters) are either both green or not green, uttering (7) becomes infelicitous. Absolute adjectives are therefore not considered to be universally context-sensitive, but rather existentially so. They appear to be able to shift their criteria of application in some comparison classes, but do so only when they appear in the context of `imprecision' (cf. \textcite[42]{Burnett2017}. Hence, when the absolute adjective \textit{green} is used loosely, modification with a \textit{for}-phrase becomes more acceptable:

\begin{examples}
	\item This lawn is green for midsummer in Texas.
\end{examples}

Example (8) is acceptable when we describe a lawn that only shows a few brown patches here and there, but is not completely dead. The same utterance would seem infelicitous in a context in which the lawn is completely lush green.

By using the notion \textit{potentially vague}, \textcite[49]{Burnett2017} aims at showing that the vagueness of relative adjectives and the vagueness in the case of `loose' uses of absolute adjectives is due to a single underlying phenomenon. In her conception, vagueness is a stage-level property that is subject to contextual variation, and potential vagueness is defined as follows:

\begin{examples}
	\item An adjective \textit{P} is \textit{potentially vague} iff there is some context \textit{c} such that \textit{P} gives rise to a Soritical argument in \textit{c} (\textcite[50]{Burnett2017}, emphasis in original). % Check citations - done
\end{examples}

The potential vagueness just described is not distributed symmetrically over absolute adjectives. While relative adjectives have potentially vague positive (P) and negative forms ($\neg$P), total absolute adjectives only have positive potentially vague forms (P), and partial absolute adjectives display the opposite pattern, i.e. they only have negative potentially vague forms ($\neg$P). In order to illuminate the relationship, consider the following examples: % Check neg operator - done

\begin{examples}
	\item (RA) \textit{Expensive}: For all x, y, if x is expensive, and x and y's cost differ by one monetary unit (i.e. a Dollar, a Euro, etc.), then y is expensive {$\rightarrow$ \0} positive potential vagueness
	\item (RA) \textit{Expensive}: For all x, y, if x is not expensive and x and y's cost differ by one monetary unit (see above), then y is not expensive {$\rightarrow$ \0} negative potential vagueness
	\item (AA\textsuperscript{T}) \textit{Empty}: For all x, y if x is empty, and x and y's contents differ by a single item, then y is empty {$\rightarrow$ \0} positive potential vagueness
	\item (AA\textsuperscript{T}) \textit{Empty}: For all x, y, if x is not empty and x and y's contents differ by a single item, then y is not empty {$\rightarrow$ \0} \textbf{False} (no negative potential vagueness)
	\item (AA\textsuperscript{P}) \textit{Dirty}: For all x, y, if x is dirty, and x and y differ by one single stain, then y is dirty {$\rightarrow$ \0} \textbf{False} (no positive potential vagueness)
	\item (AA\textsuperscript{P}) \textit{Dirty}: For all x, y, if x is not dirty, and x and y differ by one single stain, then y is not dirty {$\rightarrow$ \0} negative potential vagueness
\end{examples} % Check the arrows - done

The examples just given require some clarification. In order to illustrate (12), consider two containers, one of which is entirely empty (\textit{x}), the other of which holds exactly one bean (\textit{y}). In a tolerant use of the adjective, both containers would be judged empty. Conversely, container \textit{x} in example (13) holds a single bean, whereas \textit{y} contains no bean at all. Thus, container \textit{y} is not judged $\neg$\textit{empty} by comparison and the principle of tolerance does not hold in this case (cf. \textcite[51]{Burnett2017}. Let us now turn to the examples illustrating AA\textsuperscript{P}s, i.e. adjectives typically associated with scales that have minimal elements. Example (14) claims intolerance concerning the positive \textit{dirty}: An object \textit{y} that is completely clean will not be considered dirty, even if object \textit{x} only differs in having one stain. The negative form \textit{not dirty} satisfies tolerance however, i.e. it is P-vague. Consider a situation where Mary wants to paint her living room. In that case, choosing an outfit which has a speck of dirt on it will be considered as being not dirty, i.e. in the context of painting, a single stain on the chosen outfit will be perceived as irrelevant (cf. \textcite[52]{Burnett2017}.

In sum, relative adjectives are symmetrically vague in that they do not discriminate between positive and negative forms (hence, they are potentially vague with either form of the predicate). Absolute adjectives, by contrast, exhibit an asymmetric distribution of vagueness. Total absolute adjectives are tolerant with positive forms, but intolerant when it comes to distinguishing individuals situated at the bottom end of some scale from those that are at the second to last degree: In \textcite[52]{Burnett2017} example, \textit{empty} is infelicitous with \textit{x} containing a single spectator (in a theater, for instance), and \textit{y} having no spectator at all. In other words, if \textit{x} is \textit{not empty} and differs by only one individual from \textit{y}, \textit{y} cannot be considered tolerantly true in this context.
With partial absolute adjectives, the picture is reversed. They exhibit negative potential vagueness in that they are tolerant with respect to individuals at the lower endpoint of a scale (15), but display no potential vagueness with corresponding positive forms (14).

Finally, non-scalar adjectives exhibit neither context-sensitivity \linebreak nor potentially vague forms in their precise uses. Burnett however claims that they can be turned into scalar absolute adjectives when they assume gradable interpretations (\citeyear{Burnett2017}:44). In a `loose' use of \textit{pregnant}, for instance, we can see the properties of a gradable, context-sensitive adjective: 

\begin{examples}
	\item Sue is very pregnant, for being in the third month.
\end{examples}

Example (16) is only felicitous when we assume that Sue's pregnancy is already much more showing in her third month when compared to other women. \textcite[44]{Burnett2017} observes that the scalar modifier \textit{very} facilitates this gradable use of the adjective. In assuming that the non-scalar adjective has been coerced into an absolute adjective, Burnett is able to keep the semantic class of non-scalar adjectives `pure' and can claim that as such they are not context-sensitive and non-gradable. She assumes that the distinction of AAs and NSs is of a pragmatic nature, i.e. a shifting operation in the level of precision with which the adjective is used (2017:95). This assumption involves the view that both AAs and NSs have precise semantic denotations, but are variable with respect to their pragmatic denotations. Applying this line of reasoning to the workings of complex words addresses a desideratum for morphological theory formulated in \textcite[226]{Plag1999}. Specifically, it takes the study of pragmatics into account and investigates the effect context has on the use of complex words.\footnote{I thank the anonymous reviewer who pointed me to their study.} Exactly what this perspective entails will be the subject of the following sections.

\section{The \textit{Delineation Tolerant, Classical, Strict} framework}
\label{sec:deltcs} %Check the label names

The semantics of \textit{-ish} has already been discussed within a degree semantics framework by \citet*{Bochnak2014}. They observed that \textit{-ish} is felicitous with adjectives containing an open scale or those exhibiting an upper bound (i.e. a maximal value), but are questionable with adjectives which contain a lower bounded scale \textcite[435--436]{Bochnak2014}, i.e. relative adjectives, total absolute adjectives, and partial absolute adjectives, respectively. The few cases where \textit{-ish} attaches to non-scalar adjectives like \textit{dead} are not discussed in their framework.

In the present paper, I propose an analysis in an alternative framework based on \citet*{Cobreros2012} notion of \textit{Tolerant, Classical, Strict} (henceforth \textit{TCS}), which has been applied to vague adjectives in the recent framework of \citet{Burnett2017}. The idea to formalize vague predicates in a trivalent non-classical logic stems from the fact that ``the principle of tolerance gives rise to the sorites paradox'' in classical logic \textcite[348]{Cobreros2012}. That is, in order to allow a truth value which makes reference to tolerance as exemplified in (17) a third value has to be introduced. % Check quotation marks - done

\begin{examples}
	\item If some individual \textit{x} is \textit{P}, and \textit{x} and \textit{y} are only imperceptibly different in respects relevant for the application of the predicate \textit{P}, then \textit{y} is \textit{P} as well \textcite[348]{Cobreros2012}.
\end{examples}

This general idea has been implemented in different ways, as for example in the pragmatically oriented framework by \citet{Lasersohn1999}, which assigns a pragmatic halo around expressions such as the following:

\begin{examples}
	\item Mary arrived at three o'clock.
\end{examples}

The time expression is taken to be close enough to truth in case of Mary arriving 15 seconds after three o'clock, for instance. The expression in (18) is assigned a denotation under which it is true and additionally contains a set of times that `differ from the denotation only in some respect that is pragmatically ignorable in context' \textcite[526]{Lasersohn1999}. This set of times is then understood as the pragmatic halo of the expression in (18) which is at the halo's center. \textcite[29]{Burnett2017} notes that the \textit{Halo} framework was not originally designed as a semantic theory of vagueness, but observes that the basic ideas are very similar to hers (and the model of \textit{TCS} in general). In particular, the notion of \textit{tolerant truth} is taken to be the equivalent to Lasersohn's \textit{close enough to truth} \textcite[32]{Burnett2017}. Further, what is described as \textit{pragmatically ignorable} in Lasersohn's framework is paralleled by a notion of \textit{indifference} in \textit{TCS} \textcite[32]{Burnett2017}. Thus, both frameworks include the ``core intuition that at least one aspect of vagueness/pragmatic slack involves loosening the conditions of application of an expression with a precise semantic denotation to include other objects that are considered to differ in only `pragmatically ignorable' ways'' \textcite[32--33]{Burnett2017}. However, even though the two frameworks may superficially be understood as mirror images of each other, Burnett suggests the \textit{TCS} framework to constitute a refinement of pragmatic halos. Specifically, in \textit{TCS} it is possible to derive non-classical denotations and orderings, which are simply given in the model by \citet{Lasersohn1999} (cf. \textcite[33]{Burnett2017}).

Burnett's \textit{Delineation Tolerant, Classical, Strict} (\textit{DelTCS} in short) is part of a class of comparison-class-based semantic frameworks going back to 
\citet{Klein1980}. The aim of her model is to provide a new relationship between the vagueness of adjectival predicates, the properties of context-sensitivity that these predicates involve as well as their corresponding scale structure. Her logical framework, based on indifference relations, ``preserves the intuition that vague predicates are tolerant, but avoids the Sorites paradox'' by a step-wise validation of tolerance \textcite[28]{Burnett2017}. Indifference relations refer to change that is marginal enough to not make a difference to categorization of a predicate (i.e. one Euro more or less will not make a watch expensive or not expensive as compared to another watch which exhibits the value P or $\neg$P, respectively) (cf. \textcite[1]{Burnett2017}). Within the TCS extension of the Delineation framework, she assumes that the classical semantic framework is enriched with tolerant/strict denotations, which are established as a second step (\textcite[72]{Burnett2017}, cf. also \citet{Cobreros2012}). In doing so, she adds the function $\sim$, which ``maps a predicate and a comparison class to an indifference relation on the members of the class'' (\citeyear{Burnett2017}:72). Applied to the pattern of potential vagueness of the AA\textsuperscript{T} \textit{empty} (see examples (12) and (13) above) we obtain the following distribution (cf. \textcite[76]{Burnett2017}):
\begin{examples}
	\item Container \textit{x} with no bean in it $\sim$\textsubscript{\textit{empty}} container \textit{y} with one bean in it
	\item Container \textit{x} with one bean in it $\not\sim$\textsubscript{\textit{empty}} container \textit{y} with no bean in it
\end{examples}

The definition for a tolerant model according to \textcite[72]{Burnett2017} is given below:
% Check for packages (denotations) - done
\begin{examples}
	\item Tolerant model: For all P and all X $\subseteq$ D, $\sim_P^X$ is a binary relation on X.
\end{examples}

The novelty and difference to \citet{Cobreros2012} lies in the fact that a predicate \textit{and} a CC are mapped to the indifference relation $\sim$ (cf. \textcite[72]{Burnett2017}). 
She splits the denotational system in half, assuming the classical denotations to be semantic and the secondary tolerant and strict denotations to be pragmatic in nature, the latter of which are formally defined in (22) and (23) (cf. \textcite[73]{Burnett2017}):

\begin{examples}
	\item Tolerant denotation: \[ \llbracket P \rrbracket_X^t	= \{x : \exists d \sim_P^X  x : d \in \llbracket P\rrbracket x\}\text{.}\]
	\item Strict denotation: \[\llbracket P\rrbracket_X^s  = \{ { x :  \forall d \sim _P^X  x, d  \in   \llbracket P \rrbracket x } \}\text{.}\]
\end{examples}

By adopting the view that tolerant and strict denotations are pragmatic in nature, she presents a solution to the paradox of absolute adjectives which become gradable through a derivational process from comparison-class-based (existential) context-sensitivity that is essentially pragmatic, not semantic as in the classical interpretation (\citeyear{Burnett2017}:89).

The vague and context-sensitive properties that adjectival predicates possess are modeled by assuming a series of constraints that pertain to their different distributions. For the present purposes I will restrict the discussion of constraints to those that differ for the four subgroups of adjectives. Hence, given that relative adjectives are potentially vague with P and $\neg$P, they are symmetrical in their indifference relations, whereas absolute adjectives display an asymmetric distribution (see above), which is encoded into their indifference relations\footnote{A note on vocabulary: a\textsubscript{1}, a\textsubscript{2}, a\textsubscript{n} refers to individual constants, Q refers to AA\textsuperscript{T}, R to AA\textsuperscript{P}. For a comprehensive vocabulary of Burnett's model, see her p. 56.}, as shown below (cf. \textcite[77]{Burnett2017}):

\begin{examples}
	\item Total axiom: If $\llbracket$Q\textsubscript{1}(a\textsubscript{1})$\rrbracket$\textsubscript{M,D} = 1 and $\llbracket$Q\textsubscript{1}(a\textsubscript{2})$\rrbracket$\textsubscript{M,D} = 0, then a\textsubscript{2} $ \not\sim_{Q_{1}}^X  a_{1}, for \, all \, X \subseteq D.$
	\item Partial axiom: If $\llbracket$R\textsubscript{1}(a\textsubscript{1})$\rrbracket$\textsubscript{M,D} = 1 and $\llbracket$R\textsubscript{1}(a\textsubscript{2})$\rrbracket$\textsubscript{M,D} = 0, then a\textsubscript{1} $ \not\sim_{R_{1}}^X  a_{2}, for \, all \, X \subseteq D.$
\end{examples}

The axioms ensure that total absolute adjectives are identical in their classical and strict denotations, and partial absolute adjectives have identical classical and tolerant denotations. Since non-scalar adjectives do not show potential vagueness, it is assumed that P and $\neg$P are both precise, i.e. their classical (semantic) denotations coincide with their pragmatic denotations, which is ensured by the pragmatic constraint \textit{Be precise} (cf. \textcite[77--78]{Burnett2017}).

The assumptions for context-sensitivity patterns follow in a straightforward manner. Given that relative adjectives exhibit both universal and existential \linebreak context-sensitivity, their denotations are much less restricted in variation depending on a comparison class. Thus, relative adjectives are consistently felicitous with \textit{for-}phrases. Consider the following example, which depicts the assessment to an exchange about the prices of two bottles of wine, one of which costs \$130 and another which is of a more affordable price of \$17. The  \textit{for-}phrase characterizing the less expensive of the two is fully felicitous with the RA  \textit{cheap}: 

\begin{examples}
	\item Gifford: Seventeen, that's cheap for a bottle... \\ (COCA, Spoken, NBC Today: \textit{Today's talk}, 27.04.2011)
\end{examples}
	
Absolute adjectives have been shown to only allow for existential context-sensitivity and their denotations vary according to subtype: Total absolute adjectives have context-sensitive tolerant denotations (the classical and strict denotations are identical across comparison class, see above), partial absolute adjectives involve corresponding strict denotations. Non-scalar adjectives do not show variation in their tolerant and strict denotations since they are subject to the pragmatic constraint \textit{Be precise} (cf. \textcite[85]{Burnett2017}). The denotations associated with these patterns are given below (cf. \textcite[85]{Burnett2017}, slightly adapted):

\begin{examples}
	\item AA\textsuperscript{T}: For all X $\subseteq$ D, \[ \llbracket Q_{T} \rrbracket_X^s = \llbracket Q_{T}\rrbracket_{X}\text{.}\]
	\item AA\textsuperscript{P}: For all X $\subseteq$ D, \[\llbracket R_{P}\rrbracket_X^t = \llbracket R_{P}\rrbracket_{X}\text{.}\]
\end{examples}

The patterns can give an explanation for why AAs become more acceptable with  \textit{for-}phrases: The  \textit{for-}phrase specifies explicitly the content of a given comparison class. \textcite[86]{Burnett2017} holds that the informational content CCs contribute to the interpretation of absolute adjectives is non-trivial and informative in their tolerant or strict uses. Thus while the semantic denotations remain fixed across contexts, sentences like (8) above improve when they are used loosely.
Since the scale structure is derived from the context-sensitivity patterns associated with the different classes of adjectives, corresponding effects are predicted. The scale structure properties will be subject of the section below.

\section{Scale structure and the suffix \textit{-ish} with adjectives in \textit{DelTCS}}
\label{scale-ish-deltcs}

As we have seen in the examples at the beginning, the suffix \textit{-ish} felicitously selects those adjectives that fall in the classes of RAs and AA\textsuperscript{T}s, respectively (i.e. \textit{tallish}, \textit{greenish}), but is rather infrequent with the other two classes, i.e. AA\textsuperscript{P}s and NSs, even if we find a number of attestations.
In \citet*{Bochnak2014} framework, it was assumed that the reason why \textit{-ish} was felicitous with open-scale adjectives (i.e. RAs) and those that feature a maximum value (i.e. AA\textsuperscript{T}s) was due to the fact that they pick out a degree that is slightly less than a contextually given standard (\citeyear{Bochnak2014}:436). With adjectives that make reference to a lower-bounded scale, this option is not given, since they are already situated at the lower end of the scale and hence cannot be below the minimum standard.

In the \textit{DelTCS} framework introduced above, scales are derived from the adjective's corresponding denotations. Relative adjectives show both universal and existential context-sensitivity and are potentially vague for the positive form of a predicate as well as for its negation. Their scales are assumed to be derived from their semantic denotations, i.e. they have neither maximal nor minimal elements (cf. \textcite[90,107]{Burnett2017}), which corresponds to Bochnak and Csipak's open scale. Total absolute adjectives were shown to be potentially vague for the positive form and to exhibit existential context-sensitivity. They are associated with scales that are derived from their tolerant denotations, i.e. their scales exhibit maximal elements, whereas the opposite is true for partial absolute adjectives (\citeyear{Burnett2017}:90,106). AA\textsuperscript{P}s exhibited an asymmetry in their potential vagueness pattern that showed the reverse, i.e. they were considered to be (potentially) vague for the negation of the predicate ($\neg$P). Their scales are correspondingly derived from their strict denotations and thus contain minimal elements (\citeyear{Burnett2017}:90,106). Since the pragmatic denotations (i.e. the tolerant and strict denotations) of non-scalar adjectives are identical with their semantic denotations, they will not be associated with any scales (\citeyear{Burnett2017}:90). Burnett however notes an exception to this observation. Hence, non-scalar adjectives can be coerced into scalar predicates, making them subject to the same constraints that hold for absolute adjectives. In that case, they do not follow the axiom \textit{Be precise}, which has the consequence that they can be analyzed in the same way as absolute adjectives. Here, Burnett's approach is very different from other frameworks in that she assumes non-scalar adjectives to actually be absolute adjectives which are used with a higher degree of precision (\citeyear{Burnett2017}:97-98). In other words, the difference between AAs and NSs is manifested in their pragmatic denotations, not their semantic ones (cf. \textcite[98]{Burnett2017} and it depends on how an adjective is used in a particular context: In a precise use, a non-scalar adjective will not allow for variable meaning by the constraint \textit{Be precise}, whereas in a context that allows for a lower level of precision, the NS is coerced and becomes context-sensitive, i.e. showing the characteristics of a vague and gradable predicate. For instance example (16) above, which was used in a context in which a non-scalar adjective (i.e. \textit{pregnant}) can turn into a gradable one with the help of a \textit{for-}phrase, shows that these adjectives can be used felicitously with a gradable meaning given that the context allows for a loose use. In these cases, NSs are also quite natural in comparative constructions (example from \textcite[96]{Burnett2017}:

\begin{examples}
	\item Sarah is \textit{more pregnant} than Sue; Sarah is showing more.
\end{examples}

What do these assumptions mean for the suffix \textit{-ish} and its application? For RAs and both AAs the account can be laid out in a quite straightforward manner. The scale structures show the same characteristics as in \citet*{Bochnak2014} and \textit{-ish} targets these scales. In doing so, it lowers the precision with which the adjectives are used, making them available for `loose' use. The difference in Burnett's \citeyearpar{Burnett2017} account is that the scales are derived from the adjective's context-sensitivity. Recall that with relative adjectives, both universal and existential context-sensitivity is possible. This allows RAs to be associated with an open scale that shows neither maximal nor minimal elements (as for instance with \textit{tall}). With its approximative meaning, \textit{-ish} can approach the quality instantiated by the adjective (e.g. \textit{tall-ish}), but does not reach it in full measure. The predictions concerning potential vagueness can be explained by the indifference relations of relative adjectives, which were said to be symmetric, i.e. they are potentially vague with both P and $\neg$P. With \textit{tallish} (i.e. P) and \textit{not tallish} (i.e. $\neg$P), we can observe that the respective antonyms \textit{shortish} and \textit{not shortish} are also relative adjectives, i.e. the two adjectival types display a mirror image of each other.

Total absolute adjectives were shown to be associated with scales derived from their tolerant denotations, i.e. they exhibited maximal elements and only have potential vagueness with their positive forms. That is, the positive forms of \textit{clean}, \textit{dry}, or \textit{straight} can be targeted by \textit{-ish}, again with \textit{-ish} adding the meaning of approximation to the positive form. Their antonyms \textit{dirty}, \textit{wet}, or \textit{bent}, however are all partial absolute adjectives, which are associated with scales that have minimal elements which are derived from their strict denotations. These adjectives show potentially vague negative forms ($\neg$P) and cannot be targeted as easily by \textit{-ish} (cf. also \textcite[437]{Bochnak2014}. We can thus say that the applicability with \textit{-ish} correlates with the type of adjective and their associated properties (with a few exceptions, see example (3) above).

Since non-scalar adjectives generally occur in contexts that favor precise uses (cf. \textcite[95--96]{Burnett2017}), they are hardly found with \textit{-ish}. However, as we have seen in (4) above, in some cases they can be turned into an absolute adjective when the standard of precision with which they are used is loosened. This is what happened with the NS \textit{dead}. In example (4) above, \textit{-ish} is able to attach to \textit{dead} because the adjective's conditions of application have been loosened. In this case, it features an upper bound that is approached by \textit{-ish}, i.e. it has a maximal element as is generally the case with total absolute adjectives. \textcite[112--113]{Burnett2017} notes however that many coerced non-scalar adjectives are able to be associated with both types of scale structure, i.e. those that have maximal and those that have minimal elements, which her examples show, given slightly adapted in (30) to (32), respectively.

\begin{examples}
	\item DEA agent 1: Bring me up to speed on Tuco Salamanca. \\
		DEA agent 2: Dead. \\
		DEA agent 1: Still? \\
		DEA agent 2: \textit{Completely}. \\
		(\textit{Breaking Bad} 2009. Season 2, episode 5, ``Breakage.'')
	\item The coma patient is \textit{almost} dead.
	\item 'Dead Person is Actually Only \textit{Slightly} Dead.' \\ 
		(Headline from http://www.therobotsvoice.com)
\end{examples}

Examples (30) and (31) illustrate the coerced NS \textit{dead} with a scale associated with maximal endpoints (i.e. characteristic of AA\textsuperscript{T}s), whereas (32) is an example for a partial absolute adjective, indicated by the modifier \textit{slightly}. Thus, \textit{dead} may be coerced into either type of adjective, depending on context. The NS \textit{dead} is not exclusive in showing both patterns. \textcite[113]{Burnett2017} gives ethnic adjectives such as \textit{Canadian} and the non-scalar adjective \textit{illegal} as further examples. The exact conditions that are responsible for an NS to select which end of a scale remain to be elucidated.

The findings of attachability for \textit{-ish} should not be taken to be absolute. For example, we find cases of relative adjectives that so far have not been attested with \textit{-ish} (e.g. \textit{?intelligent-ish}), whereas some non-scalar adjectives occur quite freely with \textit{-ish} (e.g. \textit{squarish}, \textit{dead-ish} to some extent). Thus, there has to be a further factor that plays a role in the applicability of \textit{-ish} that has not yet been accounted for. One factor that immediately comes to mind is a non-semantic one. It has to do with the syllable structure of adjectives that are favored by \textit{-ish}. As has been mentioned above, \textit{-ish} preferably attaches to monosyllabic bases which is true for \textit{squarish} and \textit{dead-ish}, but not for \textit{?intelligent-ish}. This might be a factor which rather concerns productivity of \textit{-ish} derivatives however, and not so much whether they are generally well-formed and felicitous as we do of course find bases with more than one syllable to which \textit{-ish} attaches. Thus, this factor alone will undoubtedly not be sufficient to account for the patterns we have encountered with \textit{-ish}, but it could be seen as a contributing factor.

\section{Conclusion}
\label{sec:summary}

This paper has investigated the English suffix \textit{-ish} with the four subtypes of adjectives that are discussed in the literature. In doing so, the \textit{Delineation Tolerant, Classical, Strict} framework recently implemented by \citet{Burnett2017} was used. It has been found that the patterns \textit{-ish} shows with different types of adjectives can felicitously be described in a framework in which the scale structure of adjectives is derived by the patterns of context-sensitivity they depict. Context-sensitivity thus correlates with the patterns found for potential vagueness and scales. By mapping tolerant and strict pragmatic denotations on basic classical (semantic) ones, the framework approaches issues of vagueness from a more pragmatic angle than other well-known frameworks such as \textit{Degree Semantics}.

It has further been noted that suffixal \textit{-ish} does not show an absolute fit concerning its productivity patterns with different adjectival types. Therefore, it has been suggested to look for further (e.g. phonological, but presumably also other semantic) factors in conjunction with the semantic ones introduced above to explain these divergent occurrences of \textit{-ish}. To be sure, the general tendency of productive derivations with a certain type of base is not disputed, but the counterexamples mentioned above should nevertheless be accounted for in a semantic theory, even though they amount to only a few attestations. However, rather than dismissing them as rare and subscribing to the view that unless frequent (counter-)examples of a phenomenon are attested, the phenomenon is non-existent, I take the attestations that deviate from the general pattern as being deemed acceptable enough by speakers in particular contexts. In order to make this claim stronger, a full-fledged corpus analysis could be devised in order to see which forms are actually attested with which frequencies and in what contexts they are found to occur.

%\newpage

%\section*{References}
% % \begin{thebibliography}{}
% % \bibitem{Bauer2013} Bauer, Laurie; Rochelle Lieber \& Ingo Plag. 2013. \textit{The Oxford Reference Guide to English Morphology}. Oxford University Press. \\
% % 
% % \bibitem{Bochnak2014} Bochnak, M. Ryan \& Eva Csipak. 2014. A new metalinguistic degree morpheme. \textit{Proceedings of Semantics and Linguistic Theory (SALT)} 24. 432-452. \\
% % 
% % \bibitem{Burnett2012a} Burnett, Heather. 2012a. (A)symmetric Vagueness and the Absolute/Relative} Distinction. In A. Aguilar Guevara, A. Chernilovskaya \& R. Nouwen (eds.), \textit{Proceedings of Sinn und Bedeutung}, vol. 16. MIT Working Papers in Linguistics. \\
% % 
% % \bibitem{Burnett2012b} Burnett, Heather. 2012b. The Puzzle(s) of Absolute Adjectives. On Vagueness, Comparison, and the Origin of Scale Structure. \textit{UCLA Working Papers in Linguistics} 16. 1-50. \\
% % 
% % \bibitem{Burnett2017} Burnett, Heather. 2017. \textit{Gradability in Natural Language: Logical and Grammatical Foundations}. Oxford University Press. \\
% % 
% % \bibitem{Cobreros2012} Cobreros, Pablo; Paul Egr\?{e}, David Ripley \& Robert van Rooij. 2012. Tolerant, classical, strict. \textit{Journal of Philosophical Logic} 41. 347-385. \\
% % 
% % \bibitem{Davies2008} Davies, Mark. 2008-. \textit{The Corpus of Contemporary American English (COCA)}. https://english-corpora.org/coca/. \\
% % 
% % \bibitem{Davies2013} Davies, Mark. 2013. \textit{Corpus of Global Web-Based English (GloWbE)}. https://english-corpora.org/glowbe/. \\
% % 
% % \bibitem{Davies2018} Davies, Mark. 2018-. \textit{The 14 Billion Word iWeb Corpus}. https://english-corpora.org/iWeb/. \\
% % 
% % \bibitem{Dixon2014} Dixon, Robert M. W. 2014. \textit{Making new words: Morphological derivation in English}. Oxford University Press. \\
% % 
% % \bibitem{Hansen2017} Hansen, Nat \& Emmanuel Chemla. 2017. Color adjectives, standards, and thresholds: an experimental investigation. \textit{Linguistics and Philosophy} 40(3). 239-278. \\
% % 
% % \bibitem{Kennedy2007} Kennedy, Christopher. 2007. Vagueness and grammar: The study of relative and absolute gradable predicates. \textit{Linguistics and Philosophy} 30(1). 1-45. \\
% % 
% % \bibitem{Kennedy2005} Kennedy, Christopher \& Louise McNally. 2005. Scale structure, degree modification, and the semantics of gradable predicates. \textit{Language} 81. 345-381. \\
% % 
% % \bibitem{Klein1980} Klein, Ewan. 1980. A semantics for positive and comparative adjectives. \textit{Linguistics and Philosophy} 4. 1-45. \\
% % 
% % \bibitem{PPCME2} Kroch, Anthony \& Ann Taylor. 2000. \textit{The Penn-Helsinki Parsed Corpus of Middle English, Second Edition (PPCME2)}. Philadelphia: University of Pennsylvania. \\
% % 
% % \bibitem{Kuzmack2007} Kuzmack, Stephanie. 2007. Ish: A new case of antigrammaticalization. Paper given at the 2007 LSA Annual Meeting, 4 January 2007. Anaheim, CA. \\
% % 
% % \bibitem{Lasersohn1999} Lasersohn, Peter. 1999. Pragmatic halos. \textit{Linguistics and Philosophy} 75. 522-571. \\
% % 
% % \bibitem{BNCweb} Lehmann, Hans-Martin; Peter Schneider \& Sebastian Hoffmann. 2000. BNCweb. In: John Kirk (ed.). \textit{Corpora Galore: Analysis and Techniques in Describing English}. Amsterdam: Rodopi. 259-266. \\
% % 
% % \bibitem{Lewis1979} Lewis, David. 1979. Scorekeeping in a language game. \textit{Journal of Philosophical Logic} 8(1). 339-359. \\
% % 
% % \bibitem{MED-MEC} McSparran, Frances; Paul Schaffner, John Latta, Alan Pagliere, Christina Powell \& Matt Stoeffler. 2000-2018. \textit{Middle English Dictionary. Online edition in Middle English Compendium}. Ann Arbor: University of Michigan Press. \\
% % 
% % \bibitem{OED2017} OED Online. 2017. \textit{-ish, suffix1}. Oxford University Press. http://www.oed.com/view/Entry/99965. \\
% % 
% % \bibitem{Plagetal} Plag, Ingo; Christiane Dalton-Puffer \& Harald Baayen. 1999. Morphological productivity across speech and writing. \textit{English Language and Linguistics} 3(2). 209-228. \\
% % 
% % \bibitem{Rotstein2004} Rotstein, Carmen \& Yoad Winter. 2004. Total Adjectives vs. Partial Adjectives: Scale Structure and Higher-Order Modifiers. \textit{Natural Language Semantics} 12. 259-288. \\
% % 
% % \bibitem{Toledo2011} Toledo, Assaf \& Galit Sassoon. 2011. Absolute vs. Relative Adjectives: Variation Within vs. Between Individuals. \textit{Proceedings of Semantics and Linguistic Theory (SALT)} 21. 135-154. \\
% % 
% % \bibitem{Traugott2013} Traugott, Elizabeth C. \& Graeme Trousdale. 2013. \textit{Constructionalization and Constructional Changes}. Oxford University Press. \\
% % 
% % \bibitem{Unger1975} Unger, Peter. 1975. \textit{Ignorance: A Case for Scepticism}. Oxford: Clarendon Press. \\
% % 
% % \bibitem{vonStechow1984} von Stechow, Arnim. 1984. Comparing semantic theories of comparison. \textit{Journal of Semantics} 3. 1-77. \\
% % 
% % \bibitem{vonStechow2009} von Stechow, Arnim. 2009. The temporal adjectives fr�h(er)/ sp�t(er) 'early(er)'/ 'late(r)' and the semantics of the positive. In: Anastasia Giannakidou \& Monika Rathert (eds.). \textit{Quantification, Definiteness and Nominalization}. Oxford: Oxford University Press. 214-233. \\
% % 
% % \bibitem{Yoon1996} Yoon, Youngeun. 1996. Total and partial predicates and the weak and strong interpretations. \textit{Natural Language Semantics} 4. 217-236. \\
% % \end{thebibliography}

 % Check references - done

{\sloppy\printbibliography[heading=subbibliography,notkeyword=this]}
\end{document}
