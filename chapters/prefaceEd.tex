\documentclass[output=paper]{langsci/langscibook}
\ChapterDOI{10.5281/zenodo.3929235}
\title{Editors' preface}
\author{Remus Gergel\affiliation{Universität des Saarlandes}\lastand Jonathan Watkins\affiliation{Universität des Saarlandes}}
\abstract{\noabstract}
\begin{document}
\maketitle

\noindent The articles for the current volume have emerged from presentations delivered at the second edition of Formal Diachronic Semantics, held at Saarland University, Saarbrücken from November 21--22, 2017. The conference featured key-note addresses by Ashwini Deo, Chiara Gianollo, and an integrated invited student presentation by Verena Hehl along with a fair amount of high-quality scholarly work that was accepted for presentation.

The majority of the contributions delivered to the conference revolved around topics of quantification and scales in the process of semantic change. This is a nice coincidence which we strove to incorporate in a volume. At the same time, it was also quite clear from the onset that quantification and scalarity in naturally attested case studies rarely appear as clear-cut as in idealized textbook trajectories. This led to the current compendium with Language Science Press for which a subset of the papers presented at the conference was submitted and included in accordance with standard review and revision procedures. In the remainder of this preface, we offer a brief outlook on what the readers can expect from the contributions contained within. While the order of the articles in the volume is alphabetical, we introduce the papers from a thematic point of view.

The papers by Doron, Jędrzejowski, Kellert and Simonenko \& Carlier address – to varying degrees – issues pertaining to phenomena from language change, which are standardly treated in terms of quantification and therefore are hoped to be of interest, either in terms of data or analysis, to researchers concerned with the respective sub-topics, of which we will give a slightly longer outlook below. By contrast, contributions by Gianollo, Harris, and Kopf-Giammanco are primarily concerned with scalar structures of different kinds and how they have evolved over time. Before introducing the individual articles, it seems of note to mention that all of the contributions presented contain a fair amount of discussion on the interfaces of interpretation – be it through structural facts, the pragmatic component (including e.g. information-structural factors), or important morphological factors.\largerpage

Doron’s article addresses some of the fundamental questions in the trajectories which arise in the domain of universal quantification (see \citet{Fintel95theformal,Haspelmath1995}, \cite{Beck2017} with a specific case study conducted on Biblical Hebrew. The general trajectory of meanings is roughly paraphrasable as ‘all/ any/ every’ (notice, however, that the original Hebrew noun kol had both similarities, but also key differences from English alI) and it is couched in terms of a cyclical view of language change (cf. \citet{Gelderen2011} , see e.g. \citet{Gergel2016} for an application of cyclical theory to issues of interpretation). While Doron’s paper concentrates on the development from collective to distributive readings, it also sheds light on the so-called Distributivity Cycle as a whole and its role in the history of a language which is proposed originally not to have had any distributive determiner of the ‘every’ type (but other alternative mechanisms to express such meanings).

While Doron’s focus is on the development of universal interpretations, Simonenko \& Carlier’s contribution incorporates the interaction of what they regard as a non-presuppositional existential inference with variation in the constituent order facts in the history of French. A key component of the analysis is a version of *New > Given principle of \citet{Kucerova:2012}. The authors attribute the principle to a binding configuration, specifically to how situation binding operates in clauses. The resulting account and the quantitative data obtained should be of interest to researchers in the diachrony of French syntax and how it could be modeled in tandem with pragmatic factors such as givenness. The diachronic span covered ranges from the twelfth to the seventeenth century.

Kellert’s paper relates to the large topic of indefinites of indifference. She investigates the linguistic item called \textit {equis} (x) in Mexican Spanish by giving a synchronic description and offering a diachronic suggestion. While the original meaning of the expression is that of the letter x, a discourse-related function is claimed to have appeared very recently.  According to the discourse function, \textit{equis} is used to refer to some utterance from the discourse which denotes a proposition and the speaker expresses her indifference as to whether this proposition is true or not. Descriptively, \textit{equis} is claimed to have developed into a discourse adverb. The key idea beyond the diachronic analysis is that the language has undergone a shift in the domain of indifference associated with the word under discussion, namely from indifference with respect to the identities of entities towards indifference with respect to answers to questions under discussion. The latter meaning is claimed to be lexicalized, while the initial is taken to have been pragmatic. The syntactic correlate is suggested to reside in the reanalysis from a nominal modifier to a sentential one.\largerpage

Jędrzejowski addresses the topic of modality, another domain classically\linebreak
treated as quantificational, namely over possible worlds, in semantic theory. His focus in the paper is on the morphosyntactic facts related to the appearance of an interesting clausal evidential marker in the history of Polish.  The key argument is that the word \textit{jakoby} developed from an original complementizer, with the meaning ‘as if’, into a hearsay complementizer. The case study laid out in the paper offers empirical evidence supporting the idea that the process happened around the 1500s, i.e. in the late Old Polish period. Jędrzejowski claims that the original presence of what he takes to be an equative comparison and the counterfactual meaning were the decisive factors in realizing the semantic reanalysis.

At the center of Giannollo’s paper is the topic of scalar alternatives. She focuses on the Latin focus-sensitive negation \textit{nec} (`furthermore not'; `neither'; `not even') and suggests a trajectory from a discourse-structuring particle with an additive component to a new emphatic (scalar) negative polarity item, which is later reanalyzed as an element of negative concord. The larger question in the background has to do with the issues of scope and the cyclicity of semantic change; cf. Lehmann’s parameters of grammaticalization, which Gianollo re-evaluates with respect to semantic changes. The key proposal is tied to the way alternatives are retrieved from the context and the idea that an increase in bondedness and a decrease in syntagmatic variability correlate with a change in the form taken by alternatives, which decrease in scope from discourse units to individual alternatives.

Harris directly addresses classical degree scales and the issues posed by establishing the precise type of scale structure when it comes to the application of the affix –\textit{ish} in English. By applying \citegen{Burnett2017} framework situated within Delineation Semantics (cf. \citealt{Cobreros2012}) she proposes to account for the distribution of the affix. While certain attested corpus examples observed by Harris are left for further research, her main claim is that the relevant scale structure is derived from the adjective’s context-sensitivity and vagueness patterns.

Kopf-Giammanco’s article combines the issues of degree-scales (including, but not restricted to, the temporal ones) with the topic of presuppositions by focusing on the semantics of present day German \textit{noch} in comparative readings (cf. \citealt{Beck2019} for a recent overview and synchronic analysis). He presents both experimental synchronic and diachronic corpus-based evidence from Old High German suggesting a reanalysis of the particular reading of \textit{noch} under investigation based on temporal readings.

The assorted treatment of quantification across Biblical Hebrew, Old French, Mexican Spanish, and Polish, to issues of scalarity in Latin, English and German offers (researchers) first and foremost a broad linguistic research palate. By delving into quantification and scales the contributors to this volume shed light on both specific subfields as well as as well as on the way interpretations can change.  By addressing quantification from universal interpretations through to modality and scalartiy from scalar alternatives to degree scales, the volume lends itself nicely to anyone with research interests in semantic and pragmatic change at the interfaces. The papers in this volume help us once more to deepen our understanding of those particular languages. They show the diversity of a growing field and at the same time offer perspective towards the more general enterprise of  understanding mechanisms of semantic change.

{\sloppy\printbibliography[heading=subbibliography,notkeyword=this]}
\end{document}
