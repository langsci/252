\author{Remus Gergel\lastand Jonathan Watkins}
\title{Quantification and scales in change}  
% \subtitle{Change your subtitle in localmetadata.tex}
% \BackTitle{Change your backtitle in localmetadata.tex} % Change if BackTitle is different from Title

\renewcommand{\lsSeries}{tgdi} % use lowercase acronym, e.g. sidl, eotms, tgdi
\renewcommand{\lsSeriesNumber}{} %will be assigned when the book enters the proofreading stage

\BackBody{This volume contains thematic papers on semantic change which emerged from the second edition of Formal Diachronic Semantics held at Saarland University. Its authorship ranges from established scholars in the field of language change to advanced PhD students whose contributions have equally qualified and have been selected after a two-step peer-review process.

The key foci are variablity and diachronic trajectories in scale structures and quantification, but readers will also find a variety of further (and clearly non-disjoint) issues covered  including reference, modality, givenness, presuppositions, alternatives in language change, temporality, epistemic indefiniteness, as well as -- in more general terms -- the interfaces of semantics with syntax, pragmatics and morphology. 

Given the nature of the field, the contributions are primarily based on original corpus studies (in one case also on synchronic experimental data) and present a series of new findings and theoretical analyses of several languages, first and foremost from  the Germanic, Romance, and Slavic subbranches of Indo-European (English, French, German, Italian, Polish, Spanish) and from Semitic (with an analysis of universal quantification in Biblical Hebrew).}

%\dedication{Change dedication in localmetadata.tex}
%\typesetter{Change typesetter in localmetadata.tex}
%\proofreader{Change proofreaders in localmetadata.tex}

\renewcommand{\lsID}{252} % contact the coordinator for the right number
%\BookDOI{}%ask coordinator for DOI
\renewcommand{\lsISBNdigital}{000-0-000000-00-0}
\renewcommand{\lsISBNhardcover}{000-0-000000-00-0} 
